% mathe.tex - Mathematische (Mengen)Zeichen, Symbole und Funktionen
%             fuer LaTeX 2.09 und LaTeX 2e
%
%             Original von Dieter Boss, 08/94
%             Letzte Aenderung: H.Schmidt, 26-April-2000 (NDFT ... NIFFT)
%

\newcommand{\ds}[1]{\displaystyle{#1}}    % Kurzschreibweise fuer \displaystyle
\newcommand{\ml}[1]{\hbox{\large $#1$}}   % math large : Angenehme Schrift-
                                          % groesse bei der Darstellung von
                                          % Bruechen z.B. \ml{1\over\sqrt{2}}

% #####  Haeufig benoetigte Funktionen  #####
\newcommand{\E}[1]{\ensuremath{\mathrm{E}\left\{#1\right\}}}
\newcommand{\real}[1]{\ensuremath{\mathrm{Re}\left\{#1\right\}}}
\newcommand{\imag}[1]{\ensuremath{\mathrm{Im}\left\{#1\right\}}}
\newcommand{\rect}[1]{\ensuremath{\mathrm{rect}\left(#1\right)}}
\newcommand{\tri}[1]{\ensuremath{\mathrm{tri}\left(#1\right)}}
\newcommand{\si}{\ensuremath{\mathrm{si}}}
\newcommand{\di}{\ensuremath{\mathrm{di}}}
\newcommand{\ld}{\ensuremath{\mathrm{ld}}}  
\newcommand{\erf}{\ensuremath{\mathrm{erf}}}
\newcommand{\erfc}{\ensuremath{\mathrm{erfc}}}
\newcommand{\eds}[1]{\ensuremath{\mbox{e }^{\ds{#1}}}}
\newcommand{\ex}[1]{\ensuremath{e^{#1}}}
\newcommand{\ejO}{\ex{j\Omega}}

% #####  Mathematische Sonderzeichen  #####
\newcommand{\defas}{\ensuremath{\stackrel{\Delta}{=}}}

% #####  Mengenzeichen  #####
\newcommand{\Reell}{\mathsf{I} \kern -0.15em \mathsf{R}} 
\newcommand{\Nat}{\mathsf{I}  \kern -0.15em \mathsf{N}}
\newcommand{\Feld}{\mathsf{I} \kern -0.15em \mathsf{F}}
\newcommand{\Zahl}{\mathsf{Z} \kern -0.45em \mathsf{Z}}

% ##  Korrespondenz-"Knochen": ##
\newcommand{\korrespond}{\ensuremath{\;\circ \hskip-1ex -\hskip-1.2ex -\hskip-1.2ex- \hskip-1ex \bullet\;}}
\newcommand{\ikorrespond}{\ensuremath{\;\bullet \hskip-1ex -\hskip-1.2ex -\hskip-1.2ex- \hskip-1ex \circ\;}}

% #####  Transformationen  #####
\newcommand{\FT}[1]{\ensuremath{{\cal F}\left\{#1\right\}}}        % (kont.) Fourier-Trafo
\newcommand{\IFT}[1]{\ensuremath{{\cal F}^{-1}\left\{#1\right\}}}
\newcommand{\HT}[1]{\ensuremath{{\cal H}\left\{#1\right\}}}        % Hilbert-Trafo
\newcommand{\IHT}[1]{\ensuremath{{\cal H}^{-1}\left\{#1\right\}}}
\newcommand{\LT}[1]{\ensuremath{{\cal L}\left\{#1\right\}}}        % Laplace-Trafo
\newcommand{\ILT}[1]{\ensuremath{{\cal L}^{-1}\left\{#1\right\}}}
\newcommand{\DFT}[1]{\ensuremath{\mathrm{DFT}\left\{#1\right\}}}   % Diskrete Fourier-Trafo
\newcommand{\IDFT}[1]{\ensuremath{\mathrm{IDFT}\left\{#1\right\}}}
\newcommand{\FFT}[1]{\ensuremath{\mathrm{FFT}\left\{#1\right\}}}   % Fast Fourier-Trafo
\newcommand{\IFFT}[1]{\ensuremath{\mathrm{IFFT}\left\{#1\right\}}}
\newcommand{\NDFT}[2]{\ensuremath{\mathrm{DFT}_{#2}\left\{#1\right\}}}   % Diskrete Fourier-Trafo
\newcommand{\NIDFT}[2]{\ensuremath{\mathrm{IDFT}_{#2}\left\{#1\right\}}}
\newcommand{\NFFT}[2]{\ensuremath{\mathrm{FFT}_{#2}\left\{#1\right\}}}   % Fast Fourier-Trafo
\newcommand{\NIFFT}[2]{\ensuremath{\mathrm{IFFT}_{#2}\left\{#1\right\}}}\newcommand{\ZT}[1]{\ensuremath{{\cal Z}\left\{#1\right\}}}        % Z-Trafo
\newcommand{\IZT}[1]{\ensuremath{{\cal Z}^{-1}\left\{#1\right\}}}
\newcommand{\DTFT}[1]{\ensuremath{\mathrm{DTFT}\left\{#1\right\}}}   % Diskrete Time Fourier-Trafo
\newcommand{\IDTFT}[1]{\ensuremath{\mathrm{IDTFT}\left\{#1\right\}}}

% #####  Einheiten und Groessen  #####
\newcommand{\Hz}{\ensuremath{\mathrm{\:Hz}}}
\newcommand{\kHz}{\ensuremath{\mathrm{\:kHz}}}
\newcommand{\MHz}{\ensuremath{\mathrm{\:MHz}}}
\newcommand{\Mbits}{\ensuremath{\mathrm{\:Mbit/s}}}
\newcommand{\GHz}{\ensuremath{\mathrm{\:GHz}}}
\newcommand{\ms}{\ensuremath{\mathrm{\:ms}}}
\newcommand{\ns}{\ensuremath{\mathrm{\:ns}}}
\newcommand{\mus}{\ensuremath{\mathrm{\:\mu s}}}
\newcommand{\kmh}{\ensuremath{\mathrm{\:km/h}}}
\newcommand{\dB}{\ensuremath{\mathrm{\:dB}}}
\newcommand{\kbits}{\ensuremath{\mathrm{\:kbit/s}}}
\newcommand{\kBaud}{\ensuremath{\mathrm{\:kBaud}}}
\newcommand{\SNR}{\ensuremath{\frac{S}{N}}}
\newcommand{\EbN}{\ensuremath{\frac{E_b}{N_0}}}
\newcommand{\EbNh}{\ensuremath{\frac{E_b}{N_0/2}}}

% #####  Worte, die haeufig in Gleichungen gebraucht werden  #####
\newcommand{\Mit}{\quad\mathrm{mit}\;\,}          % kleingeschrieben existiert \mit schon!
\newcommand{\und}{\quad\mathrm{und}\;\,}
\newcommand{\da}{\quad\mathrm{da}\;\,}
\newcommand{\fuer}{\quad\mathrm{f"ur}\;\,}
\newcommand{\wobei}{\quad\mathrm{wobei}\;\,}
\newcommand{\mindex}[1]{\mbox{\scriptsize \sl #1}}

% #####  Fettschrift fuer Vektoren  #####
\newcommand{\vek}[1]{\ensuremath{\mathbf{#1}}}    % (fette) Vektoren oder Matrizen (mit Buchstabe als Argument)
\newcommand{\bs}[1]{\mbox{\boldmath$#1$}}         % (fette) schraege Vektoren oder Matrizen



% ############################################################################################
% Aus Kompatibilitaetsgruenden wird die folgende Datei an dieser Stelle eingebunden. Die dort
% aufgefuehrten Befehle sollten nicht mehr verwendet werden, weil alle dortigen \newcommand's
% kuenftig NICHT mehr unterstuetzt werden!!
% ############################################################################################
% mathecomp.tex - Mathematische (Mengen)Zeichen, Symbole und Funktionen
%                 fuer LaTeX 2.09 und LaTeX 2e.
%                 Kompatibilitaetsdatei als Anhang von mathe.tex.
%                 
%
%             Original von Dirk Nikolai, 01/96
%             Letzte Aenderung: Nikolai, 25/01/96
%

% #################################################################################################
% Alles folgende ist nur aus Kompatibilitaetsgruenden in mathe.tex eingebunden und sollte deshalb 
% in der Zukunft nicht mehr verwendet werden, weil alle folgenden \newcommand's kuenftig NICHT 
% weiter unterstuetzt werden!!
% #################################################################################################

\newcommand{\realu}[2]{\mathrm{Re}#1\{#2#1\}} % muss statt \real verwendet werden, falls das
                                              % Argument \underbrace o.ae. enthaelt, #1 ist eine
                                              % zulaessige TeX-Groessenangabe, #2 der 
                                              % eigentliche Text

% Definition der mathematischen Zahlenmengen (Symbole fuer die natuerl., reellen etc. Zahlen) (Thomas Boltze)
% Damit sind solche Dinge wie ... \in \R^{m \times n+1} mit kniffligen Definitionen von \R passe.
% Die neuen Befehle lauten z.B. "\In{R}[m][n+1]" oder "\In{C}[n]"
\newcommand{\C}{\mathrm C \hspace{-0.43em}\rule[0.1ex]{0.02em}{1.45ex}\hspace{0.58em}}
\newcommand{\R}{\mathrm R \hspace{-0.44em}\rule[0.05ex]{0.02em}{1.45ex}\hspace{0.5em}}
\newcommand{\N}{\mathrm N \hspace{-0.44em}\rule[0.05ex]{0.02em}{1.45ex}\hspace{0.5em}}
\newcommand{\Z}{\mathrm Z \hspace{-0.44em}/\hspace{0.58em}}
\mathchardef\inn="3232                    % Verr"ucken des \in Zeichens nach oben
\def\in{\raisebox{0.2ex}{$\inn$}}
\catcode`\@=11                            % ein wenig Tex-Gewurstel f"ur die optionalen Parameter:
\def\room#1{\@ifnextchar[{\@nameuse{#1}\@roomarg}{\@nameuse{#1}}}
\def\@roomarg[#1]{\@ifnextchar[{\@@roomarg{#1}}{^#1}}
\def\@@roomarg#1[#2]{^{#1 \times #2}}
\catcode`\@=12
\def\In#1{\raisebox{0.2ex}{$\inn$} \room{#1}}

% \newcommand{\matC}{\makebox[.6em]{\makebox[-.18em]{C}\rule{.03em}{1.5ex}}}
\newcommand{\matC}{\mathrm C \hspace{-0.43em}\rule[0.1ex]{0.02em}{1.45ex}\hspace{0.58em}}
\newcommand{\matRR}{\mathrm R \hspace{-0.44em}\rule[0.05ex]{0.02em}{1.45ex}\hspace{0.5em}}

\newcommand{\matN}{\mathsf{I\!N}}         % Setze die math. Sonderzeichen zusammen
\newcommand{\matR}{\mathsf{I\!R}}         % aus Sans Serif-Zeichen.
\newcommand{\matZ}{\mathsf{Z\!\!\!\!\:Z}} % Problem:  "C" fehlt.


\newcommand{\nullvec}{\mathbf{0}}   % (fette) Vektoren oder Matrizen (jeder einzeln...)
\newcommand{\avec}{\mathbf{a}}
\newcommand{\Avec}{\mathbf{A}}
\newcommand{\bvec}{\mathbf{b}}
\newcommand{\Bvec}{\mathbf{B}}
\newcommand{\cvec}{\mathbf{c}}
\newcommand{\Cvec}{\mathbf{C}}
\newcommand{\dvec}{\mathbf{d}}
\newcommand{\Dvec}{\mathbf{D}}
\newcommand{\evec}{\mathbf{e}}
\newcommand{\Evec}{\mathbf{E}}
\newcommand{\fvec}{\mathbf{f}}
\newcommand{\Fvec}{\mathbf{F}}
\newcommand{\gvec}{\mathbf{g}}
\newcommand{\Gvec}{\mathbf{G}}
\newcommand{\hvec}{\mathbf{h}}
\newcommand{\Hvec}{\mathbf{H}}
\newcommand{\ivec}{\mathbf{i}}
\newcommand{\Ivec}{\mathbf{I}}
\newcommand{\lvec}{\mathrm{\mathbf $\ell$}}
\newcommand{\Mvec}{\mathbf{M}}
\newcommand{\Nvec}{\mathbf{N}}
\newcommand{\Pvec}{\mathbf{P}}
\newcommand{\Qvec}{\mathbf{Q}}
\newcommand{\rvec}{\mathbf{r}}
\newcommand{\Rvec}{\mathbf{R}}
\newcommand{\svec}{\mathbf{s}}
\newcommand{\Svec}{\mathbf{S}}
\newcommand{\Tvec}{\mathbf{T}}
\newcommand{\uvec}{\mathbf{u}}
\newcommand{\Uvec}{\mathbf{U}}
\newcommand{\vvec}{\mathbf{v}}
\newcommand{\Vvec}{\mathbf{V}}
\newcommand{\wvec}{\mathbf{w}}
\newcommand{\Wvec}{\mathbf{W}}
\newcommand{\xvec}{\mathbf{x}}
\newcommand{\Xvec}{\mathbf{X}}
\newcommand{\yvec}{\mathbf{y}}
\newcommand{\Yvec}{\mathbf{Y}}
\newcommand{\zvec}{\mathbf{z}}
\newcommand{\Zvec}{\mathbf{Z}}


% EOF


% EOF
