% #####################################################################################
% ##       latex2e.tex - LaTeX 2e-specific Definitions  by  Dieter Boss, 01/96       ##
% ##  -----------------------------------------------------------------------------  ##
% ##  The following commands are defined appropriately:                              ##
% ##  - \comment{}      : Multiple line comments                                     ##
% ##  - \bild{file}{1.0}: Include graphics and scale in multiples of "\textwidth"    ##
% ##  - \centerbild{...}: Include centered graphics                                  ##
% ##  - \bfm            : Bold face in math mode                                     ##
% ##  - \bf, \sf ...    : can be applied simultaneously (and work together!)         ##
% ##  -----------------------------------------------------------------------------  ##
% ##  Example of a LaTeX2e document header:                                          ##
% ##    \documentclass[twocolumn]{article}                                           ##
% ##    \usepackage{rotate,colordvi}  % Swap "colordvi" for "blackdvi" when printing ##
% ##    \usepackage[english]{babel}   %     a color LaTeX document on a B/W printer. ##
% ##    \usepackage{options,my9pt}    % ... and "showkeys" to view names of labels.  ##
% ##    % #####################################################################################
% ##       latex2e.tex - LaTeX 2e-specific Definitions  by  Dieter Boss, 01/96       ##
% ##  -----------------------------------------------------------------------------  ##
% ##  The following commands are defined appropriately:                              ##
% ##  - \comment{}      : Multiple line comments                                     ##
% ##  - \bild{file}{1.0}: Include graphics and scale in multiples of "\textwidth"    ##
% ##  - \centerbild{...}: Include centered graphics                                  ##
% ##  - \bfm            : Bold face in math mode                                     ##
% ##  - \bf, \sf ...    : can be applied simultaneously (and work together!)         ##
% ##  -----------------------------------------------------------------------------  ##
% ##  Example of a LaTeX2e document header:                                          ##
% ##    \documentclass[twocolumn]{article}                                           ##
% ##    \usepackage{rotate,colordvi}  % Swap "colordvi" for "blackdvi" when printing ##
% ##    \usepackage[english]{babel}   %     a color LaTeX document on a B/W printer. ##
% ##    \usepackage{options,my9pt}    % ... and "showkeys" to view names of labels.  ##
% ##    % #####################################################################################
% ##       latex2e.tex - LaTeX 2e-specific Definitions  by  Dieter Boss, 01/96       ##
% ##  -----------------------------------------------------------------------------  ##
% ##  The following commands are defined appropriately:                              ##
% ##  - \comment{}      : Multiple line comments                                     ##
% ##  - \bild{file}{1.0}: Include graphics and scale in multiples of "\textwidth"    ##
% ##  - \centerbild{...}: Include centered graphics                                  ##
% ##  - \bfm            : Bold face in math mode                                     ##
% ##  - \bf, \sf ...    : can be applied simultaneously (and work together!)         ##
% ##  -----------------------------------------------------------------------------  ##
% ##  Example of a LaTeX2e document header:                                          ##
% ##    \documentclass[twocolumn]{article}                                           ##
% ##    \usepackage{rotate,colordvi}  % Swap "colordvi" for "blackdvi" when printing ##
% ##    \usepackage[english]{babel}   %     a color LaTeX document on a B/W printer. ##
% ##    \usepackage{options,my9pt}    % ... and "showkeys" to view names of labels.  ##
% ##    % #####################################################################################
% ##       latex2e.tex - LaTeX 2e-specific Definitions  by  Dieter Boss, 01/96       ##
% ##  -----------------------------------------------------------------------------  ##
% ##  The following commands are defined appropriately:                              ##
% ##  - \comment{}      : Multiple line comments                                     ##
% ##  - \bild{file}{1.0}: Include graphics and scale in multiples of "\textwidth"    ##
% ##  - \centerbild{...}: Include centered graphics                                  ##
% ##  - \bfm            : Bold face in math mode                                     ##
% ##  - \bf, \sf ...    : can be applied simultaneously (and work together!)         ##
% ##  -----------------------------------------------------------------------------  ##
% ##  Example of a LaTeX2e document header:                                          ##
% ##    \documentclass[twocolumn]{article}                                           ##
% ##    \usepackage{rotate,colordvi}  % Swap "colordvi" for "blackdvi" when printing ##
% ##    \usepackage[english]{babel}   %     a color LaTeX document on a B/W printer. ##
% ##    \usepackage{options,my9pt}    % ... and "showkeys" to view names of labels.  ##
% ##    \input{latex2e.tex}           % Include this file.                           ##
% ##    \input{mathe.tex}             % Must be included AFTER this file!            ##
% #####################################################################################

% Komfortables Auskommentieren ganzer LaTeX-Passagen
\newcommand{\comment}[1]{}

% Grafikeinbindung
\usepackage[dvips]{graphicx}
\newcommand{\bildwi}[2]{\includegraphics[width=#2\textwidth]{#1}}
\newcommand{\bildsc}[2]{\includegraphics[scale=#2]{#1}}
\newcommand{\bild}[2]{\includegraphics[width=#2\textwidth]{#1}}  % for compatibility
\newcommand{\centerbild}[1]{\centerline{#1}}                     % for compatibility

% Fettschrift im Mathemodus
\newcommand{\bfm}[1]{\mathbf{#1}}

% Definiere \bf, \sf etc. so um, dass sie sich bei gleichzeitiger Anwendung 
% sich nicht gegenseitig unwirksam machen.
\renewcommand{\rm}{\rmfamily}
\renewcommand{\sf}{\sffamily}
\renewcommand{\tt}{\ttfamily}
\renewcommand{\bf}{\bfseries}
\renewcommand{\it}{\itshape}
\renewcommand{\sl}{\slshape}
\renewcommand{\sc}{\scshape}
\newcommand{\md}{\mdseries}
\newcommand{\up}{\upshape}

% ###############  EOF  ###############
           % Include this file.                           ##
% ##    % mathe.tex - Mathematische (Mengen)Zeichen, Symbole und Funktionen
%             fuer LaTeX 2.09 und LaTeX 2e
%
%             Original von Dieter Boss, 08/94
%             Letzte Aenderung: H.Schmidt, 26-April-2000 (NDFT ... NIFFT)
%

\newcommand{\ds}[1]{\displaystyle{#1}}    % Kurzschreibweise fuer \displaystyle
\newcommand{\ml}[1]{\hbox{\large $#1$}}   % math large : Angenehme Schrift-
                                          % groesse bei der Darstellung von
                                          % Bruechen z.B. \ml{1\over\sqrt{2}}

% #####  Haeufig benoetigte Funktionen  #####
\newcommand{\E}[1]{\ensuremath{\mathrm{E}\left\{#1\right\}}}
\newcommand{\real}[1]{\ensuremath{\mathrm{Re}\left\{#1\right\}}}
\newcommand{\imag}[1]{\ensuremath{\mathrm{Im}\left\{#1\right\}}}
\newcommand{\rect}[1]{\ensuremath{\mathrm{rect}\left(#1\right)}}
\newcommand{\tri}[1]{\ensuremath{\mathrm{tri}\left(#1\right)}}
\newcommand{\si}{\ensuremath{\mathrm{si}}}
\newcommand{\di}{\ensuremath{\mathrm{di}}}
\newcommand{\ld}{\ensuremath{\mathrm{ld}}}  
\newcommand{\erf}{\ensuremath{\mathrm{erf}}}
\newcommand{\erfc}{\ensuremath{\mathrm{erfc}}}
\newcommand{\eds}[1]{\ensuremath{\mbox{e }^{\ds{#1}}}}
\newcommand{\ex}[1]{\ensuremath{e^{#1}}}
\newcommand{\ejO}{\ex{j\Omega}}

% #####  Mathematische Sonderzeichen  #####
\newcommand{\defas}{\ensuremath{\stackrel{\Delta}{=}}}

% #####  Mengenzeichen  #####
\newcommand{\Reell}{\mathsf{I} \kern -0.15em \mathsf{R}} 
\newcommand{\Nat}{\mathsf{I}  \kern -0.15em \mathsf{N}}
\newcommand{\Feld}{\mathsf{I} \kern -0.15em \mathsf{F}}
\newcommand{\Zahl}{\mathsf{Z} \kern -0.45em \mathsf{Z}}

% ##  Korrespondenz-"Knochen": ##
\newcommand{\korrespond}{\ensuremath{\;\circ \hskip-1ex -\hskip-1.2ex -\hskip-1.2ex- \hskip-1ex \bullet\;}}
\newcommand{\ikorrespond}{\ensuremath{\;\bullet \hskip-1ex -\hskip-1.2ex -\hskip-1.2ex- \hskip-1ex \circ\;}}

% #####  Transformationen  #####
\newcommand{\FT}[1]{\ensuremath{{\cal F}\left\{#1\right\}}}        % (kont.) Fourier-Trafo
\newcommand{\IFT}[1]{\ensuremath{{\cal F}^{-1}\left\{#1\right\}}}
\newcommand{\HT}[1]{\ensuremath{{\cal H}\left\{#1\right\}}}        % Hilbert-Trafo
\newcommand{\IHT}[1]{\ensuremath{{\cal H}^{-1}\left\{#1\right\}}}
\newcommand{\LT}[1]{\ensuremath{{\cal L}\left\{#1\right\}}}        % Laplace-Trafo
\newcommand{\ILT}[1]{\ensuremath{{\cal L}^{-1}\left\{#1\right\}}}
\newcommand{\DFT}[1]{\ensuremath{\mathrm{DFT}\left\{#1\right\}}}   % Diskrete Fourier-Trafo
\newcommand{\IDFT}[1]{\ensuremath{\mathrm{IDFT}\left\{#1\right\}}}
\newcommand{\FFT}[1]{\ensuremath{\mathrm{FFT}\left\{#1\right\}}}   % Fast Fourier-Trafo
\newcommand{\IFFT}[1]{\ensuremath{\mathrm{IFFT}\left\{#1\right\}}}
\newcommand{\NDFT}[2]{\ensuremath{\mathrm{DFT}_{#2}\left\{#1\right\}}}   % Diskrete Fourier-Trafo
\newcommand{\NIDFT}[2]{\ensuremath{\mathrm{IDFT}_{#2}\left\{#1\right\}}}
\newcommand{\NFFT}[2]{\ensuremath{\mathrm{FFT}_{#2}\left\{#1\right\}}}   % Fast Fourier-Trafo
\newcommand{\NIFFT}[2]{\ensuremath{\mathrm{IFFT}_{#2}\left\{#1\right\}}}\newcommand{\ZT}[1]{\ensuremath{{\cal Z}\left\{#1\right\}}}        % Z-Trafo
\newcommand{\IZT}[1]{\ensuremath{{\cal Z}^{-1}\left\{#1\right\}}}
\newcommand{\DTFT}[1]{\ensuremath{\mathrm{DTFT}\left\{#1\right\}}}   % Diskrete Time Fourier-Trafo
\newcommand{\IDTFT}[1]{\ensuremath{\mathrm{IDTFT}\left\{#1\right\}}}

% #####  Einheiten und Groessen  #####
\newcommand{\Hz}{\ensuremath{\mathrm{\:Hz}}}
\newcommand{\kHz}{\ensuremath{\mathrm{\:kHz}}}
\newcommand{\MHz}{\ensuremath{\mathrm{\:MHz}}}
\newcommand{\Mbits}{\ensuremath{\mathrm{\:Mbit/s}}}
\newcommand{\GHz}{\ensuremath{\mathrm{\:GHz}}}
\newcommand{\ms}{\ensuremath{\mathrm{\:ms}}}
\newcommand{\ns}{\ensuremath{\mathrm{\:ns}}}
\newcommand{\mus}{\ensuremath{\mathrm{\:\mu s}}}
\newcommand{\kmh}{\ensuremath{\mathrm{\:km/h}}}
\newcommand{\dB}{\ensuremath{\mathrm{\:dB}}}
\newcommand{\kbits}{\ensuremath{\mathrm{\:kbit/s}}}
\newcommand{\kBaud}{\ensuremath{\mathrm{\:kBaud}}}
\newcommand{\SNR}{\ensuremath{\frac{S}{N}}}
\newcommand{\EbN}{\ensuremath{\frac{E_b}{N_0}}}
\newcommand{\EbNh}{\ensuremath{\frac{E_b}{N_0/2}}}

% #####  Worte, die haeufig in Gleichungen gebraucht werden  #####
\newcommand{\Mit}{\quad\mathrm{mit}\;\,}          % kleingeschrieben existiert \mit schon!
\newcommand{\und}{\quad\mathrm{und}\;\,}
\newcommand{\da}{\quad\mathrm{da}\;\,}
\newcommand{\fuer}{\quad\mathrm{f"ur}\;\,}
\newcommand{\wobei}{\quad\mathrm{wobei}\;\,}
\newcommand{\mindex}[1]{\mbox{\scriptsize \sl #1}}

% #####  Fettschrift fuer Vektoren  #####
\newcommand{\vek}[1]{\ensuremath{\mathbf{#1}}}    % (fette) Vektoren oder Matrizen (mit Buchstabe als Argument)
\newcommand{\bs}[1]{\mbox{\boldmath$#1$}}         % (fette) schraege Vektoren oder Matrizen


% EOF
             % Must be included AFTER this file!            ##
% #####################################################################################

% Komfortables Auskommentieren ganzer LaTeX-Passagen
\newcommand{\comment}[1]{}

% Grafikeinbindung
\usepackage[dvips]{graphicx}
\newcommand{\bildwi}[2]{\includegraphics[width=#2\textwidth]{#1}}
\newcommand{\bildsc}[2]{\includegraphics[scale=#2]{#1}}
\newcommand{\bild}[2]{\includegraphics[width=#2\textwidth]{#1}}  % for compatibility
\newcommand{\centerbild}[1]{\centerline{#1}}                     % for compatibility

% Fettschrift im Mathemodus
\newcommand{\bfm}[1]{\mathbf{#1}}

% Definiere \bf, \sf etc. so um, dass sie sich bei gleichzeitiger Anwendung 
% sich nicht gegenseitig unwirksam machen.
\renewcommand{\rm}{\rmfamily}
\renewcommand{\sf}{\sffamily}
\renewcommand{\tt}{\ttfamily}
\renewcommand{\bf}{\bfseries}
\renewcommand{\it}{\itshape}
\renewcommand{\sl}{\slshape}
\renewcommand{\sc}{\scshape}
\newcommand{\md}{\mdseries}
\newcommand{\up}{\upshape}

% ###############  EOF  ###############
           % Include this file.                           ##
% ##    % mathe.tex - Mathematische (Mengen)Zeichen, Symbole und Funktionen
%             fuer LaTeX 2.09 und LaTeX 2e
%
%             Original von Dieter Boss, 08/94
%             Letzte Aenderung: H.Schmidt, 26-April-2000 (NDFT ... NIFFT)
%

\newcommand{\ds}[1]{\displaystyle{#1}}    % Kurzschreibweise fuer \displaystyle
\newcommand{\ml}[1]{\hbox{\large $#1$}}   % math large : Angenehme Schrift-
                                          % groesse bei der Darstellung von
                                          % Bruechen z.B. \ml{1\over\sqrt{2}}

% #####  Haeufig benoetigte Funktionen  #####
\newcommand{\E}[1]{\ensuremath{\mathrm{E}\left\{#1\right\}}}
\newcommand{\real}[1]{\ensuremath{\mathrm{Re}\left\{#1\right\}}}
\newcommand{\imag}[1]{\ensuremath{\mathrm{Im}\left\{#1\right\}}}
\newcommand{\rect}[1]{\ensuremath{\mathrm{rect}\left(#1\right)}}
\newcommand{\tri}[1]{\ensuremath{\mathrm{tri}\left(#1\right)}}
\newcommand{\si}{\ensuremath{\mathrm{si}}}
\newcommand{\di}{\ensuremath{\mathrm{di}}}
\newcommand{\ld}{\ensuremath{\mathrm{ld}}}  
\newcommand{\erf}{\ensuremath{\mathrm{erf}}}
\newcommand{\erfc}{\ensuremath{\mathrm{erfc}}}
\newcommand{\eds}[1]{\ensuremath{\mbox{e }^{\ds{#1}}}}
\newcommand{\ex}[1]{\ensuremath{e^{#1}}}
\newcommand{\ejO}{\ex{j\Omega}}

% #####  Mathematische Sonderzeichen  #####
\newcommand{\defas}{\ensuremath{\stackrel{\Delta}{=}}}

% #####  Mengenzeichen  #####
\newcommand{\Reell}{\mathsf{I} \kern -0.15em \mathsf{R}} 
\newcommand{\Nat}{\mathsf{I}  \kern -0.15em \mathsf{N}}
\newcommand{\Feld}{\mathsf{I} \kern -0.15em \mathsf{F}}
\newcommand{\Zahl}{\mathsf{Z} \kern -0.45em \mathsf{Z}}

% ##  Korrespondenz-"Knochen": ##
\newcommand{\korrespond}{\ensuremath{\;\circ \hskip-1ex -\hskip-1.2ex -\hskip-1.2ex- \hskip-1ex \bullet\;}}
\newcommand{\ikorrespond}{\ensuremath{\;\bullet \hskip-1ex -\hskip-1.2ex -\hskip-1.2ex- \hskip-1ex \circ\;}}

% #####  Transformationen  #####
\newcommand{\FT}[1]{\ensuremath{{\cal F}\left\{#1\right\}}}        % (kont.) Fourier-Trafo
\newcommand{\IFT}[1]{\ensuremath{{\cal F}^{-1}\left\{#1\right\}}}
\newcommand{\HT}[1]{\ensuremath{{\cal H}\left\{#1\right\}}}        % Hilbert-Trafo
\newcommand{\IHT}[1]{\ensuremath{{\cal H}^{-1}\left\{#1\right\}}}
\newcommand{\LT}[1]{\ensuremath{{\cal L}\left\{#1\right\}}}        % Laplace-Trafo
\newcommand{\ILT}[1]{\ensuremath{{\cal L}^{-1}\left\{#1\right\}}}
\newcommand{\DFT}[1]{\ensuremath{\mathrm{DFT}\left\{#1\right\}}}   % Diskrete Fourier-Trafo
\newcommand{\IDFT}[1]{\ensuremath{\mathrm{IDFT}\left\{#1\right\}}}
\newcommand{\FFT}[1]{\ensuremath{\mathrm{FFT}\left\{#1\right\}}}   % Fast Fourier-Trafo
\newcommand{\IFFT}[1]{\ensuremath{\mathrm{IFFT}\left\{#1\right\}}}
\newcommand{\NDFT}[2]{\ensuremath{\mathrm{DFT}_{#2}\left\{#1\right\}}}   % Diskrete Fourier-Trafo
\newcommand{\NIDFT}[2]{\ensuremath{\mathrm{IDFT}_{#2}\left\{#1\right\}}}
\newcommand{\NFFT}[2]{\ensuremath{\mathrm{FFT}_{#2}\left\{#1\right\}}}   % Fast Fourier-Trafo
\newcommand{\NIFFT}[2]{\ensuremath{\mathrm{IFFT}_{#2}\left\{#1\right\}}}\newcommand{\ZT}[1]{\ensuremath{{\cal Z}\left\{#1\right\}}}        % Z-Trafo
\newcommand{\IZT}[1]{\ensuremath{{\cal Z}^{-1}\left\{#1\right\}}}
\newcommand{\DTFT}[1]{\ensuremath{\mathrm{DTFT}\left\{#1\right\}}}   % Diskrete Time Fourier-Trafo
\newcommand{\IDTFT}[1]{\ensuremath{\mathrm{IDTFT}\left\{#1\right\}}}

% #####  Einheiten und Groessen  #####
\newcommand{\Hz}{\ensuremath{\mathrm{\:Hz}}}
\newcommand{\kHz}{\ensuremath{\mathrm{\:kHz}}}
\newcommand{\MHz}{\ensuremath{\mathrm{\:MHz}}}
\newcommand{\Mbits}{\ensuremath{\mathrm{\:Mbit/s}}}
\newcommand{\GHz}{\ensuremath{\mathrm{\:GHz}}}
\newcommand{\ms}{\ensuremath{\mathrm{\:ms}}}
\newcommand{\ns}{\ensuremath{\mathrm{\:ns}}}
\newcommand{\mus}{\ensuremath{\mathrm{\:\mu s}}}
\newcommand{\kmh}{\ensuremath{\mathrm{\:km/h}}}
\newcommand{\dB}{\ensuremath{\mathrm{\:dB}}}
\newcommand{\kbits}{\ensuremath{\mathrm{\:kbit/s}}}
\newcommand{\kBaud}{\ensuremath{\mathrm{\:kBaud}}}
\newcommand{\SNR}{\ensuremath{\frac{S}{N}}}
\newcommand{\EbN}{\ensuremath{\frac{E_b}{N_0}}}
\newcommand{\EbNh}{\ensuremath{\frac{E_b}{N_0/2}}}

% #####  Worte, die haeufig in Gleichungen gebraucht werden  #####
\newcommand{\Mit}{\quad\mathrm{mit}\;\,}          % kleingeschrieben existiert \mit schon!
\newcommand{\und}{\quad\mathrm{und}\;\,}
\newcommand{\da}{\quad\mathrm{da}\;\,}
\newcommand{\fuer}{\quad\mathrm{f"ur}\;\,}
\newcommand{\wobei}{\quad\mathrm{wobei}\;\,}
\newcommand{\mindex}[1]{\mbox{\scriptsize \sl #1}}

% #####  Fettschrift fuer Vektoren  #####
\newcommand{\vek}[1]{\ensuremath{\mathbf{#1}}}    % (fette) Vektoren oder Matrizen (mit Buchstabe als Argument)
\newcommand{\bs}[1]{\mbox{\boldmath$#1$}}         % (fette) schraege Vektoren oder Matrizen


% EOF
             % Must be included AFTER this file!            ##
% #####################################################################################

% Komfortables Auskommentieren ganzer LaTeX-Passagen
\newcommand{\comment}[1]{}

% Grafikeinbindung
\usepackage[dvips]{graphicx}
\newcommand{\bildwi}[2]{\includegraphics[width=#2\textwidth]{#1}}
\newcommand{\bildsc}[2]{\includegraphics[scale=#2]{#1}}
\newcommand{\bild}[2]{\includegraphics[width=#2\textwidth]{#1}}  % for compatibility
\newcommand{\centerbild}[1]{\centerline{#1}}                     % for compatibility

% Fettschrift im Mathemodus
\newcommand{\bfm}[1]{\mathbf{#1}}

% Definiere \bf, \sf etc. so um, dass sie sich bei gleichzeitiger Anwendung 
% sich nicht gegenseitig unwirksam machen.
\renewcommand{\rm}{\rmfamily}
\renewcommand{\sf}{\sffamily}
\renewcommand{\tt}{\ttfamily}
\renewcommand{\bf}{\bfseries}
\renewcommand{\it}{\itshape}
\renewcommand{\sl}{\slshape}
\renewcommand{\sc}{\scshape}
\newcommand{\md}{\mdseries}
\newcommand{\up}{\upshape}

% ###############  EOF  ###############
           % Include this file.                           ##
% ##    % mathe.tex - Mathematische (Mengen)Zeichen, Symbole und Funktionen
%             fuer LaTeX 2.09 und LaTeX 2e
%
%             Original von Dieter Boss, 08/94
%             Letzte Aenderung: H.Schmidt, 26-April-2000 (NDFT ... NIFFT)
%

\newcommand{\ds}[1]{\displaystyle{#1}}    % Kurzschreibweise fuer \displaystyle
\newcommand{\ml}[1]{\hbox{\large $#1$}}   % math large : Angenehme Schrift-
                                          % groesse bei der Darstellung von
                                          % Bruechen z.B. \ml{1\over\sqrt{2}}

% #####  Haeufig benoetigte Funktionen  #####
\newcommand{\E}[1]{\ensuremath{\mathrm{E}\left\{#1\right\}}}
\newcommand{\real}[1]{\ensuremath{\mathrm{Re}\left\{#1\right\}}}
\newcommand{\imag}[1]{\ensuremath{\mathrm{Im}\left\{#1\right\}}}
\newcommand{\rect}[1]{\ensuremath{\mathrm{rect}\left(#1\right)}}
\newcommand{\tri}[1]{\ensuremath{\mathrm{tri}\left(#1\right)}}
\newcommand{\si}{\ensuremath{\mathrm{si}}}
\newcommand{\di}{\ensuremath{\mathrm{di}}}
\newcommand{\ld}{\ensuremath{\mathrm{ld}}}  
\newcommand{\erf}{\ensuremath{\mathrm{erf}}}
\newcommand{\erfc}{\ensuremath{\mathrm{erfc}}}
\newcommand{\eds}[1]{\ensuremath{\mbox{e }^{\ds{#1}}}}
\newcommand{\ex}[1]{\ensuremath{e^{#1}}}
\newcommand{\ejO}{\ex{j\Omega}}

% #####  Mathematische Sonderzeichen  #####
\newcommand{\defas}{\ensuremath{\stackrel{\Delta}{=}}}

% #####  Mengenzeichen  #####
\newcommand{\Reell}{\mathsf{I} \kern -0.15em \mathsf{R}} 
\newcommand{\Nat}{\mathsf{I}  \kern -0.15em \mathsf{N}}
\newcommand{\Feld}{\mathsf{I} \kern -0.15em \mathsf{F}}
\newcommand{\Zahl}{\mathsf{Z} \kern -0.45em \mathsf{Z}}

% ##  Korrespondenz-"Knochen": ##
\newcommand{\korrespond}{\ensuremath{\;\circ \hskip-1ex -\hskip-1.2ex -\hskip-1.2ex- \hskip-1ex \bullet\;}}
\newcommand{\ikorrespond}{\ensuremath{\;\bullet \hskip-1ex -\hskip-1.2ex -\hskip-1.2ex- \hskip-1ex \circ\;}}

% #####  Transformationen  #####
\newcommand{\FT}[1]{\ensuremath{{\cal F}\left\{#1\right\}}}        % (kont.) Fourier-Trafo
\newcommand{\IFT}[1]{\ensuremath{{\cal F}^{-1}\left\{#1\right\}}}
\newcommand{\HT}[1]{\ensuremath{{\cal H}\left\{#1\right\}}}        % Hilbert-Trafo
\newcommand{\IHT}[1]{\ensuremath{{\cal H}^{-1}\left\{#1\right\}}}
\newcommand{\LT}[1]{\ensuremath{{\cal L}\left\{#1\right\}}}        % Laplace-Trafo
\newcommand{\ILT}[1]{\ensuremath{{\cal L}^{-1}\left\{#1\right\}}}
\newcommand{\DFT}[1]{\ensuremath{\mathrm{DFT}\left\{#1\right\}}}   % Diskrete Fourier-Trafo
\newcommand{\IDFT}[1]{\ensuremath{\mathrm{IDFT}\left\{#1\right\}}}
\newcommand{\FFT}[1]{\ensuremath{\mathrm{FFT}\left\{#1\right\}}}   % Fast Fourier-Trafo
\newcommand{\IFFT}[1]{\ensuremath{\mathrm{IFFT}\left\{#1\right\}}}
\newcommand{\NDFT}[2]{\ensuremath{\mathrm{DFT}_{#2}\left\{#1\right\}}}   % Diskrete Fourier-Trafo
\newcommand{\NIDFT}[2]{\ensuremath{\mathrm{IDFT}_{#2}\left\{#1\right\}}}
\newcommand{\NFFT}[2]{\ensuremath{\mathrm{FFT}_{#2}\left\{#1\right\}}}   % Fast Fourier-Trafo
\newcommand{\NIFFT}[2]{\ensuremath{\mathrm{IFFT}_{#2}\left\{#1\right\}}}\newcommand{\ZT}[1]{\ensuremath{{\cal Z}\left\{#1\right\}}}        % Z-Trafo
\newcommand{\IZT}[1]{\ensuremath{{\cal Z}^{-1}\left\{#1\right\}}}
\newcommand{\DTFT}[1]{\ensuremath{\mathrm{DTFT}\left\{#1\right\}}}   % Diskrete Time Fourier-Trafo
\newcommand{\IDTFT}[1]{\ensuremath{\mathrm{IDTFT}\left\{#1\right\}}}

% #####  Einheiten und Groessen  #####
\newcommand{\Hz}{\ensuremath{\mathrm{\:Hz}}}
\newcommand{\kHz}{\ensuremath{\mathrm{\:kHz}}}
\newcommand{\MHz}{\ensuremath{\mathrm{\:MHz}}}
\newcommand{\Mbits}{\ensuremath{\mathrm{\:Mbit/s}}}
\newcommand{\GHz}{\ensuremath{\mathrm{\:GHz}}}
\newcommand{\ms}{\ensuremath{\mathrm{\:ms}}}
\newcommand{\ns}{\ensuremath{\mathrm{\:ns}}}
\newcommand{\mus}{\ensuremath{\mathrm{\:\mu s}}}
\newcommand{\kmh}{\ensuremath{\mathrm{\:km/h}}}
\newcommand{\dB}{\ensuremath{\mathrm{\:dB}}}
\newcommand{\kbits}{\ensuremath{\mathrm{\:kbit/s}}}
\newcommand{\kBaud}{\ensuremath{\mathrm{\:kBaud}}}
\newcommand{\SNR}{\ensuremath{\frac{S}{N}}}
\newcommand{\EbN}{\ensuremath{\frac{E_b}{N_0}}}
\newcommand{\EbNh}{\ensuremath{\frac{E_b}{N_0/2}}}

% #####  Worte, die haeufig in Gleichungen gebraucht werden  #####
\newcommand{\Mit}{\quad\mathrm{mit}\;\,}          % kleingeschrieben existiert \mit schon!
\newcommand{\und}{\quad\mathrm{und}\;\,}
\newcommand{\da}{\quad\mathrm{da}\;\,}
\newcommand{\fuer}{\quad\mathrm{f"ur}\;\,}
\newcommand{\wobei}{\quad\mathrm{wobei}\;\,}
\newcommand{\mindex}[1]{\mbox{\scriptsize \sl #1}}

% #####  Fettschrift fuer Vektoren  #####
\newcommand{\vek}[1]{\ensuremath{\mathbf{#1}}}    % (fette) Vektoren oder Matrizen (mit Buchstabe als Argument)
\newcommand{\bs}[1]{\mbox{\boldmath$#1$}}         % (fette) schraege Vektoren oder Matrizen


% EOF
             % Must be included AFTER this file!            ##
% #####################################################################################

% Komfortables Auskommentieren ganzer LaTeX-Passagen
\newcommand{\comment}[1]{}

% Grafikeinbindung
\usepackage[dvips]{graphicx}
\newcommand{\bildwi}[2]{\includegraphics[width=#2\textwidth]{#1}}
\newcommand{\bildsc}[2]{\includegraphics[scale=#2]{#1}}
\newcommand{\bild}[2]{\includegraphics[width=#2\textwidth]{#1}}  % for compatibility
\newcommand{\centerbild}[1]{\centerline{#1}}                     % for compatibility

% Fettschrift im Mathemodus
\newcommand{\bfm}[1]{\mathbf{#1}}

% Definiere \bf, \sf etc. so um, dass sie sich bei gleichzeitiger Anwendung 
% sich nicht gegenseitig unwirksam machen.
\renewcommand{\rm}{\rmfamily}
\renewcommand{\sf}{\sffamily}
\renewcommand{\tt}{\ttfamily}
\renewcommand{\bf}{\bfseries}
\renewcommand{\it}{\itshape}
\renewcommand{\sl}{\slshape}
\renewcommand{\sc}{\scshape}
\newcommand{\md}{\mdseries}
\newcommand{\up}{\upshape}

% ###############  EOF  ###############
