\chapter{Coded CDMA Receiver Block}
In this chapter we study about the various optimum and sub-optimum detection schemes and analyze each scheme in detail by studying advantages and dis-advantages of those schemes and later we choose the best suited detection scheme to implement in our communication system.
\section{Optimum Detection}
The vector of information bits $\mathrm{\mathbf{x}}_u$ is transmitted over channel with channel impulse response $\mathrm{\mathbf{h}}_u$ and at the receiver all the signals are superimposed and corrupted by additive white Gaussian noise $\mathrm{\mathbf{n}}$. Then the received vector is of the form
\begin{equation}
\mathrm{\mathbf{y}}=\mathrm{\mathbf{H}}\cdot\mathrm{\mathbf{x}}+\mathrm{\mathbf{n}}=\mathrm{\mathbf{S}}\cdot\mathrm{\mathbf{a}}+\mathrm{\mathbf{n}}
\end{equation}
where $\mathrm{\mathbf{H}}=[\mathrm{\mathbf{T_{h_1}}}\cdots\mathrm{\mathbf{T_{h_U}}}]$ is the concatenation of several user-specific convolution matrices $\mathrm{\mathbf{T_{h_u}}}$. The received vector $\mathrm{\mathbf{x}}$ consist of user-specific spreading codes $\mathrm{\mathbf{c}}_u$ and the user data $\mathrm{\mathbf{a}}_u$. The user-specific spreading codes when convolve with channel impulse response $\mathrm{\mathbf{h}}_u$ forms the signature $\mathrm{\mathbf{s}}_u$ of a user. Thus $\mathrm{\mathbf{S}}$ represents the signature of all the users.\\ \\
We now detect users jointly and see what complexity arises in terms of computation cost to implement as a detector in our communication systems.
\subsection{Optimum Joint Sequence Detection}
In optimum joint sequence detection scheme we assume that we have perfect knowledge of the channel impulse response and the spreading codes employed in the communication system. Optimum detector is one which performs joint maximum a-posteriori decoding for all users upon receiving the sequence $\mathrm{\mathbf{y}}$. Mathematically we can express the joint maximum a-posteriori decoding as
\begin{equation}
\mathrm{\hat{\mathbf{d}}^{map}}=\argmax_{\mathrm{\tilde{\mathbf{d}}}} \mathrm{Pr}\{\mathrm{\tilde{\mathbf{d}}}\vert \mathrm{\mathbf{S}},\mathrm{\mathbf{y}}\} = \argmax_{\mathrm{\tilde{\mathbf{d}}}} p_{\underline{\mathcal{Y}}\vert\mathrm{\tilde{\mathbf{d}}},\mathrm{\mathbf{S}}}(\mathrm{\mathbf{y}})\cdot \mathrm{Pr}\{\mathrm{\tilde{\mathbf{d}}}\}
\end{equation}
by using Bayes rule we can obtain the second equality, now we can exploit the fact that $p_{\underline{\mathcal{Y}}\vert \mathrm{\tilde{\mathbf{d}}},\mathrm{\mathbf{S}}}$ \cite{K05} is independent from the hypothesis $\mathrm{\tilde{\mathbf{d}}}$, so it does not contribute to the decision. In case a-priori probabilities $\mathrm{Pr\{\tilde{\mathbf{d}}\}}$ is not known then we can apply joint maximum-likelihood (ML) detection scheme on the received sequence. Mathematically joint ML detection scheme can be written as
\begin{equation}
\mathrm{\hat{\mathbf{d}}^{mld}}=\argmax_{\mathrm{\tilde{\mathbf{d}}}}p_{\underline{\mathcal{Y}}\vert\mathrm{\tilde{\mathbf{d}}},\mathrm{\mathbf{S}}}(\mathrm{\mathbf{y}})
\end{equation}
after some algebraic simplification we obtain as follows
\begin{equation*}
\mathrm{\hat{\mathbf{d}}^{mld}}=\argmax_{\mathrm{\tilde{\mathbf{d}}}} \mathrm{ln\;exp}\left(-\frac{\hbox{$[\mathrm{\mathbf{y}}-\mathrm{\mathbf{S}}\cdot \mathrm{\mathbf{a}}(\mathrm{\tilde{\mathbf{d}}})]^{H}\;[\mathrm{\mathbf{y}}-\mathrm{\mathbf{S}}\cdot \mathrm{\mathbf{a}}(\mathrm{\tilde{\mathbf{d}}})]$}} {\hbox{$\sigma^2_{\mathcal{N}}$}}\right) \\ \\
\end{equation*}

\begin{equation}
\mathrm{\hat{\mathbf{d}}^{mld}}=\argmin_{\mathrm{\tilde{\mathbf{d}}}}\vert \vert \mathrm{\mathbf{y}}-\mathrm{\mathbf{S}}\cdot\mathrm{\mathbf{a}}\mathrm{(\tilde{\mathbf{d}})}\vert\vert^2
\end{equation}
where $\sigma^2_{\mathcal{N}}$ is the variance of the noise and $\mathrm{\mathbf{a}}(\mathrm{\tilde{\mathbf{d}}})$ is the modulated sequence associated with the hypothesis $\mathrm{\tilde{\mathbf{d}}}$.\\ \\
From (4.4) we interpret that joint maximum likelihood detector search for the hypothesis $\mathrm{\tilde{\mathbf{d}}}$ that minimizes the squared Euclidian distance between the received sequence $\mathrm{\mathbf{y}}$ and $\mathrm{\mathbf{S\cdot a(\tilde{d})}}$. \\ \\
Since $\mathrm{\mathbf{d}}$ is the set of finite discrete alphabets, an exhaustive search is required to fulfill the criterion in (4.4). The complexity in terms of computations required to reach the decision is far too high, even for medium size systems. Hence, it is practically infeasible to implement as a detector in our communication system.
\subsection{Optimum Preprocessing and Subsequent Decoding}
One approach to reduce computation complexity in detecting users is to separate FEC decoding block and the multi-user detection block and process each block separately so that one stretch computations can be avoided. With that idea we construct such a system and analyze whether it is feasible to implement or not. \\ \\ 
In such a system, employing soft-decision multi-user detector will provide information gain rather then hard-decision multi-user detector and there by the soft-outputs from the multi-user detector are feed directly to the soft-decoders as a-priori information. Further we employ BPSK as a modulation scheme $b_k\in\{0,1\}$ is equivalent to $a_k\in\{+1,-1\}$ \cite{K05}.
\begin{figure}[htb]
\centerline{ \bildsc{ps/joint_preproc.eps} {1.0} }
\caption{Joint Pre-processing with Subsequent Decoding.}
%\label{Convolutional Encoder}
\end{figure}\\
The receiver starts processing the received vector $\mathrm{\mathbf{y}}$ by calculating the log-likelihood ratios as
\begin{eqnarray}
\begin{array}{lllll}
L(\hat{a}_u\vert \mathrm{\mathbf{y}})&=&\mathrm{ln}\frac{\hbox{$\mathrm{Pr}\{a_u=+1 \vert \mathrm{\mathbf{y}}\}$}}{\hbox{$\mathrm{Pr}\{a_u=-1 \vert \mathrm{\mathbf{y}}\}$}}&=&\mathrm{ln}\frac{\hbox{$p_{\underline{\mathcal{Y}}\vert a_u=+1}(\mathrm{\mathbf{y}})\cdot \mathrm{Pr}\{a_u=+1\}$}}{\hbox{$p_{\underline{\mathcal{Y}}\vert a_u=+1}(\mathrm{\mathbf{y}})\cdot \mathrm{Pr}\{a_u=-1\}$}} \\ \\
&=&L(\mathrm{\mathbf{y}}\vert \hat{a}_u)+L_a(a_u)
\end{array}
\end{eqnarray}
After some algebraic manipulation finally we get
\begin{eqnarray}
\begin{array}{lll}
L(\hat{a}_u\vert \mathrm{\mathbf{y}})&=&
\mathrm{ln}
\frac
{\hbox{$\sum_{\mathrm{\mathbf{a}},a_u=+1}p_{\underline{\mathcal{Y}}\vert a}(\mathrm{\mathbf{y}})\cdot \mathrm{Pr}\{\mathrm{\mathbf{a}}\}$}}
{\hbox{$\sum_{\mathrm{\mathbf{a}},a_u=-1}p_{\underline{\mathcal{Y}}\vert a}(\mathrm{\mathbf{y}})\cdot \mathrm{Pr}\{\mathrm{\mathbf{a}}\}$}} \\ \\
&=&
\mathrm{ln}
\frac
{\hbox{$\sum_{\mathrm{\mathbf{a}},a_u=+1}\mathrm{exp}[-\vert\vert \mathrm{\mathbf{y}}-\mathrm{\mathbf{S}}\cdot \mathrm{\mathbf{a}}\vert \vert ^2 / \sigma^2_{\mathcal{N}}]\cdot \mathrm{Pr}\{\mathrm{\mathbf{a}}\}$}}
{\hbox{$\sum_{\mathrm{\mathbf{a}},a_u=-1}\mathrm{exp}[-\vert\vert \mathrm{\mathbf{y}}-\mathrm{\mathbf{S}}\cdot \mathrm{\mathbf{a}}\vert \vert ^2 / \sigma^2_{\mathcal{N}}]\cdot \mathrm{Pr}\{\mathrm{\mathbf{a}}\}$}}
\end{array}
\end{eqnarray}
where $\mathrm{Pr}\{\mathrm{\mathbf{a}}\}$ are the a-priori probabilities which are not know in prior and can be neglected during the calculation. Further manipulation on the exponential part gives 
\begin{eqnarray}
\begin{array}{lll}
-\vert\vert\mathrm{\mathbf{y}}-\mathrm{\mathbf{S}}\cdot\mathrm{\mathbf{a}}\vert\vert ^2&\Rightarrow&2\cdot\mathrm{Re}\{\mathrm{\mathbf{a}}^H\cdot\mathrm{\mathbf{S}}^H\cdot\mathrm{\mathbf{y}}\}-\mathrm{\mathbf{a}}^H\cdot\mathrm{\mathbf{S}}^H\cdot\mathrm{\mathbf{S}}\cdot\mathrm{\mathbf{a}}\\\\
&=&2\cdot \mathrm{Re}\{\mathrm{\mathbf{a}}^H\cdot\mathrm{\mathbf{r}}\}-\mathrm{\mathbf{a}}^H \cdot \mathrm{\mathbf{R}}\cdot \mathrm{\mathbf{a}}
\end{array}
\end{eqnarray}
where $\mathrm{\mathbf{R}}\;=\;\mathrm{\mathbf{S}}^H\cdot\mathrm{\mathbf{S}}$ is the correlation matrix.\\ \\
From (4.7) we see $\mathrm{\mathbf{r}}=\mathrm{\mathbf{S}}^H\cdot\mathrm{\mathbf{y}}$, is nothing but matched filter output which is correlated with the hypothesis $\mathrm{\mathbf{a}}$. As we employ BPSK modulation in our system $\mathrm{\mathbf{a}}$ and $\mathrm{\mathbf{r}}^{'}\;=\;\mathrm{Re}\{\mathrm{\mathbf{S}}^H\cdot \mathrm{\mathbf{y}}\}$ becomes real-valued then, (4.7) simplifies to
\begin{equation}
-\vert\vert\mathrm{\mathbf{y}}-\mathrm{\mathbf{S}}\cdot\mathrm{\mathbf{a}}\vert\vert ^2 \Longrightarrow 2\mathrm{\mathbf{a}}^T\mathrm{\mathbf{r}}^{'}-\mathrm{\mathbf{a}}^T\cdot \mathrm{Re}\{\mathrm{\mathbf{R}}\}\cdot \mathrm{\mathbf{a}}
\end{equation}
As we know the output of the matched filter delivers sufficient statistics and with that further pre-processing still provides the optimum log-likelihood ratios as described below
\begin{equation}
L(\mathrm{\mathbf{r}}^{'}\vert \hat{a}_u)=\mathrm{ln}\frac{\hbox{$\sum_{\mathrm{\mathbf{a}},a_u=+1}\mathrm{exp}\left([2\mathrm{\mathbf{a}}^T\mathrm{\mathbf{r}}^{'}-\mathrm{\mathbf{a}}^T\mathrm{\mathbf{R}}^{'}\mathrm{\mathbf{a}}]/(2\sigma^2_{\mathcal{N}^{'}})\right)$}}{\hbox{$\sum_{\mathrm{\mathbf{a}},a_u=-1}\mathrm{exp}\left([2\mathrm{\mathbf{a}}^T\mathrm{\mathbf{r}}^{'}-\mathrm{\mathbf{a}}^T\mathrm{\mathbf{R}}^{'}\mathrm{\mathbf{a}}]/(2\sigma^2_{\mathcal{N}^{'}})\right)$}}
\end{equation}
where $\mathrm{\mathbf{R}}^{'}\;=\;\mathrm{Re}\{\mathrm{\mathbf{R}}\}$. \\ \\
As we see in (4.9) the exponent still remains, hence the computation complexity grows exponentially with the number of users. Hence it is infeasible to implement in our system. 
\subsection{Turbo Detection with Optimum Preprocessing and Decoding}
In combating the exponential growth of complexity we analyze yet an another scheme in which we employ individual de-interleaving and FEC decoding, and there by the information is feed back to the pre-processors as according to the Turbo principle. \\ \\
Soft output decoders such as BCJR \cite{BCJR} delivers the log likelihood ratios $L(\hat{a}_u)$ which can be used as a-priori LLR's $L_a(a_u)$ to calculate the a-priori probabilities
\begin{figure}[htb]
\centerline{ \bildsc{ps/turbo_receiver.eps} {1.0} }
\caption{Turbo Detection with Optimum Pre-processing.}
%\label{Convolutional Encoder}
\end{figure}
\begin{equation}
\mathrm{Pr}\{a_u\}=\frac{\hbox{$e^{L_u(a_u)/2}$}}{\hbox{$1+e^{L_u(a_u)}$}}\cdot e^{a_uL_u(a_u)/2}
\end{equation}
which can be used for subsequent processing. Individual interleavers guarantee the assumption that, $L_a(a_u)$ are statistically independent. So we write a-priori information as follows
\begin{equation}
\mathrm{Pr}\{\mathrm{\mathbf{a}}\}=\prod\limits_{u=1}^{N_U}\frac{e^{L_a(a_u)/2}}{1+e^{L_u(a_u)}}\cdot e^{a_uL_u(a_u)/2}
\end{equation}
Using equation (4.8) and inserting (4.11) in (4.6) and subsequent simplification we obtain $L(\hat{a}_u\vert \mathrm{\mathbf{r}})$ as
\begin{equation}
L(\hat{a}_u\vert \mathrm{\mathbf{r}})=
\mathrm{ln}
\frac
{
\hbox{$\sum_{\mathrm{\mathbf{a}},a_u=+1}e^{\left[2\mathrm{\mathbf{a}}^T\mathrm{\mathbf{r}}^{'}-\mathrm{\mathbf{a}}^{T}\mathrm{\mathbf{R}}^{'}\mathrm{\mathbf{a}}\right]/(2\sigma^2_{\mathcal{N}^{'}})+\sum_{\mu = 1}^{N_U}}a_{\mu}L_a(a_{\mu})/2$}
}
{
\hbox{$\sum_{\mathrm{\mathbf{a}},a_u=-1}e^{\left[2\mathrm{\mathbf{a}}^T\mathrm{\mathbf{r}}^{'}-\mathrm{\mathbf{a}}^{T}\mathrm{\mathbf{R}}^{'}\mathrm{\mathbf{a}}\right]/(2\sigma^2_{\mathcal{N}^{'}})+\sum_{\mu = 1}^{N_U}}a_{\mu}L_a(a_{\mu})/2$}
}
\end{equation}
Above equation can be written in the form after some manipulation as follows
\begin{equation}
L(\hat{a}_u\vert \mathrm{\mathbf{r}})=L_a(a_u)+L_e(\hat{a}_u)
\end{equation}
Equation (4.12) can be approximated as follows
\begin{equation*}
L(\hat{a}\vert \mathrm{\mathbf{r}})\approx L_a(a_u)
+\max_{\mathrm{\mathbf{a}},a_u=+1}\left[ \frac{\hbox{$2\mathrm{\mathbf{a}}^T\mathrm{\mathbf{r}}^{'}-\mathrm{\mathbf{a}}^T\mathrm{\mathbf{R}}^{'}\mathrm{\mathbf{a}}$}}{\hbox{$2\sigma^2_{\mathcal{N}^{'}}$}}+\frac{\hbox{$1$}}{\hbox{$2$}} \sum\limits_{\stackrel{\mu=1}{\mu \ne u}}^{N_U}a_{\mu}L_a(a_\mu) \right] \\ \\
\end{equation*}
\begin{equation}
-\max_{\mathrm{\mathbf{a}},a_u=-1}\left[ \frac{\hbox{$2\mathrm{\mathbf{a}}^T\mathrm{\mathbf{r}}^{'}-\mathrm{\mathbf{a}}^T\mathrm{\mathbf{R}}^{'}\mathrm{\mathbf{a}}$}}{\hbox{$2\sigma^2_{\mathcal{N}^{'}}$}}+\frac{\hbox{$1$}}{\hbox{$2$}} \sum\limits_{\stackrel{\mu=1}{\mu \ne u}}^{N_U}a_{\mu}L_a(a_\mu) \right]
\end{equation}
We now realize a receiver structure as shown in \textbf{Figure 4.2} to calculate the solutions presented in (4.12) and (4.14). During first iteration no a-priori information is available and only term given in (4.8) is calculated which remains constant for subsequent iteration, are feed directly to the FEC decoders which then delivers the LLR's $L_a(a_u)$. Then $L_a(a_u)$ is feed-back as a-priori information for the pre-processor to improve the estimates of $L(\hat{a}_u\vert \mathrm{\mathbf{r}})$. After certain iterations decoder delivers most reliable information which can be decided by taking hard-decision.\\ \\
In this step we reduce the complexity of FEC decoders to grow linearly with the users. But still exponential growth is seen in pre-processors.
\section{Linear and Non-Linear Multi-User Detection}
In a step towards reducing the complexity of pre-processors is to employ linear joint pre-processors, they assume the transmitted signals are continuously distributed and hence they do not exploit the knowledge of the finite alphabet. Due to this, complexity grows linearly with respect to the growth of the user, hence we have to solve the linear equation system $\mathrm{\mathbf{y}}=\mathrm{\mathbf{S}}\mathrm{\mathbf{a}}+\mathrm{\mathbf{n}}$ to find $\mathrm{\mathbf{a}}$. Solution can be obtained if we find a suitable matrix such that multiplication with $\mathrm{\mathbf{S}}$ gives identity term. Let we assume the matrix as $\mathrm{\mathbf{W}}$ and multiplication with $\mathrm{\mathbf{y}}$ gives the solution to the linear equation system $\mathrm{\mathbf{\hat{a}}}\;=\;\mathrm{\mathbf{W}}\cdot\mathrm{\mathbf{y}}$.
\subsection{Decorrelator}
Decorrelator or equivalently the zero-forcing equalizer searches for the symbol vectors that minimizes the squared Euclidean distance from the received vector $\mathrm{\mathbf{y}}$. Mathematically it is given as follows
\begin{equation}
\mathrm{\mathbf{\hat{a}_{ZF}}}=\argmin_{\mathrm{\mathbf{\tilde{a}}}\in\mathrm{\mathbf{C}}^{N_U}}\vert\vert\mathrm{\mathbf{y}}-\mathrm{\mathbf{S}}\mathrm{\mathbf{\tilde{a}}}\vert\vert^2
\end{equation}
then the solution is
\begin{equation}
\mathrm{\mathbf{\tilde{a}}_{ZF}}=\mathrm{\mathbf{W}_{ZF}} \cdot \mathrm{\mathbf{y}} = (\mathrm{\mathbf{S}}^H\mathrm{\mathbf{S}})^{-1} \cdot \mathrm{\mathbf{S}}^H\cdot\mathrm{\mathbf{y}}=\mathrm{\mathbf{R}}^{-1}\cdot\mathrm{\mathbf{r}}
\end{equation}
Decorrelator takes the output from the channel $\mathrm{\mathbf{y}}$ which then do the matched filtering by multiplying with $\mathrm{\mathbf{S}}^H$ then the output of the matched filter $\mathrm{\mathbf{r}}$ is decorrelated with $\mathrm{\mathbf{R}}^{-1}$ which then results in
\begin{equation}
\mathrm{\mathbf{\hat{a}}_{ZF}}=\mathrm{\mathbf{R}}^{-1}\mathrm{\mathbf{S}}^H\cdot(\mathrm{\mathbf{S}}\mathrm{\mathbf{a}}+\mathrm{\mathbf{n}})=\mathrm{\mathbf{a}}+\mathrm{\mathbf{R}}^{-1}\mathrm{\mathbf{S}}^H\cdot\mathrm{\mathbf{n}}=\mathrm{\mathbf{a}}+\mathrm{\mathbf{W}_{ZF}}\cdot\mathrm{\mathbf{n}}
\end{equation}
The output of the matched filter consist of the desired symbol vector $\mathrm{\mathbf{a}}$ and modified noise vector. By using zero-forcing techniques we can completely suppress the interference with the cost amplifying the noise by $\mathrm{\mathbf{R}}^{-1}$. This is main drawback as system see strong amplification of noise at very high system loads $\beta$ leads to very low SNRs. Error rate performance curves are shown for different levels of signal-to-noise ratio levels in \cite{K05}.
\subsection{Minimum Mean Square Error Receiver}
As we know that matched filter concentrates only on the background noise and completely ignores the interference, and where as the decorrelator concentrates only on the interference and ignores the background noise. Hence, we can treat Zero-forcing and matched filtering as two extreme cases. Now a compromising solution for those two extreme cases is obtained with MMSE detector, which minimizes the average squared Euclidean distance between the estimate $\mathrm{\mathbf{\hat{a}}_{MMSE}}\;=\;\mathrm{\mathbf{W}_{MMSE}}\cdot\mathrm{\mathbf{y}}$ and the true data vector $\mathrm{\mathbf{a}}$
\begin{equation}
\mathrm{\mathbf{W}_{MMSE}}=\argmin_{\tilde{\mathrm{\mathbf{W}}}\in\mathrm{\mathbf{C}}^{N_U\mathrm{x}N_s}}\mathrm{E}\left\{\vert\vert\mathrm{\mathbf{\tilde{W}y}}-\mathrm{\mathbf{a}}\vert\vert^2\right\}
\end{equation}
After some assumptions finally we reach to
\begin{equation}
\mathrm{\mathbf{W}_{MMSE}}=(\mathrm{\mathbf{S}}^H\mathrm{\mathbf{S}}+\frac{\hbox{$\sigma^2_{\mathcal{N}}$}}{\hbox{$\sigma^2_{\mathcal{A}}$}}\mathrm{\mathbf{I}}N_U)^{-1}\cdot\mathrm{\mathbf{S}}^{H}=\left(\mathrm{\mathbf{R}}+\frac{\hbox{$N_0$}}{\hbox{$E_s$}}\mathrm{\mathbf{I}}_{N_U}\right)^{-1}\cdot\mathrm{\mathbf{S}}^H
\end{equation}
As we mentioned already the MMSE is a compromise between the decorrelator and the matched filter, which is obvious when we let $\sigma^2_{\mathcal{N}}\rightarrow 0$ then the identity part $\mathrm{\mathbf{I}}_{N_U}$ get canceled and we obtain the simple decorrelator, thereby suppressing the interference completely. Suppose when  $\sigma^2_{\mathcal{N}}\rightarrow \infty$ then $\mathrm{\mathbf{R}}$ can be neglected, thus arises the case for the matched filter. As we see that no complete suppression of interference in MMSE detector, some residual interference remains which can disturb the data.
\subsection{Linear Parallel Interference Cancellation}
As we see in previous sections that the solution to the linear system of equation can be obtained by calculating the inverse of the system matrix $\mathrm{\mathbf{S}}$, but the solution can also be obtained by iterative methods. We use those methods and see what advantage we get when compared to the previous method. We start from the output of the matched filter bank
\begin{figure}[htb]
\centerline{ \bildsc{ps/pic_multi_stage.eps} {1.0} }
\caption{Parallel Interference Canceler}
%\label{Convolutional Encoder}
\end{figure}
\begin{equation}
\mathrm{\mathbf{r}}=\mathrm{\mathbf{R}}\mathrm{\mathbf{a}}+\mathrm{\mathbf{S}}^H\mathrm{\mathbf{n}}\hspace{5mm}\Rightarrow\hspace{5mm}\mathrm{\mathbf{\hat{a}}}=\mathrm{\mathbf{W}}\cdot\mathrm{\mathbf{r}}\hspace{5mm}\Leftrightarrow\hspace{5mm}\mathrm{\mathbf{M}}\cdot\mathrm{\mathbf{\hat{a}}}=\mathrm{\mathbf{r}}
\end{equation}
We now obtain the solution by solving the linear equation systems $\mathrm{\mathbf{M}}\cdot\mathrm{\mathbf{\hat{a}}}\;=\;\mathrm{\mathbf{r}}$. Rather writing in matrix form we now write the single element in the matrix as follows.
\begin{equation}
r_u=M_{u,u}\hat{a}_u+\sum\limits_{v=1}^{u-1}M_{u,v}\hat{a}_v+\sum\limits_{v=u+1}^{N_U}M_{u,v}\hat{a}_v
\end{equation}
In (4.21) received value $r_u$ consist of the superposition of the scaled desired symbol $\hat{a}_u$ and the weighted interfering symbols $\hat{a}_{v\ne u}$. Now we start the iterative solution with the weighted matched filter output of the interfering symbols as the starting values $\hat{a}^{(0)}_{v\ne u}\;=\;r_{v\ne u}/M_{v,u}$, which can be subtracted on $r_u$ leads to improvised estimates $\hat{a}_u^{(1)}$ after the first iteration. In such a way we proceed for all the users and repeat the iteration till we reach the convergence. At the $\mu$-th iteration, the $u$-th symbol becomes
\begin{equation}
\hat{a}_u^{(\mu)}=M_{u,u}^{-1}\cdot\left[r_u-\sum\limits_{v=1}^{u-1}M_{u,v}\hat{a}_v^{(\mu-1)}-\sum\limits_{v=u+1}^{N_U}M_{u,v}\hat{a}_v^{(\mu-1)}\right]
\end{equation}
This method of finding solutions to simultaneous linear equation is called Jacobi algorithm and which is also know as parallel interference cancellation.\\ \\
Interestingly the choice of the matrix $\mathrm{\mathbf{M}}$ determines type of detector, if $\mathrm{\mathbf{M}}=\mathrm{\mathbf{R}}$ then all the co-efficients of $M_{u,v}$ and $R_{u,v}$ are then the obtained detector is decorrelator. For MMSE filter the choice of $\mathrm{\mathbf{M}}$ equal to $\mathrm{\mathbf{R}}+\sigma_{\mathcal{N}}^{2}/\sigma_{\mathcal{A}}^{2}\cdot\mathrm{\mathbf{I}}_{N_U}$. \\ \\
The convergence properties of parallel interference cancellation are rather very poor which is well discussed in \cite{K05}.
\subsection{Linear Successive Interference Cancellation}
A substantial improvement in convergence behavior is seen when cancellation of interference takes place successively starting with $u=1$ and $u=U$. At the $\mu$-th iteration for user $u$ uses only the estimates $\hat{a}_{v\ne u}^{\mu-1}$ of previous iteration $\mu$-1. However at the $\mu$-th iteration we already have the updated estimates $\hat{a}_{v<u}^{(\mu)}$ for users $1\leq v < u$, rather then the estimates $\hat{a}_{v<u}^{(\mu-1)}$ as in (4.22).  The $\mu$-th element is given as 
\begin{equation}
a_u^{(\mu)}=M_{u,u}^{-1}\cdot \left[r_u-\sum\limits_{v=1}^{u-1}M_{u,v}\hat{a}_v^{(\mu)}-\sum\limits_{v=u+1}^{N_U}M_{u,v}\hat{a}_v^{(\mu-1)}\right]
\end{equation}
This method of finding solution to linear equation system is called Gauss-Seidel algorithm. The convergence properties for linear successive interference cancellation are shown in \cite{K05}.
\section{Non-linear Iterative Multi-User Detection}
One of the main drawback in linear multi-user detectors is, they do not exploit the finite nature of the transmitted signals. Thereby certain information is lost and this can be overcome by introducing a non-linear devices to exploit the discrete alphabet. Introducing a non-linear device means the signals $\hat{a}_{v\ne u}^{\mu}$ in (4.22) or (4.23) is passed through a non-linear device before they used for interference cancellation. \\ \\
We study some of non-linear device and find which is more suitable for our communication system.
\subsection{Non-linear Devices}
We start with very simple hard-decision, where decision is made based upon the sign of the signal.
\begin{equation}
\mathcal{Q}_{\mathrm{HD}}(y)=\mathrm{sgn}(y)
\end{equation}
Though simplicity in structure has main advantage, the potential disadvantage is that if the decision is made wrong, then the interference gets doubled then the original levels. Otherwise interference gets completely canceled when the decision is correct, but making wrong decisions are highly probable during the initial stage under very large system loads $\beta$. 
\begin{figure}[htb]
\centerline{ \bildsc{ps/clipping.eps} {1.0} }
\caption{Non-Linear Devices.}
%\label{Convolutional Encoder}
\end{figure}
In order to keep the undesirable effect caused by wrong decisions as less as possible, we leave un-reliable samples un-decided. Of-course with these approach we do not cancel interference perfectly but the error is rather very small when compared to hard-decision. Such an approach can be realized through very simple clipper function.
\begin{eqnarray}
\mathcal{Q}_{\mathrm{clip}}(y)=\left\{
\begin{array}{rllr}
-1&\hspace{2mm}&\mathrm{for}&\;y < -1 \\ 
 y&\hspace{2mm}&\mathrm{for}&\vert\; y \vert \leq 1 \\ 
+1&\hspace{2mm}&\mathrm{for}&\;y > +1
\end{array}
\right.
\end{eqnarray}
Clipper is designed by exploiting the fact that transmitted signals cannot go larger than one, with such a design the interference is totally canceled if the signal has correct sign and the magnitude of the signal is larger then one. For the small values whose reliability is low which is left undecided. Due to this the interference is partly reduced, but when compared with hard-decision an improvised performance is seen.\\ \\
As we see that clipper still behaves more or less like hard-decision, hence the performance we obtain is quite far from optimum. So we go for another approach in which we employ tanh-function that will give more softer decisions then the above two methods. Most interesting thing in this approach is that we can model the tanh function in according to the level of interference we have in your system, such a modeling require exact level of interference to be known but in most of the cases it is quite unknown. So we introduce a parameter called $\alpha$ that depends on SNR as well as the effective interference.
\begin{equation}
\mathcal{Q}_{\mathrm{tan}}(y)=\mathrm{tanh}(\alpha y)
\end{equation}
\begin{figure}[htb]
\centerline{ \bildsc{ps/non_linear_dev.eps} {1.0} }
\caption{tahn for different $\alpha$ with hard decision and clipper}
%\label{Convolutional Encoder}
\end{figure}\\
\textbf{Figure 4.5} shows the tanh function with different values of $\alpha$. For small $\alpha$ function is very smooth and the values which we obtain are rather very un-reliable even for very large inputs. When $\alpha=1$ tanh function behaves like a clipper and for $\alpha>1$ function leads to hard-decision.
\subsection{Non-Linear Coded Interference Cancellation}
We extend our discussion from linear interference cancelers to non-linear interference cancelers for a coded CDMA system. The receiver part of such a system with parallel interference canceler is shown in \textbf{Figure 4.6}, in which the output of the matched filter $r_u$ is feed directly to the de-interleavers and followed by FEC decoder. The decoder delivers soft-outputs $L(\hat{b}_u)$ which then again interleaved and feed to non-linear function which limits the $L(\hat{b}_u)$ to 0 and 1. Such a limiting is required because the coded bits are either +1 or -1. The output from the non-linear function is weighted with the correlation coefficients $M_{u,v}$ and passed through the parallel interference canceler to get the estimates of the interference. 
\begin{figure}[htb]
\centerline{ \bildsc{ps/nonlinear_pic.eps} {1.0} }
\caption{Single Stage Non-Linear Parallel Interference Cancellation}
%\label{Convolutional Encoder}
\end{figure}
\begin{equation}
\hat{a}_v^{(\mu-1)}=\mathcal{Q}\left(M_{u,v}^{-1}\cdot \tilde{r}_v^{(\mu-1)}\right)
\end{equation}
These estimates are subtracted from the matched filter output and the process is repeated again until the global optimum is reached.
\begin{equation}
\tilde{r}_u^{(\mu)}=r_u-\sum\limits_{v\ne u}M_{u,v}\cdot \hat{a}_v^{(\mu-1)}
\end{equation}
After analzying various schemes we finally conclude that employing non-linear multi-user detection schemes with individual interleaver and FEC decoder and applying turbo principle is optimum in terms of computation cost to implement as a communication system. For such a system it is necessary to analyze the convergence behavior which is main goal in our project. So, in the next chapter we introduce various analysis tools which studies the convergence properties incorporating only parallel interference canceler. For to analyze successive interference cancellation we understand the dynamics of the extrinsic information which eventually lead to model it by a physical device. We explain that model in detail and see whether it is possible to have a breakthrough in reducing the dimensions of a problem pose when successive interference cancellation is employed.
