\chapter{ANT-Spezifisches}\label{CPTbeisp} 
%
In diesem Kapitel geht es um Features, die nicht zum Umfang der
\LaTeX-Distribution geh"oren, sondern von flei"sigen
ANT-Mitarbeitern\footnote{\bildsc{ANTonly}{0.134} und
\bildsc{ANTonly}{0.132}} ersonnen wurden, um sich und dem Rest der
ANT-Welt das Textverarbeitungs-Leben leichter zu machen. So definieren
die Dateien {\tt latex2e.tex} und {\tt mathe.tex} Befehle, die mit
hoher Wahrscheinlichkeit auch f"ur Dich n"utzlich sein werden. Sie
stehen allen Benutzern des Rechnernetzes im Arbeitsbereich automatisch
zur Verf"ugung, vorausgesetzt sie werden mit \verb+% #####################################################################################
% ##       latex2e.tex - LaTeX 2e-specific Definitions  by  Dieter Boss, 01/96       ##
% ##  -----------------------------------------------------------------------------  ##
% ##  The following commands are defined appropriately:                              ##
% ##  - \comment{}      : Multiple line comments                                     ##
% ##  - \bild{file}{1.0}: Include graphics and scale in multiples of "\textwidth"    ##
% ##  - \centerbild{...}: Include centered graphics                                  ##
% ##  - \bfm            : Bold face in math mode                                     ##
% ##  - \bf, \sf ...    : can be applied simultaneously (and work together!)         ##
% ##  -----------------------------------------------------------------------------  ##
% ##  Example of a LaTeX2e document header:                                          ##
% ##    \documentclass[twocolumn]{article}                                           ##
% ##    \usepackage{rotate,colordvi}  % Swap "colordvi" for "blackdvi" when printing ##
% ##    \usepackage[english]{babel}   %     a color LaTeX document on a B/W printer. ##
% ##    \usepackage{options,my9pt}    % ... and "showkeys" to view names of labels.  ##
% ##    % #####################################################################################
% ##       latex2e.tex - LaTeX 2e-specific Definitions  by  Dieter Boss, 01/96       ##
% ##  -----------------------------------------------------------------------------  ##
% ##  The following commands are defined appropriately:                              ##
% ##  - \comment{}      : Multiple line comments                                     ##
% ##  - \bild{file}{1.0}: Include graphics and scale in multiples of "\textwidth"    ##
% ##  - \centerbild{...}: Include centered graphics                                  ##
% ##  - \bfm            : Bold face in math mode                                     ##
% ##  - \bf, \sf ...    : can be applied simultaneously (and work together!)         ##
% ##  -----------------------------------------------------------------------------  ##
% ##  Example of a LaTeX2e document header:                                          ##
% ##    \documentclass[twocolumn]{article}                                           ##
% ##    \usepackage{rotate,colordvi}  % Swap "colordvi" for "blackdvi" when printing ##
% ##    \usepackage[english]{babel}   %     a color LaTeX document on a B/W printer. ##
% ##    \usepackage{options,my9pt}    % ... and "showkeys" to view names of labels.  ##
% ##    % #####################################################################################
% ##       latex2e.tex - LaTeX 2e-specific Definitions  by  Dieter Boss, 01/96       ##
% ##  -----------------------------------------------------------------------------  ##
% ##  The following commands are defined appropriately:                              ##
% ##  - \comment{}      : Multiple line comments                                     ##
% ##  - \bild{file}{1.0}: Include graphics and scale in multiples of "\textwidth"    ##
% ##  - \centerbild{...}: Include centered graphics                                  ##
% ##  - \bfm            : Bold face in math mode                                     ##
% ##  - \bf, \sf ...    : can be applied simultaneously (and work together!)         ##
% ##  -----------------------------------------------------------------------------  ##
% ##  Example of a LaTeX2e document header:                                          ##
% ##    \documentclass[twocolumn]{article}                                           ##
% ##    \usepackage{rotate,colordvi}  % Swap "colordvi" for "blackdvi" when printing ##
% ##    \usepackage[english]{babel}   %     a color LaTeX document on a B/W printer. ##
% ##    \usepackage{options,my9pt}    % ... and "showkeys" to view names of labels.  ##
% ##    \input{latex2e.tex}           % Include this file.                           ##
% ##    \input{mathe.tex}             % Must be included AFTER this file!            ##
% #####################################################################################

% Komfortables Auskommentieren ganzer LaTeX-Passagen
\newcommand{\comment}[1]{}

% Grafikeinbindung
\usepackage[dvips]{graphicx}
\newcommand{\bildwi}[2]{\includegraphics[width=#2\textwidth]{#1}}
\newcommand{\bildsc}[2]{\includegraphics[scale=#2]{#1}}
\newcommand{\bild}[2]{\includegraphics[width=#2\textwidth]{#1}}  % for compatibility
\newcommand{\centerbild}[1]{\centerline{#1}}                     % for compatibility

% Fettschrift im Mathemodus
\newcommand{\bfm}[1]{\mathbf{#1}}

% Definiere \bf, \sf etc. so um, dass sie sich bei gleichzeitiger Anwendung 
% sich nicht gegenseitig unwirksam machen.
\renewcommand{\rm}{\rmfamily}
\renewcommand{\sf}{\sffamily}
\renewcommand{\tt}{\ttfamily}
\renewcommand{\bf}{\bfseries}
\renewcommand{\it}{\itshape}
\renewcommand{\sl}{\slshape}
\renewcommand{\sc}{\scshape}
\newcommand{\md}{\mdseries}
\newcommand{\up}{\upshape}

% ###############  EOF  ###############
           % Include this file.                           ##
% ##    % mathe.tex - Mathematische (Mengen)Zeichen, Symbole und Funktionen
%             fuer LaTeX 2.09 und LaTeX 2e
%
%             Original von Dieter Boss, 08/94
%             Letzte Aenderung: H.Schmidt, 26-April-2000 (NDFT ... NIFFT)
%

\newcommand{\ds}[1]{\displaystyle{#1}}    % Kurzschreibweise fuer \displaystyle
\newcommand{\ml}[1]{\hbox{\large $#1$}}   % math large : Angenehme Schrift-
                                          % groesse bei der Darstellung von
                                          % Bruechen z.B. \ml{1\over\sqrt{2}}

% #####  Haeufig benoetigte Funktionen  #####
\newcommand{\E}[1]{\ensuremath{\mathrm{E}\left\{#1\right\}}}
\newcommand{\real}[1]{\ensuremath{\mathrm{Re}\left\{#1\right\}}}
\newcommand{\imag}[1]{\ensuremath{\mathrm{Im}\left\{#1\right\}}}
\newcommand{\rect}[1]{\ensuremath{\mathrm{rect}\left(#1\right)}}
\newcommand{\tri}[1]{\ensuremath{\mathrm{tri}\left(#1\right)}}
\newcommand{\si}{\ensuremath{\mathrm{si}}}
\newcommand{\di}{\ensuremath{\mathrm{di}}}
\newcommand{\ld}{\ensuremath{\mathrm{ld}}}  
\newcommand{\erf}{\ensuremath{\mathrm{erf}}}
\newcommand{\erfc}{\ensuremath{\mathrm{erfc}}}
\newcommand{\eds}[1]{\ensuremath{\mbox{e }^{\ds{#1}}}}
\newcommand{\ex}[1]{\ensuremath{e^{#1}}}
\newcommand{\ejO}{\ex{j\Omega}}

% #####  Mathematische Sonderzeichen  #####
\newcommand{\defas}{\ensuremath{\stackrel{\Delta}{=}}}

% #####  Mengenzeichen  #####
\newcommand{\Reell}{\mathsf{I} \kern -0.15em \mathsf{R}} 
\newcommand{\Nat}{\mathsf{I}  \kern -0.15em \mathsf{N}}
\newcommand{\Feld}{\mathsf{I} \kern -0.15em \mathsf{F}}
\newcommand{\Zahl}{\mathsf{Z} \kern -0.45em \mathsf{Z}}

% ##  Korrespondenz-"Knochen": ##
\newcommand{\korrespond}{\ensuremath{\;\circ \hskip-1ex -\hskip-1.2ex -\hskip-1.2ex- \hskip-1ex \bullet\;}}
\newcommand{\ikorrespond}{\ensuremath{\;\bullet \hskip-1ex -\hskip-1.2ex -\hskip-1.2ex- \hskip-1ex \circ\;}}

% #####  Transformationen  #####
\newcommand{\FT}[1]{\ensuremath{{\cal F}\left\{#1\right\}}}        % (kont.) Fourier-Trafo
\newcommand{\IFT}[1]{\ensuremath{{\cal F}^{-1}\left\{#1\right\}}}
\newcommand{\HT}[1]{\ensuremath{{\cal H}\left\{#1\right\}}}        % Hilbert-Trafo
\newcommand{\IHT}[1]{\ensuremath{{\cal H}^{-1}\left\{#1\right\}}}
\newcommand{\LT}[1]{\ensuremath{{\cal L}\left\{#1\right\}}}        % Laplace-Trafo
\newcommand{\ILT}[1]{\ensuremath{{\cal L}^{-1}\left\{#1\right\}}}
\newcommand{\DFT}[1]{\ensuremath{\mathrm{DFT}\left\{#1\right\}}}   % Diskrete Fourier-Trafo
\newcommand{\IDFT}[1]{\ensuremath{\mathrm{IDFT}\left\{#1\right\}}}
\newcommand{\FFT}[1]{\ensuremath{\mathrm{FFT}\left\{#1\right\}}}   % Fast Fourier-Trafo
\newcommand{\IFFT}[1]{\ensuremath{\mathrm{IFFT}\left\{#1\right\}}}
\newcommand{\NDFT}[2]{\ensuremath{\mathrm{DFT}_{#2}\left\{#1\right\}}}   % Diskrete Fourier-Trafo
\newcommand{\NIDFT}[2]{\ensuremath{\mathrm{IDFT}_{#2}\left\{#1\right\}}}
\newcommand{\NFFT}[2]{\ensuremath{\mathrm{FFT}_{#2}\left\{#1\right\}}}   % Fast Fourier-Trafo
\newcommand{\NIFFT}[2]{\ensuremath{\mathrm{IFFT}_{#2}\left\{#1\right\}}}\newcommand{\ZT}[1]{\ensuremath{{\cal Z}\left\{#1\right\}}}        % Z-Trafo
\newcommand{\IZT}[1]{\ensuremath{{\cal Z}^{-1}\left\{#1\right\}}}
\newcommand{\DTFT}[1]{\ensuremath{\mathrm{DTFT}\left\{#1\right\}}}   % Diskrete Time Fourier-Trafo
\newcommand{\IDTFT}[1]{\ensuremath{\mathrm{IDTFT}\left\{#1\right\}}}

% #####  Einheiten und Groessen  #####
\newcommand{\Hz}{\ensuremath{\mathrm{\:Hz}}}
\newcommand{\kHz}{\ensuremath{\mathrm{\:kHz}}}
\newcommand{\MHz}{\ensuremath{\mathrm{\:MHz}}}
\newcommand{\Mbits}{\ensuremath{\mathrm{\:Mbit/s}}}
\newcommand{\GHz}{\ensuremath{\mathrm{\:GHz}}}
\newcommand{\ms}{\ensuremath{\mathrm{\:ms}}}
\newcommand{\ns}{\ensuremath{\mathrm{\:ns}}}
\newcommand{\mus}{\ensuremath{\mathrm{\:\mu s}}}
\newcommand{\kmh}{\ensuremath{\mathrm{\:km/h}}}
\newcommand{\dB}{\ensuremath{\mathrm{\:dB}}}
\newcommand{\kbits}{\ensuremath{\mathrm{\:kbit/s}}}
\newcommand{\kBaud}{\ensuremath{\mathrm{\:kBaud}}}
\newcommand{\SNR}{\ensuremath{\frac{S}{N}}}
\newcommand{\EbN}{\ensuremath{\frac{E_b}{N_0}}}
\newcommand{\EbNh}{\ensuremath{\frac{E_b}{N_0/2}}}

% #####  Worte, die haeufig in Gleichungen gebraucht werden  #####
\newcommand{\Mit}{\quad\mathrm{mit}\;\,}          % kleingeschrieben existiert \mit schon!
\newcommand{\und}{\quad\mathrm{und}\;\,}
\newcommand{\da}{\quad\mathrm{da}\;\,}
\newcommand{\fuer}{\quad\mathrm{f"ur}\;\,}
\newcommand{\wobei}{\quad\mathrm{wobei}\;\,}
\newcommand{\mindex}[1]{\mbox{\scriptsize \sl #1}}

% #####  Fettschrift fuer Vektoren  #####
\newcommand{\vek}[1]{\ensuremath{\mathbf{#1}}}    % (fette) Vektoren oder Matrizen (mit Buchstabe als Argument)
\newcommand{\bs}[1]{\mbox{\boldmath$#1$}}         % (fette) schraege Vektoren oder Matrizen


% EOF
             % Must be included AFTER this file!            ##
% #####################################################################################

% Komfortables Auskommentieren ganzer LaTeX-Passagen
\newcommand{\comment}[1]{}

% Grafikeinbindung
\usepackage[dvips]{graphicx}
\newcommand{\bildwi}[2]{\includegraphics[width=#2\textwidth]{#1}}
\newcommand{\bildsc}[2]{\includegraphics[scale=#2]{#1}}
\newcommand{\bild}[2]{\includegraphics[width=#2\textwidth]{#1}}  % for compatibility
\newcommand{\centerbild}[1]{\centerline{#1}}                     % for compatibility

% Fettschrift im Mathemodus
\newcommand{\bfm}[1]{\mathbf{#1}}

% Definiere \bf, \sf etc. so um, dass sie sich bei gleichzeitiger Anwendung 
% sich nicht gegenseitig unwirksam machen.
\renewcommand{\rm}{\rmfamily}
\renewcommand{\sf}{\sffamily}
\renewcommand{\tt}{\ttfamily}
\renewcommand{\bf}{\bfseries}
\renewcommand{\it}{\itshape}
\renewcommand{\sl}{\slshape}
\renewcommand{\sc}{\scshape}
\newcommand{\md}{\mdseries}
\newcommand{\up}{\upshape}

% ###############  EOF  ###############
           % Include this file.                           ##
% ##    % mathe.tex - Mathematische (Mengen)Zeichen, Symbole und Funktionen
%             fuer LaTeX 2.09 und LaTeX 2e
%
%             Original von Dieter Boss, 08/94
%             Letzte Aenderung: H.Schmidt, 26-April-2000 (NDFT ... NIFFT)
%

\newcommand{\ds}[1]{\displaystyle{#1}}    % Kurzschreibweise fuer \displaystyle
\newcommand{\ml}[1]{\hbox{\large $#1$}}   % math large : Angenehme Schrift-
                                          % groesse bei der Darstellung von
                                          % Bruechen z.B. \ml{1\over\sqrt{2}}

% #####  Haeufig benoetigte Funktionen  #####
\newcommand{\E}[1]{\ensuremath{\mathrm{E}\left\{#1\right\}}}
\newcommand{\real}[1]{\ensuremath{\mathrm{Re}\left\{#1\right\}}}
\newcommand{\imag}[1]{\ensuremath{\mathrm{Im}\left\{#1\right\}}}
\newcommand{\rect}[1]{\ensuremath{\mathrm{rect}\left(#1\right)}}
\newcommand{\tri}[1]{\ensuremath{\mathrm{tri}\left(#1\right)}}
\newcommand{\si}{\ensuremath{\mathrm{si}}}
\newcommand{\di}{\ensuremath{\mathrm{di}}}
\newcommand{\ld}{\ensuremath{\mathrm{ld}}}  
\newcommand{\erf}{\ensuremath{\mathrm{erf}}}
\newcommand{\erfc}{\ensuremath{\mathrm{erfc}}}
\newcommand{\eds}[1]{\ensuremath{\mbox{e }^{\ds{#1}}}}
\newcommand{\ex}[1]{\ensuremath{e^{#1}}}
\newcommand{\ejO}{\ex{j\Omega}}

% #####  Mathematische Sonderzeichen  #####
\newcommand{\defas}{\ensuremath{\stackrel{\Delta}{=}}}

% #####  Mengenzeichen  #####
\newcommand{\Reell}{\mathsf{I} \kern -0.15em \mathsf{R}} 
\newcommand{\Nat}{\mathsf{I}  \kern -0.15em \mathsf{N}}
\newcommand{\Feld}{\mathsf{I} \kern -0.15em \mathsf{F}}
\newcommand{\Zahl}{\mathsf{Z} \kern -0.45em \mathsf{Z}}

% ##  Korrespondenz-"Knochen": ##
\newcommand{\korrespond}{\ensuremath{\;\circ \hskip-1ex -\hskip-1.2ex -\hskip-1.2ex- \hskip-1ex \bullet\;}}
\newcommand{\ikorrespond}{\ensuremath{\;\bullet \hskip-1ex -\hskip-1.2ex -\hskip-1.2ex- \hskip-1ex \circ\;}}

% #####  Transformationen  #####
\newcommand{\FT}[1]{\ensuremath{{\cal F}\left\{#1\right\}}}        % (kont.) Fourier-Trafo
\newcommand{\IFT}[1]{\ensuremath{{\cal F}^{-1}\left\{#1\right\}}}
\newcommand{\HT}[1]{\ensuremath{{\cal H}\left\{#1\right\}}}        % Hilbert-Trafo
\newcommand{\IHT}[1]{\ensuremath{{\cal H}^{-1}\left\{#1\right\}}}
\newcommand{\LT}[1]{\ensuremath{{\cal L}\left\{#1\right\}}}        % Laplace-Trafo
\newcommand{\ILT}[1]{\ensuremath{{\cal L}^{-1}\left\{#1\right\}}}
\newcommand{\DFT}[1]{\ensuremath{\mathrm{DFT}\left\{#1\right\}}}   % Diskrete Fourier-Trafo
\newcommand{\IDFT}[1]{\ensuremath{\mathrm{IDFT}\left\{#1\right\}}}
\newcommand{\FFT}[1]{\ensuremath{\mathrm{FFT}\left\{#1\right\}}}   % Fast Fourier-Trafo
\newcommand{\IFFT}[1]{\ensuremath{\mathrm{IFFT}\left\{#1\right\}}}
\newcommand{\NDFT}[2]{\ensuremath{\mathrm{DFT}_{#2}\left\{#1\right\}}}   % Diskrete Fourier-Trafo
\newcommand{\NIDFT}[2]{\ensuremath{\mathrm{IDFT}_{#2}\left\{#1\right\}}}
\newcommand{\NFFT}[2]{\ensuremath{\mathrm{FFT}_{#2}\left\{#1\right\}}}   % Fast Fourier-Trafo
\newcommand{\NIFFT}[2]{\ensuremath{\mathrm{IFFT}_{#2}\left\{#1\right\}}}\newcommand{\ZT}[1]{\ensuremath{{\cal Z}\left\{#1\right\}}}        % Z-Trafo
\newcommand{\IZT}[1]{\ensuremath{{\cal Z}^{-1}\left\{#1\right\}}}
\newcommand{\DTFT}[1]{\ensuremath{\mathrm{DTFT}\left\{#1\right\}}}   % Diskrete Time Fourier-Trafo
\newcommand{\IDTFT}[1]{\ensuremath{\mathrm{IDTFT}\left\{#1\right\}}}

% #####  Einheiten und Groessen  #####
\newcommand{\Hz}{\ensuremath{\mathrm{\:Hz}}}
\newcommand{\kHz}{\ensuremath{\mathrm{\:kHz}}}
\newcommand{\MHz}{\ensuremath{\mathrm{\:MHz}}}
\newcommand{\Mbits}{\ensuremath{\mathrm{\:Mbit/s}}}
\newcommand{\GHz}{\ensuremath{\mathrm{\:GHz}}}
\newcommand{\ms}{\ensuremath{\mathrm{\:ms}}}
\newcommand{\ns}{\ensuremath{\mathrm{\:ns}}}
\newcommand{\mus}{\ensuremath{\mathrm{\:\mu s}}}
\newcommand{\kmh}{\ensuremath{\mathrm{\:km/h}}}
\newcommand{\dB}{\ensuremath{\mathrm{\:dB}}}
\newcommand{\kbits}{\ensuremath{\mathrm{\:kbit/s}}}
\newcommand{\kBaud}{\ensuremath{\mathrm{\:kBaud}}}
\newcommand{\SNR}{\ensuremath{\frac{S}{N}}}
\newcommand{\EbN}{\ensuremath{\frac{E_b}{N_0}}}
\newcommand{\EbNh}{\ensuremath{\frac{E_b}{N_0/2}}}

% #####  Worte, die haeufig in Gleichungen gebraucht werden  #####
\newcommand{\Mit}{\quad\mathrm{mit}\;\,}          % kleingeschrieben existiert \mit schon!
\newcommand{\und}{\quad\mathrm{und}\;\,}
\newcommand{\da}{\quad\mathrm{da}\;\,}
\newcommand{\fuer}{\quad\mathrm{f"ur}\;\,}
\newcommand{\wobei}{\quad\mathrm{wobei}\;\,}
\newcommand{\mindex}[1]{\mbox{\scriptsize \sl #1}}

% #####  Fettschrift fuer Vektoren  #####
\newcommand{\vek}[1]{\ensuremath{\mathbf{#1}}}    % (fette) Vektoren oder Matrizen (mit Buchstabe als Argument)
\newcommand{\bs}[1]{\mbox{\boldmath$#1$}}         % (fette) schraege Vektoren oder Matrizen


% EOF
             % Must be included AFTER this file!            ##
% #####################################################################################

% Komfortables Auskommentieren ganzer LaTeX-Passagen
\newcommand{\comment}[1]{}

% Grafikeinbindung
\usepackage[dvips]{graphicx}
\newcommand{\bildwi}[2]{\includegraphics[width=#2\textwidth]{#1}}
\newcommand{\bildsc}[2]{\includegraphics[scale=#2]{#1}}
\newcommand{\bild}[2]{\includegraphics[width=#2\textwidth]{#1}}  % for compatibility
\newcommand{\centerbild}[1]{\centerline{#1}}                     % for compatibility

% Fettschrift im Mathemodus
\newcommand{\bfm}[1]{\mathbf{#1}}

% Definiere \bf, \sf etc. so um, dass sie sich bei gleichzeitiger Anwendung 
% sich nicht gegenseitig unwirksam machen.
\renewcommand{\rm}{\rmfamily}
\renewcommand{\sf}{\sffamily}
\renewcommand{\tt}{\ttfamily}
\renewcommand{\bf}{\bfseries}
\renewcommand{\it}{\itshape}
\renewcommand{\sl}{\slshape}
\renewcommand{\sc}{\scshape}
\newcommand{\md}{\mdseries}
\newcommand{\up}{\upshape}

% ###############  EOF  ###############
+
bzw.\ \verb+% mathe.tex - Mathematische (Mengen)Zeichen, Symbole und Funktionen
%             fuer LaTeX 2.09 und LaTeX 2e
%
%             Original von Dieter Boss, 08/94
%             Letzte Aenderung: H.Schmidt, 26-April-2000 (NDFT ... NIFFT)
%

\newcommand{\ds}[1]{\displaystyle{#1}}    % Kurzschreibweise fuer \displaystyle
\newcommand{\ml}[1]{\hbox{\large $#1$}}   % math large : Angenehme Schrift-
                                          % groesse bei der Darstellung von
                                          % Bruechen z.B. \ml{1\over\sqrt{2}}

% #####  Haeufig benoetigte Funktionen  #####
\newcommand{\E}[1]{\ensuremath{\mathrm{E}\left\{#1\right\}}}
\newcommand{\real}[1]{\ensuremath{\mathrm{Re}\left\{#1\right\}}}
\newcommand{\imag}[1]{\ensuremath{\mathrm{Im}\left\{#1\right\}}}
\newcommand{\rect}[1]{\ensuremath{\mathrm{rect}\left(#1\right)}}
\newcommand{\tri}[1]{\ensuremath{\mathrm{tri}\left(#1\right)}}
\newcommand{\si}{\ensuremath{\mathrm{si}}}
\newcommand{\di}{\ensuremath{\mathrm{di}}}
\newcommand{\ld}{\ensuremath{\mathrm{ld}}}  
\newcommand{\erf}{\ensuremath{\mathrm{erf}}}
\newcommand{\erfc}{\ensuremath{\mathrm{erfc}}}
\newcommand{\eds}[1]{\ensuremath{\mbox{e }^{\ds{#1}}}}
\newcommand{\ex}[1]{\ensuremath{e^{#1}}}
\newcommand{\ejO}{\ex{j\Omega}}

% #####  Mathematische Sonderzeichen  #####
\newcommand{\defas}{\ensuremath{\stackrel{\Delta}{=}}}

% #####  Mengenzeichen  #####
\newcommand{\Reell}{\mathsf{I} \kern -0.15em \mathsf{R}} 
\newcommand{\Nat}{\mathsf{I}  \kern -0.15em \mathsf{N}}
\newcommand{\Feld}{\mathsf{I} \kern -0.15em \mathsf{F}}
\newcommand{\Zahl}{\mathsf{Z} \kern -0.45em \mathsf{Z}}

% ##  Korrespondenz-"Knochen": ##
\newcommand{\korrespond}{\ensuremath{\;\circ \hskip-1ex -\hskip-1.2ex -\hskip-1.2ex- \hskip-1ex \bullet\;}}
\newcommand{\ikorrespond}{\ensuremath{\;\bullet \hskip-1ex -\hskip-1.2ex -\hskip-1.2ex- \hskip-1ex \circ\;}}

% #####  Transformationen  #####
\newcommand{\FT}[1]{\ensuremath{{\cal F}\left\{#1\right\}}}        % (kont.) Fourier-Trafo
\newcommand{\IFT}[1]{\ensuremath{{\cal F}^{-1}\left\{#1\right\}}}
\newcommand{\HT}[1]{\ensuremath{{\cal H}\left\{#1\right\}}}        % Hilbert-Trafo
\newcommand{\IHT}[1]{\ensuremath{{\cal H}^{-1}\left\{#1\right\}}}
\newcommand{\LT}[1]{\ensuremath{{\cal L}\left\{#1\right\}}}        % Laplace-Trafo
\newcommand{\ILT}[1]{\ensuremath{{\cal L}^{-1}\left\{#1\right\}}}
\newcommand{\DFT}[1]{\ensuremath{\mathrm{DFT}\left\{#1\right\}}}   % Diskrete Fourier-Trafo
\newcommand{\IDFT}[1]{\ensuremath{\mathrm{IDFT}\left\{#1\right\}}}
\newcommand{\FFT}[1]{\ensuremath{\mathrm{FFT}\left\{#1\right\}}}   % Fast Fourier-Trafo
\newcommand{\IFFT}[1]{\ensuremath{\mathrm{IFFT}\left\{#1\right\}}}
\newcommand{\NDFT}[2]{\ensuremath{\mathrm{DFT}_{#2}\left\{#1\right\}}}   % Diskrete Fourier-Trafo
\newcommand{\NIDFT}[2]{\ensuremath{\mathrm{IDFT}_{#2}\left\{#1\right\}}}
\newcommand{\NFFT}[2]{\ensuremath{\mathrm{FFT}_{#2}\left\{#1\right\}}}   % Fast Fourier-Trafo
\newcommand{\NIFFT}[2]{\ensuremath{\mathrm{IFFT}_{#2}\left\{#1\right\}}}\newcommand{\ZT}[1]{\ensuremath{{\cal Z}\left\{#1\right\}}}        % Z-Trafo
\newcommand{\IZT}[1]{\ensuremath{{\cal Z}^{-1}\left\{#1\right\}}}
\newcommand{\DTFT}[1]{\ensuremath{\mathrm{DTFT}\left\{#1\right\}}}   % Diskrete Time Fourier-Trafo
\newcommand{\IDTFT}[1]{\ensuremath{\mathrm{IDTFT}\left\{#1\right\}}}

% #####  Einheiten und Groessen  #####
\newcommand{\Hz}{\ensuremath{\mathrm{\:Hz}}}
\newcommand{\kHz}{\ensuremath{\mathrm{\:kHz}}}
\newcommand{\MHz}{\ensuremath{\mathrm{\:MHz}}}
\newcommand{\Mbits}{\ensuremath{\mathrm{\:Mbit/s}}}
\newcommand{\GHz}{\ensuremath{\mathrm{\:GHz}}}
\newcommand{\ms}{\ensuremath{\mathrm{\:ms}}}
\newcommand{\ns}{\ensuremath{\mathrm{\:ns}}}
\newcommand{\mus}{\ensuremath{\mathrm{\:\mu s}}}
\newcommand{\kmh}{\ensuremath{\mathrm{\:km/h}}}
\newcommand{\dB}{\ensuremath{\mathrm{\:dB}}}
\newcommand{\kbits}{\ensuremath{\mathrm{\:kbit/s}}}
\newcommand{\kBaud}{\ensuremath{\mathrm{\:kBaud}}}
\newcommand{\SNR}{\ensuremath{\frac{S}{N}}}
\newcommand{\EbN}{\ensuremath{\frac{E_b}{N_0}}}
\newcommand{\EbNh}{\ensuremath{\frac{E_b}{N_0/2}}}

% #####  Worte, die haeufig in Gleichungen gebraucht werden  #####
\newcommand{\Mit}{\quad\mathrm{mit}\;\,}          % kleingeschrieben existiert \mit schon!
\newcommand{\und}{\quad\mathrm{und}\;\,}
\newcommand{\da}{\quad\mathrm{da}\;\,}
\newcommand{\fuer}{\quad\mathrm{f"ur}\;\,}
\newcommand{\wobei}{\quad\mathrm{wobei}\;\,}
\newcommand{\mindex}[1]{\mbox{\scriptsize \sl #1}}

% #####  Fettschrift fuer Vektoren  #####
\newcommand{\vek}[1]{\ensuremath{\mathbf{#1}}}    % (fette) Vektoren oder Matrizen (mit Buchstabe als Argument)
\newcommand{\bs}[1]{\mbox{\boldmath$#1$}}         % (fette) schraege Vektoren oder Matrizen


% EOF
+ eingebunden (wie es auch hier der Fall
ist).  Dabei kann {\tt mathe.tex}  erst {\em hinter} {\tt latex2e.tex}
eingebunden werden, in der umgekehrten Reihenfolge funktioniert es
nicht!

Auch hier gilt, da"s auf Anregung der Benutzer weitere Elemente
eingebracht werden k"onnen. Die "Anderungen kann allerdings nicht jeder
Fuzzi eigenst"andig vornehmen (ein Gl"uck), er sollte sich bei Bedarf
an eine kompetente Person wenden.


\section{Die Datei {\tt latex2e.tex}} \label{SECbild}

\subsection{L"angere Kommentare}
%
\LaTeX\ erlaubt das Auskommentieren von Passagen aus dem \LaTeX-File
nur durch voranstellen eines Prozentzeichens. Dies kommentiert allerdings
nur den Test hinter dem Prozentzeichen {\em bis zum n"achsten Zeilenumbruch} aus.
Will man ganze Abschnitte auskommentieren, so m"u"ste man jeder Zeile
m"uhsam ein Prozentzeichen verpassen --~wenn es da nicht den Befehl
\verb+\comment{...}+ aus der Datei {\tt latex2e.tex} g"abe:
Damit sind Kommentare f"ur l"angere Textpassagen komfortabel zu
erreichen. Die (nicht) gew"unschte Textstelle einfach in die geschweiften 
Klammern des Befehls setzen --- der Text wird dann beim Compilieren nicht mehr
ber"ucksichtigt. Bei l"angeren Auskommentierungen ist allerdings sehr
zu empfehlen, das Ende des Kommentars nicht nur durch eine geschweifte Klammer
zu kennzeichnen (die man sp"ater nie wieder findet, bei soo vielen Klammern),
sondern durch eine Zeile wie z.B.\ ``\verb+} % END OF \comment+''. So hat man auch
mit einem ``dummen'' Texteditor die Chance, das Kommentarende wiederzufinden.

\subsection{Einbinden von Grafiken} \label{SUBSECbilder}
%
Die Datei {\tt latex2e.tex} definiert auch Befehle zur Grafikeinbindung.
So binden die folgenden Zeilen das Postscript-Bild {\tt muster.pst} im 
Directory {\tt PS} in das \LaTeX-Dokument ein.
%
\begin{figure}[htb]
  \centerline{ \bildsc{PS/muster.pst}{0.48} }
  \caption{Fehlerraten. Anmerkung:
           Mehrzeilige Bildunterschriften sehen "ubrigens auch super
           aus, es sei denn, man w"ahlt \emph{nicht} die hier voreingestellte
           Konfiguration, sondern versucht es auf eigene Faust \ldots}
  \label{FIGmust}
\end{figure}
%
\begin{verbatim}
\begin{figure}[htb]
  \centerline{ \bildsc{PS/muster.pst}{0.48} }
  \caption{Fehlerraten. Anmerkung: ...}
  \label{FIGmust}
\end{figure}
\end{verbatim}
%
Dazu wird der Befehl \verb+\bildsc{FILENAME}{0.48}+ verwendet,
der das Bild auf 48\% seiner urspr"unglichen Gr"o"se skaliert.
W"urde man statt \verb+\bildsc+ den Befehl \verb+\bildwi+ benutzen,
so w"urde das Bild auf 48\% der Textbreite einer Seite 
skaliert\footnote{Eselsbr"ucke:
{\tt bildsc} kommt von {\em scale}, bei {\tt bildwi} wird auf feste {\em width} skaliert.}.\\
Mit \verb+\caption{xxx}+ wird der Abbildung eine Bildnummer und die Bildunterschrift
{\tt xxx} zugewiesen. Definiert man mit \verb+\label{FIGmust}+ ein label,
so kann aus dem Text durch \verb+\ref{FIGmust}+ auf die Bildnummer zugegriffen werden
(s.a.\ Abschnitt~\ref{SECverw}).

{\bf Bild-Plazierung:} Da die {\tt figure}-Umgebung ebenso wie die
Tabellen-Umgebung {\tt table} ein {\em floating object} definiert,
gelten f"ur die Plazierung von Bildern die in Abschnitt~\ref{SECtab}
f"ur Tabellen getroffenen Aussagen.  Im allgemeinen l"a"st sich der
Plazierungswunsch mit umso geringerer Wahrscheinlichkeit erf"ullen, je
gr"o"ser das Gleitobjekt ist. Mal sehen, wo Bild~\ref{FIGmust}
landet...

Mehrere Bilder bindet man wie folgt ein. Um die vier Bilder {\tt
b1.eps} $\cdots$ {\tt b4.eps} in einer $2 \times 2$ Matrix-Anordnung
einzubinden, k"onnte man folgende Konstruktion verwenden:

\begin{verbatim}
  \centerline{ \bildsc{PS/b1.eps}{0.16}\quad\bildsc{PS/b2.eps}{0.16} }
  \centerline{ \bildsc{PS/b3.eps}{0.16}\quad\bildsc{PS/b4.eps}{0.16} }
\end{verbatim}

Je nach IQ des Lesers d"urften somit auch andere Bildanordnungen (k)ein Problem sein ...

\subsection{Fettschrift im Mathemodus}

Weiterhin wird in {\tt latex2e.tex} das Kommando \verb+\bfm+ definiert,
das in seiner Form \verb+{\bfm H}+ im Mathemodus ein fettgedrucktes `H' darstellt.



\section{Die Datei {\tt mathe.tex}} \label{SUBmathe}
%
Die Datei {\tt mathe.tex} ist als ASCII-Text in Anhang~\ref{CPTmehrascii}
auf Seite~\pageref{CPTmehrascii} eingebunden und sollte vom interessierten Nutzer
aufmerksam gelesen werden. In Tabelle~\ref{TABmathe} sind die wichtigsten Definitionen
und ihre Wirkungen (im Mathemodus) angegeben.
%
\begin{table}[htb]\label{TABmathe}
\begin{center}
\begin{tabular}{|l|l|l|}
  \hline
  Thema               & Befehl                                    & Wirkung (im Mathemodus)\\
  \hline
  Komplexe Zahlen     & \verb+ \real{2j}, \imag{3j}             + & $\real{2j}, \imag{3j}$\\
                      & \verb+ \ejO, \ex{nix}                   + & $\ejO, \ex{nix}$\\
  Funktionen          & \verb+ \rect{T_0}, \tri{\frac{T}{2}}    + & $\rect{T_0}, \tri{\frac{T}{2}}$\\
                      & \verb+ \si(\omega t), \ld(2)            + & $\si(\omega t), \ld(2)$\\
  Einheiten           & \verb+ 4\kHz, 3\MHz, 2\mus, 1\ms        + & $4\kHz, 3\MHz, 2\mus, 1\ms$\\
                      & \verb+ 3\dB                             + & $3\dB$\\
  Vek.~\& Matrizen    & \verb+ \vek{a}, \vek{A}                 + & $\vek{a}, \vek{A}$\\
  Transformationen    & \verb+ f \korrespond F, F \ikorrespond f+ & $f \korrespond F, F \ikorrespond f$\\
                      & \verb+ \FT{K\,\delta(t)}=K              + & $\FT{K\,\delta(t)}=K$\\
                      & \verb+ \IFT{2\pi K\,\delta(\omega)}=K   + & $\IFT{2\pi K\,\delta(\omega)}=K$\\
                      & \verb+ \DFT{.}, \IDFT{.}                + & $\DFT{.}, \IDFT{.}$\\
                      & \verb+ \FFT{.}, \IFFT{.}                + & $\FFT{.}, \IFFT{.}$\\
                      & \verb+ \ZT{.},\IZT{.},\HT{.},\IHT{.}    + & $\ZT{.}, \IZT{.}, \HT{.}, \IHT{.}$\\
  K"urzel             & \verb+ \SNR, \EbN                       + & $\SNR, \EbN$\\
  \hline
\end{tabular}
\end{center}
\end{table}

Wie man am Beispiel des Befehls \verb+\tri{.}+ erkennen kann, sind
alle Kommandos, an die ein Argument "ubergeben wird, so definiert,
da"s sich die Gr"o"se der geschweiften Klammern automatisch an die 
Gr"o"se des Argumentes anpa"st.

In Gleichungen ben"otigt man des "ofteren bestimmte W"orter, die im
normalen Textmodus erscheinen sollen, wie es im folgenden Beispiel 
beim Wort ``f"ur'' der Fall ist:\ \ $x=\sqrt{\lambda} \fuer \lambda\leq 0$.
Deshalb sind in {\tt mathe.tex} die Befehle \verb+\und+, 
\verb+\da+, \verb+fuer+, \verb+\Mit+ und \verb+\wobei+ 
definiert, die einen Zwischenraum lassen (\verb+\quad+), dann das entsprechende
Wort im Textmodus darstellen und danach noch einen kleinen Zwischenraum lassen 
(\verb+\;\,+).

% EOF
