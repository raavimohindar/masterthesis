\chapter{Conclusion}
The single-parameter dynamical model representation of coded CDMA paved the way for various analysis tools such as EXIT charts, Multi-User Efficiencies and Variance Transfer characteristics in which single parameter such as mutual information, effective interference and the variance of the estimation error solely characterize the system. Constructing an iterative decoder and employing the turbo principle and exchanging these parameters during iterations leads to single-parameter dynamical model. With that we predict the system very tightly and study the minimum requirements for a system to convergence and also for the system to reach the single-user bound. We have studied the characteristics of half and full loaded system with various signal-to-noise rations and also we compare the requirements for a fully loaded system to converge with the half-loaded system. We also made the comparative study between the EXIT charts and Variance Transfer Characteristics and we saw that these analysis tools characterizing the system in same way. Multi-User Efficiency works well for the large system limits we showed the plots for a system having loads more then one.\\ \\
We find the reason behind the cyclic behavior in the characteristic curves by studying an analogous model, in which we introduced the dynamics of a rotating disk and from that we conclude that to have smooth predictions we must employ the extrinsic part of the information. \\ \\
All these characteristic curve which we saw in the previous chapters are obtained only for the parallel-interference canceler. As we mentioned previously, analyzing with successive-interference cancellation leads to N-dimensional problem, that is for successive-interference canceler the characteristic curves cannot be plotted in two-dimensional plot and it remains a open problem till today. \\ \\
