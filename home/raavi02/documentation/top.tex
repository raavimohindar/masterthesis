% ##############################################################################
% ##     LaTeX2e-Muster-Datei fuer Projekt-, Studien- und Diplomarbeiten      ##
% ##  ----------------------------------------------------------------------  ##
% ##     Dieter Boss, Dirk Nikolai, Uni Bremen, FB1, Nachrichtentechnik       ##
% ##                        Version 1.2,  20.05.96                            ##
% ##############################################################################
%
% Benoetigte LaTeX2e-Packages:        - twocolumn          - 11pt
%                                     - babel[german]      - amstex
% Benoetigte LaTeX209-Packages:       - fancyheadings      - epsfig
%                                     - verbatimfiles      - options
% Benoetigte input-files:             - latex2e.tex        - mathe.tex
% Benoetigte Bilder (im Ordner PS/):  - musterBER.pst      - musterbild1.eps
%                                     - musterbild2.eps    - ...
%                                     - ...                - ...
%

\documentclass[11pt]{report}           % Seitennumerierung immer rechts
%\documentclass[twoside,11pt]{report}  % Seitennumerierung im Wechsel rechts 
                                       % und links

% ##########  Inputenc package   ######################
\usepackage[latin1]{inputenc}          % deutsche Umlaute in der Latin1-Codierung
                                       % z.B. "�������"

% ##########  German package   ######################
%\usepackage[german]{babel}             % Fuer deutsche Umlaute (ae="a, sz="s) und
                                       % Ueberschriften ("Inhaltsverzeichnis")

% ##########  Showkeys package   ####################
\usepackage{showkeys}       	       % Die Namen von allen Labels und Referenzen werden
                                       % im Dokument ausgedruckt. Diese Package muss beim 
                                       % endgueltigen Dokument auskommentiert werden!

% ##########  LaTeX2e changes  ######################
% #####################################################################################
% ##       latex2e.tex - LaTeX 2e-specific Definitions  by  Dieter Boss, 01/96       ##
% ##  -----------------------------------------------------------------------------  ##
% ##  The following commands are defined appropriately:                              ##
% ##  - \comment{}      : Multiple line comments                                     ##
% ##  - \bild{file}{1.0}: Include graphics and scale in multiples of "\textwidth"    ##
% ##  - \centerbild{...}: Include centered graphics                                  ##
% ##  - \bfm            : Bold face in math mode                                     ##
% ##  - \bf, \sf ...    : can be applied simultaneously (and work together!)         ##
% ##  -----------------------------------------------------------------------------  ##
% ##  Example of a LaTeX2e document header:                                          ##
% ##    \documentclass[twocolumn]{article}                                           ##
% ##    \usepackage{rotate,colordvi}  % Swap "colordvi" for "blackdvi" when printing ##
% ##    \usepackage[english]{babel}   %     a color LaTeX document on a B/W printer. ##
% ##    \usepackage{options,my9pt}    % ... and "showkeys" to view names of labels.  ##
% ##    % #####################################################################################
% ##       latex2e.tex - LaTeX 2e-specific Definitions  by  Dieter Boss, 01/96       ##
% ##  -----------------------------------------------------------------------------  ##
% ##  The following commands are defined appropriately:                              ##
% ##  - \comment{}      : Multiple line comments                                     ##
% ##  - \bild{file}{1.0}: Include graphics and scale in multiples of "\textwidth"    ##
% ##  - \centerbild{...}: Include centered graphics                                  ##
% ##  - \bfm            : Bold face in math mode                                     ##
% ##  - \bf, \sf ...    : can be applied simultaneously (and work together!)         ##
% ##  -----------------------------------------------------------------------------  ##
% ##  Example of a LaTeX2e document header:                                          ##
% ##    \documentclass[twocolumn]{article}                                           ##
% ##    \usepackage{rotate,colordvi}  % Swap "colordvi" for "blackdvi" when printing ##
% ##    \usepackage[english]{babel}   %     a color LaTeX document on a B/W printer. ##
% ##    \usepackage{options,my9pt}    % ... and "showkeys" to view names of labels.  ##
% ##    % #####################################################################################
% ##       latex2e.tex - LaTeX 2e-specific Definitions  by  Dieter Boss, 01/96       ##
% ##  -----------------------------------------------------------------------------  ##
% ##  The following commands are defined appropriately:                              ##
% ##  - \comment{}      : Multiple line comments                                     ##
% ##  - \bild{file}{1.0}: Include graphics and scale in multiples of "\textwidth"    ##
% ##  - \centerbild{...}: Include centered graphics                                  ##
% ##  - \bfm            : Bold face in math mode                                     ##
% ##  - \bf, \sf ...    : can be applied simultaneously (and work together!)         ##
% ##  -----------------------------------------------------------------------------  ##
% ##  Example of a LaTeX2e document header:                                          ##
% ##    \documentclass[twocolumn]{article}                                           ##
% ##    \usepackage{rotate,colordvi}  % Swap "colordvi" for "blackdvi" when printing ##
% ##    \usepackage[english]{babel}   %     a color LaTeX document on a B/W printer. ##
% ##    \usepackage{options,my9pt}    % ... and "showkeys" to view names of labels.  ##
% ##    \input{latex2e.tex}           % Include this file.                           ##
% ##    \input{mathe.tex}             % Must be included AFTER this file!            ##
% #####################################################################################

% Komfortables Auskommentieren ganzer LaTeX-Passagen
\newcommand{\comment}[1]{}

% Grafikeinbindung
\usepackage[dvips]{graphicx}
\newcommand{\bildwi}[2]{\includegraphics[width=#2\textwidth]{#1}}
\newcommand{\bildsc}[2]{\includegraphics[scale=#2]{#1}}
\newcommand{\bild}[2]{\includegraphics[width=#2\textwidth]{#1}}  % for compatibility
\newcommand{\centerbild}[1]{\centerline{#1}}                     % for compatibility

% Fettschrift im Mathemodus
\newcommand{\bfm}[1]{\mathbf{#1}}

% Definiere \bf, \sf etc. so um, dass sie sich bei gleichzeitiger Anwendung 
% sich nicht gegenseitig unwirksam machen.
\renewcommand{\rm}{\rmfamily}
\renewcommand{\sf}{\sffamily}
\renewcommand{\tt}{\ttfamily}
\renewcommand{\bf}{\bfseries}
\renewcommand{\it}{\itshape}
\renewcommand{\sl}{\slshape}
\renewcommand{\sc}{\scshape}
\newcommand{\md}{\mdseries}
\newcommand{\up}{\upshape}

% ###############  EOF  ###############
           % Include this file.                           ##
% ##    % mathe.tex - Mathematische (Mengen)Zeichen, Symbole und Funktionen
%             fuer LaTeX 2.09 und LaTeX 2e
%
%             Original von Dieter Boss, 08/94
%             Letzte Aenderung: H.Schmidt, 26-April-2000 (NDFT ... NIFFT)
%

\newcommand{\ds}[1]{\displaystyle{#1}}    % Kurzschreibweise fuer \displaystyle
\newcommand{\ml}[1]{\hbox{\large $#1$}}   % math large : Angenehme Schrift-
                                          % groesse bei der Darstellung von
                                          % Bruechen z.B. \ml{1\over\sqrt{2}}

% #####  Haeufig benoetigte Funktionen  #####
\newcommand{\E}[1]{\ensuremath{\mathrm{E}\left\{#1\right\}}}
\newcommand{\real}[1]{\ensuremath{\mathrm{Re}\left\{#1\right\}}}
\newcommand{\imag}[1]{\ensuremath{\mathrm{Im}\left\{#1\right\}}}
\newcommand{\rect}[1]{\ensuremath{\mathrm{rect}\left(#1\right)}}
\newcommand{\tri}[1]{\ensuremath{\mathrm{tri}\left(#1\right)}}
\newcommand{\si}{\ensuremath{\mathrm{si}}}
\newcommand{\di}{\ensuremath{\mathrm{di}}}
\newcommand{\ld}{\ensuremath{\mathrm{ld}}}  
\newcommand{\erf}{\ensuremath{\mathrm{erf}}}
\newcommand{\erfc}{\ensuremath{\mathrm{erfc}}}
\newcommand{\eds}[1]{\ensuremath{\mbox{e }^{\ds{#1}}}}
\newcommand{\ex}[1]{\ensuremath{e^{#1}}}
\newcommand{\ejO}{\ex{j\Omega}}

% #####  Mathematische Sonderzeichen  #####
\newcommand{\defas}{\ensuremath{\stackrel{\Delta}{=}}}

% #####  Mengenzeichen  #####
\newcommand{\Reell}{\mathsf{I} \kern -0.15em \mathsf{R}} 
\newcommand{\Nat}{\mathsf{I}  \kern -0.15em \mathsf{N}}
\newcommand{\Feld}{\mathsf{I} \kern -0.15em \mathsf{F}}
\newcommand{\Zahl}{\mathsf{Z} \kern -0.45em \mathsf{Z}}

% ##  Korrespondenz-"Knochen": ##
\newcommand{\korrespond}{\ensuremath{\;\circ \hskip-1ex -\hskip-1.2ex -\hskip-1.2ex- \hskip-1ex \bullet\;}}
\newcommand{\ikorrespond}{\ensuremath{\;\bullet \hskip-1ex -\hskip-1.2ex -\hskip-1.2ex- \hskip-1ex \circ\;}}

% #####  Transformationen  #####
\newcommand{\FT}[1]{\ensuremath{{\cal F}\left\{#1\right\}}}        % (kont.) Fourier-Trafo
\newcommand{\IFT}[1]{\ensuremath{{\cal F}^{-1}\left\{#1\right\}}}
\newcommand{\HT}[1]{\ensuremath{{\cal H}\left\{#1\right\}}}        % Hilbert-Trafo
\newcommand{\IHT}[1]{\ensuremath{{\cal H}^{-1}\left\{#1\right\}}}
\newcommand{\LT}[1]{\ensuremath{{\cal L}\left\{#1\right\}}}        % Laplace-Trafo
\newcommand{\ILT}[1]{\ensuremath{{\cal L}^{-1}\left\{#1\right\}}}
\newcommand{\DFT}[1]{\ensuremath{\mathrm{DFT}\left\{#1\right\}}}   % Diskrete Fourier-Trafo
\newcommand{\IDFT}[1]{\ensuremath{\mathrm{IDFT}\left\{#1\right\}}}
\newcommand{\FFT}[1]{\ensuremath{\mathrm{FFT}\left\{#1\right\}}}   % Fast Fourier-Trafo
\newcommand{\IFFT}[1]{\ensuremath{\mathrm{IFFT}\left\{#1\right\}}}
\newcommand{\NDFT}[2]{\ensuremath{\mathrm{DFT}_{#2}\left\{#1\right\}}}   % Diskrete Fourier-Trafo
\newcommand{\NIDFT}[2]{\ensuremath{\mathrm{IDFT}_{#2}\left\{#1\right\}}}
\newcommand{\NFFT}[2]{\ensuremath{\mathrm{FFT}_{#2}\left\{#1\right\}}}   % Fast Fourier-Trafo
\newcommand{\NIFFT}[2]{\ensuremath{\mathrm{IFFT}_{#2}\left\{#1\right\}}}\newcommand{\ZT}[1]{\ensuremath{{\cal Z}\left\{#1\right\}}}        % Z-Trafo
\newcommand{\IZT}[1]{\ensuremath{{\cal Z}^{-1}\left\{#1\right\}}}
\newcommand{\DTFT}[1]{\ensuremath{\mathrm{DTFT}\left\{#1\right\}}}   % Diskrete Time Fourier-Trafo
\newcommand{\IDTFT}[1]{\ensuremath{\mathrm{IDTFT}\left\{#1\right\}}}

% #####  Einheiten und Groessen  #####
\newcommand{\Hz}{\ensuremath{\mathrm{\:Hz}}}
\newcommand{\kHz}{\ensuremath{\mathrm{\:kHz}}}
\newcommand{\MHz}{\ensuremath{\mathrm{\:MHz}}}
\newcommand{\Mbits}{\ensuremath{\mathrm{\:Mbit/s}}}
\newcommand{\GHz}{\ensuremath{\mathrm{\:GHz}}}
\newcommand{\ms}{\ensuremath{\mathrm{\:ms}}}
\newcommand{\ns}{\ensuremath{\mathrm{\:ns}}}
\newcommand{\mus}{\ensuremath{\mathrm{\:\mu s}}}
\newcommand{\kmh}{\ensuremath{\mathrm{\:km/h}}}
\newcommand{\dB}{\ensuremath{\mathrm{\:dB}}}
\newcommand{\kbits}{\ensuremath{\mathrm{\:kbit/s}}}
\newcommand{\kBaud}{\ensuremath{\mathrm{\:kBaud}}}
\newcommand{\SNR}{\ensuremath{\frac{S}{N}}}
\newcommand{\EbN}{\ensuremath{\frac{E_b}{N_0}}}
\newcommand{\EbNh}{\ensuremath{\frac{E_b}{N_0/2}}}

% #####  Worte, die haeufig in Gleichungen gebraucht werden  #####
\newcommand{\Mit}{\quad\mathrm{mit}\;\,}          % kleingeschrieben existiert \mit schon!
\newcommand{\und}{\quad\mathrm{und}\;\,}
\newcommand{\da}{\quad\mathrm{da}\;\,}
\newcommand{\fuer}{\quad\mathrm{f"ur}\;\,}
\newcommand{\wobei}{\quad\mathrm{wobei}\;\,}
\newcommand{\mindex}[1]{\mbox{\scriptsize \sl #1}}

% #####  Fettschrift fuer Vektoren  #####
\newcommand{\vek}[1]{\ensuremath{\mathbf{#1}}}    % (fette) Vektoren oder Matrizen (mit Buchstabe als Argument)
\newcommand{\bs}[1]{\mbox{\boldmath$#1$}}         % (fette) schraege Vektoren oder Matrizen


% EOF
             % Must be included AFTER this file!            ##
% #####################################################################################

% Komfortables Auskommentieren ganzer LaTeX-Passagen
\newcommand{\comment}[1]{}

% Grafikeinbindung
\usepackage[dvips]{graphicx}
\newcommand{\bildwi}[2]{\includegraphics[width=#2\textwidth]{#1}}
\newcommand{\bildsc}[2]{\includegraphics[scale=#2]{#1}}
\newcommand{\bild}[2]{\includegraphics[width=#2\textwidth]{#1}}  % for compatibility
\newcommand{\centerbild}[1]{\centerline{#1}}                     % for compatibility

% Fettschrift im Mathemodus
\newcommand{\bfm}[1]{\mathbf{#1}}

% Definiere \bf, \sf etc. so um, dass sie sich bei gleichzeitiger Anwendung 
% sich nicht gegenseitig unwirksam machen.
\renewcommand{\rm}{\rmfamily}
\renewcommand{\sf}{\sffamily}
\renewcommand{\tt}{\ttfamily}
\renewcommand{\bf}{\bfseries}
\renewcommand{\it}{\itshape}
\renewcommand{\sl}{\slshape}
\renewcommand{\sc}{\scshape}
\newcommand{\md}{\mdseries}
\newcommand{\up}{\upshape}

% ###############  EOF  ###############
           % Include this file.                           ##
% ##    % mathe.tex - Mathematische (Mengen)Zeichen, Symbole und Funktionen
%             fuer LaTeX 2.09 und LaTeX 2e
%
%             Original von Dieter Boss, 08/94
%             Letzte Aenderung: H.Schmidt, 26-April-2000 (NDFT ... NIFFT)
%

\newcommand{\ds}[1]{\displaystyle{#1}}    % Kurzschreibweise fuer \displaystyle
\newcommand{\ml}[1]{\hbox{\large $#1$}}   % math large : Angenehme Schrift-
                                          % groesse bei der Darstellung von
                                          % Bruechen z.B. \ml{1\over\sqrt{2}}

% #####  Haeufig benoetigte Funktionen  #####
\newcommand{\E}[1]{\ensuremath{\mathrm{E}\left\{#1\right\}}}
\newcommand{\real}[1]{\ensuremath{\mathrm{Re}\left\{#1\right\}}}
\newcommand{\imag}[1]{\ensuremath{\mathrm{Im}\left\{#1\right\}}}
\newcommand{\rect}[1]{\ensuremath{\mathrm{rect}\left(#1\right)}}
\newcommand{\tri}[1]{\ensuremath{\mathrm{tri}\left(#1\right)}}
\newcommand{\si}{\ensuremath{\mathrm{si}}}
\newcommand{\di}{\ensuremath{\mathrm{di}}}
\newcommand{\ld}{\ensuremath{\mathrm{ld}}}  
\newcommand{\erf}{\ensuremath{\mathrm{erf}}}
\newcommand{\erfc}{\ensuremath{\mathrm{erfc}}}
\newcommand{\eds}[1]{\ensuremath{\mbox{e }^{\ds{#1}}}}
\newcommand{\ex}[1]{\ensuremath{e^{#1}}}
\newcommand{\ejO}{\ex{j\Omega}}

% #####  Mathematische Sonderzeichen  #####
\newcommand{\defas}{\ensuremath{\stackrel{\Delta}{=}}}

% #####  Mengenzeichen  #####
\newcommand{\Reell}{\mathsf{I} \kern -0.15em \mathsf{R}} 
\newcommand{\Nat}{\mathsf{I}  \kern -0.15em \mathsf{N}}
\newcommand{\Feld}{\mathsf{I} \kern -0.15em \mathsf{F}}
\newcommand{\Zahl}{\mathsf{Z} \kern -0.45em \mathsf{Z}}

% ##  Korrespondenz-"Knochen": ##
\newcommand{\korrespond}{\ensuremath{\;\circ \hskip-1ex -\hskip-1.2ex -\hskip-1.2ex- \hskip-1ex \bullet\;}}
\newcommand{\ikorrespond}{\ensuremath{\;\bullet \hskip-1ex -\hskip-1.2ex -\hskip-1.2ex- \hskip-1ex \circ\;}}

% #####  Transformationen  #####
\newcommand{\FT}[1]{\ensuremath{{\cal F}\left\{#1\right\}}}        % (kont.) Fourier-Trafo
\newcommand{\IFT}[1]{\ensuremath{{\cal F}^{-1}\left\{#1\right\}}}
\newcommand{\HT}[1]{\ensuremath{{\cal H}\left\{#1\right\}}}        % Hilbert-Trafo
\newcommand{\IHT}[1]{\ensuremath{{\cal H}^{-1}\left\{#1\right\}}}
\newcommand{\LT}[1]{\ensuremath{{\cal L}\left\{#1\right\}}}        % Laplace-Trafo
\newcommand{\ILT}[1]{\ensuremath{{\cal L}^{-1}\left\{#1\right\}}}
\newcommand{\DFT}[1]{\ensuremath{\mathrm{DFT}\left\{#1\right\}}}   % Diskrete Fourier-Trafo
\newcommand{\IDFT}[1]{\ensuremath{\mathrm{IDFT}\left\{#1\right\}}}
\newcommand{\FFT}[1]{\ensuremath{\mathrm{FFT}\left\{#1\right\}}}   % Fast Fourier-Trafo
\newcommand{\IFFT}[1]{\ensuremath{\mathrm{IFFT}\left\{#1\right\}}}
\newcommand{\NDFT}[2]{\ensuremath{\mathrm{DFT}_{#2}\left\{#1\right\}}}   % Diskrete Fourier-Trafo
\newcommand{\NIDFT}[2]{\ensuremath{\mathrm{IDFT}_{#2}\left\{#1\right\}}}
\newcommand{\NFFT}[2]{\ensuremath{\mathrm{FFT}_{#2}\left\{#1\right\}}}   % Fast Fourier-Trafo
\newcommand{\NIFFT}[2]{\ensuremath{\mathrm{IFFT}_{#2}\left\{#1\right\}}}\newcommand{\ZT}[1]{\ensuremath{{\cal Z}\left\{#1\right\}}}        % Z-Trafo
\newcommand{\IZT}[1]{\ensuremath{{\cal Z}^{-1}\left\{#1\right\}}}
\newcommand{\DTFT}[1]{\ensuremath{\mathrm{DTFT}\left\{#1\right\}}}   % Diskrete Time Fourier-Trafo
\newcommand{\IDTFT}[1]{\ensuremath{\mathrm{IDTFT}\left\{#1\right\}}}

% #####  Einheiten und Groessen  #####
\newcommand{\Hz}{\ensuremath{\mathrm{\:Hz}}}
\newcommand{\kHz}{\ensuremath{\mathrm{\:kHz}}}
\newcommand{\MHz}{\ensuremath{\mathrm{\:MHz}}}
\newcommand{\Mbits}{\ensuremath{\mathrm{\:Mbit/s}}}
\newcommand{\GHz}{\ensuremath{\mathrm{\:GHz}}}
\newcommand{\ms}{\ensuremath{\mathrm{\:ms}}}
\newcommand{\ns}{\ensuremath{\mathrm{\:ns}}}
\newcommand{\mus}{\ensuremath{\mathrm{\:\mu s}}}
\newcommand{\kmh}{\ensuremath{\mathrm{\:km/h}}}
\newcommand{\dB}{\ensuremath{\mathrm{\:dB}}}
\newcommand{\kbits}{\ensuremath{\mathrm{\:kbit/s}}}
\newcommand{\kBaud}{\ensuremath{\mathrm{\:kBaud}}}
\newcommand{\SNR}{\ensuremath{\frac{S}{N}}}
\newcommand{\EbN}{\ensuremath{\frac{E_b}{N_0}}}
\newcommand{\EbNh}{\ensuremath{\frac{E_b}{N_0/2}}}

% #####  Worte, die haeufig in Gleichungen gebraucht werden  #####
\newcommand{\Mit}{\quad\mathrm{mit}\;\,}          % kleingeschrieben existiert \mit schon!
\newcommand{\und}{\quad\mathrm{und}\;\,}
\newcommand{\da}{\quad\mathrm{da}\;\,}
\newcommand{\fuer}{\quad\mathrm{f"ur}\;\,}
\newcommand{\wobei}{\quad\mathrm{wobei}\;\,}
\newcommand{\mindex}[1]{\mbox{\scriptsize \sl #1}}

% #####  Fettschrift fuer Vektoren  #####
\newcommand{\vek}[1]{\ensuremath{\mathbf{#1}}}    % (fette) Vektoren oder Matrizen (mit Buchstabe als Argument)
\newcommand{\bs}[1]{\mbox{\boldmath$#1$}}         % (fette) schraege Vektoren oder Matrizen


% EOF
             % Must be included AFTER this file!            ##
% #####################################################################################

% Komfortables Auskommentieren ganzer LaTeX-Passagen
\newcommand{\comment}[1]{}

% Grafikeinbindung
\usepackage[dvips]{graphicx}
\newcommand{\bildwi}[2]{\includegraphics[width=#2\textwidth]{#1}}
\newcommand{\bildsc}[2]{\includegraphics[scale=#2]{#1}}
\newcommand{\bild}[2]{\includegraphics[width=#2\textwidth]{#1}}  % for compatibility
\newcommand{\centerbild}[1]{\centerline{#1}}                     % for compatibility

% Fettschrift im Mathemodus
\newcommand{\bfm}[1]{\mathbf{#1}}

% Definiere \bf, \sf etc. so um, dass sie sich bei gleichzeitiger Anwendung 
% sich nicht gegenseitig unwirksam machen.
\renewcommand{\rm}{\rmfamily}
\renewcommand{\sf}{\sffamily}
\renewcommand{\tt}{\ttfamily}
\renewcommand{\bf}{\bfseries}
\renewcommand{\it}{\itshape}
\renewcommand{\sl}{\slshape}
\renewcommand{\sc}{\scshape}
\newcommand{\md}{\mdseries}
\newcommand{\up}{\upshape}

% ###############  EOF  ###############

\usepackage{epsfig}                   % Zum Einbinden von Postscript-Bildern

% ##########  Mathematische Definitionen  ###########
% mathe.tex - Mathematische (Mengen)Zeichen, Symbole und Funktionen
%             fuer LaTeX 2.09 und LaTeX 2e
%
%             Original von Dieter Boss, 08/94
%             Letzte Aenderung: H.Schmidt, 26-April-2000 (NDFT ... NIFFT)
%

\newcommand{\ds}[1]{\displaystyle{#1}}    % Kurzschreibweise fuer \displaystyle
\newcommand{\ml}[1]{\hbox{\large $#1$}}   % math large : Angenehme Schrift-
                                          % groesse bei der Darstellung von
                                          % Bruechen z.B. \ml{1\over\sqrt{2}}

% #####  Haeufig benoetigte Funktionen  #####
\newcommand{\E}[1]{\ensuremath{\mathrm{E}\left\{#1\right\}}}
\newcommand{\real}[1]{\ensuremath{\mathrm{Re}\left\{#1\right\}}}
\newcommand{\imag}[1]{\ensuremath{\mathrm{Im}\left\{#1\right\}}}
\newcommand{\rect}[1]{\ensuremath{\mathrm{rect}\left(#1\right)}}
\newcommand{\tri}[1]{\ensuremath{\mathrm{tri}\left(#1\right)}}
\newcommand{\si}{\ensuremath{\mathrm{si}}}
\newcommand{\di}{\ensuremath{\mathrm{di}}}
\newcommand{\ld}{\ensuremath{\mathrm{ld}}}  
\newcommand{\erf}{\ensuremath{\mathrm{erf}}}
\newcommand{\erfc}{\ensuremath{\mathrm{erfc}}}
\newcommand{\eds}[1]{\ensuremath{\mbox{e }^{\ds{#1}}}}
\newcommand{\ex}[1]{\ensuremath{e^{#1}}}
\newcommand{\ejO}{\ex{j\Omega}}

% #####  Mathematische Sonderzeichen  #####
\newcommand{\defas}{\ensuremath{\stackrel{\Delta}{=}}}

% #####  Mengenzeichen  #####
\newcommand{\Reell}{\mathsf{I} \kern -0.15em \mathsf{R}} 
\newcommand{\Nat}{\mathsf{I}  \kern -0.15em \mathsf{N}}
\newcommand{\Feld}{\mathsf{I} \kern -0.15em \mathsf{F}}
\newcommand{\Zahl}{\mathsf{Z} \kern -0.45em \mathsf{Z}}

% ##  Korrespondenz-"Knochen": ##
\newcommand{\korrespond}{\ensuremath{\;\circ \hskip-1ex -\hskip-1.2ex -\hskip-1.2ex- \hskip-1ex \bullet\;}}
\newcommand{\ikorrespond}{\ensuremath{\;\bullet \hskip-1ex -\hskip-1.2ex -\hskip-1.2ex- \hskip-1ex \circ\;}}

% #####  Transformationen  #####
\newcommand{\FT}[1]{\ensuremath{{\cal F}\left\{#1\right\}}}        % (kont.) Fourier-Trafo
\newcommand{\IFT}[1]{\ensuremath{{\cal F}^{-1}\left\{#1\right\}}}
\newcommand{\HT}[1]{\ensuremath{{\cal H}\left\{#1\right\}}}        % Hilbert-Trafo
\newcommand{\IHT}[1]{\ensuremath{{\cal H}^{-1}\left\{#1\right\}}}
\newcommand{\LT}[1]{\ensuremath{{\cal L}\left\{#1\right\}}}        % Laplace-Trafo
\newcommand{\ILT}[1]{\ensuremath{{\cal L}^{-1}\left\{#1\right\}}}
\newcommand{\DFT}[1]{\ensuremath{\mathrm{DFT}\left\{#1\right\}}}   % Diskrete Fourier-Trafo
\newcommand{\IDFT}[1]{\ensuremath{\mathrm{IDFT}\left\{#1\right\}}}
\newcommand{\FFT}[1]{\ensuremath{\mathrm{FFT}\left\{#1\right\}}}   % Fast Fourier-Trafo
\newcommand{\IFFT}[1]{\ensuremath{\mathrm{IFFT}\left\{#1\right\}}}
\newcommand{\NDFT}[2]{\ensuremath{\mathrm{DFT}_{#2}\left\{#1\right\}}}   % Diskrete Fourier-Trafo
\newcommand{\NIDFT}[2]{\ensuremath{\mathrm{IDFT}_{#2}\left\{#1\right\}}}
\newcommand{\NFFT}[2]{\ensuremath{\mathrm{FFT}_{#2}\left\{#1\right\}}}   % Fast Fourier-Trafo
\newcommand{\NIFFT}[2]{\ensuremath{\mathrm{IFFT}_{#2}\left\{#1\right\}}}\newcommand{\ZT}[1]{\ensuremath{{\cal Z}\left\{#1\right\}}}        % Z-Trafo
\newcommand{\IZT}[1]{\ensuremath{{\cal Z}^{-1}\left\{#1\right\}}}
\newcommand{\DTFT}[1]{\ensuremath{\mathrm{DTFT}\left\{#1\right\}}}   % Diskrete Time Fourier-Trafo
\newcommand{\IDTFT}[1]{\ensuremath{\mathrm{IDTFT}\left\{#1\right\}}}

% #####  Einheiten und Groessen  #####
\newcommand{\Hz}{\ensuremath{\mathrm{\:Hz}}}
\newcommand{\kHz}{\ensuremath{\mathrm{\:kHz}}}
\newcommand{\MHz}{\ensuremath{\mathrm{\:MHz}}}
\newcommand{\Mbits}{\ensuremath{\mathrm{\:Mbit/s}}}
\newcommand{\GHz}{\ensuremath{\mathrm{\:GHz}}}
\newcommand{\ms}{\ensuremath{\mathrm{\:ms}}}
\newcommand{\ns}{\ensuremath{\mathrm{\:ns}}}
\newcommand{\mus}{\ensuremath{\mathrm{\:\mu s}}}
\newcommand{\kmh}{\ensuremath{\mathrm{\:km/h}}}
\newcommand{\dB}{\ensuremath{\mathrm{\:dB}}}
\newcommand{\kbits}{\ensuremath{\mathrm{\:kbit/s}}}
\newcommand{\kBaud}{\ensuremath{\mathrm{\:kBaud}}}
\newcommand{\SNR}{\ensuremath{\frac{S}{N}}}
\newcommand{\EbN}{\ensuremath{\frac{E_b}{N_0}}}
\newcommand{\EbNh}{\ensuremath{\frac{E_b}{N_0/2}}}

% #####  Worte, die haeufig in Gleichungen gebraucht werden  #####
\newcommand{\Mit}{\quad\mathrm{mit}\;\,}          % kleingeschrieben existiert \mit schon!
\newcommand{\und}{\quad\mathrm{und}\;\,}
\newcommand{\da}{\quad\mathrm{da}\;\,}
\newcommand{\fuer}{\quad\mathrm{f"ur}\;\,}
\newcommand{\wobei}{\quad\mathrm{wobei}\;\,}
\newcommand{\mindex}[1]{\mbox{\scriptsize \sl #1}}

% #####  Fettschrift fuer Vektoren  #####
\newcommand{\vek}[1]{\ensuremath{\mathbf{#1}}}    % (fette) Vektoren oder Matrizen (mit Buchstabe als Argument)
\newcommand{\bs}[1]{\mbox{\boldmath$#1$}}         % (fette) schraege Vektoren oder Matrizen


% EOF

%\usepackage{amstex}                   % Fuer fortgeschrittene Mathekonstuktionen

% ##########  Fancyheadings, Options etc  ###########
\usepackage{fancyheadings}             % Fuer einheitlichen Seitenkopf
\usepackage{options}                   % Box mit Schatten
%\usepackage{longtable}                % fuer mehrseitige Tabellen (z.B. im Anhang)

% ##########  Caption package   ######################
\usepackage[hang,small,bf]{caption}    % Angenehmere Darstellung von (Bild-)Unterschriften
%\renewcommand{\figurename}{Bild}       % Bezeichnung 'Bild' statt 'Abbildung'

% ##########  Seitenmasse und -position  ##########
% \textwidth16cm
% \textheight25cm
% \oddsidemargin 8mm
% \evensidemargin -1mm
% \topmargin-13mm 

% ##########  Zusatzabstand bei Absaetzen  ##########
\parskip1.5ex plus 0.3ex

% ##########  Keine Einrueckung bei Absaetzen  ##########
\parindent0cm

% ##########  Zeilenabstand  ##########
\newcommand{\zeilenabstand}{1.0}
\renewcommand{\baselinestretch}{\zeilenabstand}

% ##########  Verbatim package  ######################
%\usepackage{verbatimfiles}             % Zum Einbinden von ASCII-Files, z.B.
                                       % Matlab-m-Files im Anhang
%\newcommand{\getASCII}[1]{\renewcommand{\baselinestretch}{0.8}\begin{footnotesize}%
%\verbatimfile{#1}\end{footnotesize}\renewcommand{\baselinestretch}{\zeilenabstand}}

% ##########  Stilparameter fuer gleitende Objekte  ##########
\renewcommand{\topfraction}{1.0}
\renewcommand{\bottomfraction}{1.0}
\renewcommand{\textfraction}{0.0}
\renewcommand{\floatpagefraction}{1.0}

% ##########  Trennungen weitestgehend unterdruecken  ##########
\hyphenpenalty=5000
\doublehyphendemerits=5000
\brokenpenalty=5000
\finalhyphendemerits=5000

% ################  Neue Headings f"ur Bookstyle  #################
% Beispiele s. /usr/local/lib/tex/inputs/fancyheadings.sty, angepasst von Dieter Boss
\pagestyle{fancyplain}    % must appear after changes to \textwidth
\plainheadrulewidth 0.4pt % damit auf erster Seite auch Unterstrich erscheint (Boe)
%\headsep1cm
\renewcommand{\chaptermark}[1]%
   {\markboth{\uppercase{\thechapter.\ #1}}{}}
\renewcommand{\sectionmark}[1]%
   {\markright{\uppercase{\thesection.\ #1}}}
%\setlength{\headrulewidth}{0pt}
%\setlength{\footrulewidth}{0pt}
\lhead[\thepage]{\scriptsize\rightmark}
\rhead[\scriptsize\leftmark]{\thepage}
\cfoot{}
\newcommand{\clearemptydoublepage}{\newpage{\pagestyle{empty}\cleardoublepage}}

% ################  Modifizierter Fussnotenstil  ##################
\newlength{\footnotelaenge}
\makeatletter
\renewcommand{\@makefntext}[1]{%
\setlength{\footnotelaenge}{\columnwidth}
\addtolength{\footnotelaenge}{-0.7em}
    \noindent
    \hbox to 0.7em {\hss\@makefnmark}\parbox[t]{\footnotelaenge}{#1}}
\makeatother


% #####  Hier sind die Daten der Arbeit einzutragen.      #####
% #####  Sie werden automatisch ins Dokument uebernommen. #####
\newcommand{\Arbeitsart}{Master Thesis}     % oder: Studienarbeit, Projektarbeit
\newcommand{\Ausgabetermin}{02.11.2004}
\newcommand{\Abgabetermin}{15.03.2005}
\newcommand{\Berichtsdatum}{20.03.2005}
\newcommand{\Betreuer}{Dr.-Ing. Volker Kuehn}
\newcommand{\Autor}{Raavi M. Mohindar Rao}
\newcommand{\Berichtstitel}{Analysis tools for Multi User Detection}


% #####  Hier die selbst definierten Kommandos eintragen:       #####
% #####  Bsp: - Kommando ohne Argumente:                        #####
% #####         \newcommand{\neuesKommando}{blabla}             #####
% #####       - Kommando mit 2 Argumenten:                      #####
% #####         \newcommand{\neuesKommando}[2]{{\large #1:} #2} #####
% ...


% #####  Es wird nur kompiliert, was hier angegeben wird!   #####
% #####  Einzelne Eintraege also loeschen, falls sie zeit-  #####
% #####  weilig nicht kompiliert werden sollen (damit es    #####
% #####  schneller geht!):                                  #####
\includeonly{intro}%,ANTanh}  % alles kompilieren
%\includeonly{ANTkap1,ANTkap2,ANTkap3}   % nur teilweise kompilieren



\begin{document}              % Hier geht's erst richtig los ...
  
\include{intro}          % Titelseiten
%   \clearemptydoublepage
%   \thispagestyle{fancyplain}
%   \pagenumbering{Roman}
%   \tableofcontents            % Inhaltsverzeichnis
%   \thispagestyle{fancy}
%   \clearemptydoublepage
%   \pagenumbering{arabic}
%   \include{intro}           % Kapitel 1 (Einleitung)
  %\chapter{Prinzipielles} \label{CPTerst}
%
\section{Deutsche Umlaute und G"ansef"u"schen}
%
Sofern man keine Umlaute auf der Tastatur hat, m"ussen Umlaute in \LaTeX\ so 
eingegeben werden:

\qquad "a durch \verb+"a+,\quad "o durch \verb+"o+,\quad "u durch \verb+"u+,\quad "s durch \verb+"s+.

Die gro"sen Umlaute funktionieren entsprechend. Damit dies alles so klappt,
mu"s "ubrigens der {\tt german} style eingebunden worden sein (was hier der Fall ist).

Verwendet man die Umlaute auf der deutschen Tastatur(in Latin-1 Codierung), so
mu� �brigens das Paket \verb+inputenc+ mit der Option \verb+latin1+
eingebunden werden.\\
\verb+\usepackage[latin1]{inputenc}+


Was die G"ansef"u"schen angeht, so stellt sich zun"achst die Frage, wie man sie gerne haben will.
Zieht man ``diese'' Form vor (d.h.\ vorne die 66er-, hinten die 99er-Gestalt),
so mu"s man das in \LaTeX\ mit dem doppelten {\em accent grave} bzw.\ Apostroph formulieren: \verb+``Zitat''+.

Die "`richtigen"' deutschen G"ansehaxen (vorne unten die 99er-, hinten oben die 66er-Gestalt)
sehen allerdings "`so"' aus und sind noch komplizierter einzutippen: \verb+"`Zitat"'+ .

A vous de choisir! Nur von den doppelten Anf"uhrungszeichen auf der Tastatur ist abzuraten, 
da sie mit der oben erw"ahnten Umlaut-Konstruktion kollidieren und au"serdem vorne und hinten die 
gleiche Gestalt produzieren.


\section{Schriftstile und -gr"o"sen}
%
{\bf Fettdruck} erreicht man mittels des \verb+\bf+-Kommandos (engl.\ {\em boldface}):
\verb+{\bf Fettdruck}+. Analog dazu kann man Text auch durch
{\em Kursivschrift} hervorheben: \verb+{\em Kursivschrift}+.
Mit \verb+\tt+ wie {\em teletype} erreicht man die 
{\tt Schreibmaschinenschrift} und mit \verb+\sc+ eine Schriftart, die
sich {\sc Small Capitals} nennt.
Diese (und weitere) Schriftstile stehen allerdings nur im ``Textmodus''
zur Verf"ugung, nicht aber im ``Mathemodus'' (s.~Abschnitt~\ref{CPTmath}).

L"angere Textpassagen kann man z.B.\ durch
%
\begin{verbatim}
\begin{bf}
Lange Textpassage im Fettdruck ...
\end{bf}
\end{verbatim}
%
in den gew"unschten Stil bringen.

Die Schrift{\em gr"o"sen} lauten --~in kleinerwerdender Reihenfolge: 
{\tt Huge, huge, LARGE, Large, large, normalsize, small, footnotesize, tiny}.
Auch sie sind in der Form \verb+{\large xxx}+ sowie mit \verb+\begin{large}+
und \verb+\end{large}+ anzusprechen.

\section{Leerzeichen (Spaces), Abst"ande zwischen Worten}
%
\LaTeX\ behandelt 10 aufeinanderfolgende Leerzeichen im ``\verb+*.tex+''-File 
genauso, als ob man nur ein einziges getippt h"atte. Will man l"angere horizonale 
Abst"ande erzwingen, so stehen dazu die Befehle \verb+\hspace{...}+, \verb+\quad+,
\verb+\qquad+ etc.\ zur Verf"ugung.

Innerhalb eines Absatzes generiert \LaTeX\ einen sch"onen rechten Rand (Blocksatz),
indem die Abst"ande zwischen den Worten k"unstlich gestreckt oder gestaucht 
werden.
Dabei streckt \LaTeX\ die Zwischenr"aume nach Satzzeichen mehr als die anderen, so
da"s sich z.B.~nach einem Punkt ein l"angerer Zwischenraum ergibt. Allerdings kann
\LaTeX\ nicht wissen, welcher Punkt ein Satzende darstellt und welcher nicht.
Deshalb mu"s man \LaTeX\ mitteilen, welche Zwischenr"aume nicht gestreckt werden d"urfen.
Dies kann dadurch geschehen, da"s man ein normales Leerzeichen ersetzt durch
``\verb+\ +'' (Backslash, Space) oder ``\verb+~+'' (Tilde). Beide Befehle bewirken, da"s 
dieses Leerzeichen wie ein Abstand zwischen Worten ohne Satzzeichen behandelt wird.
Die Tilde verhindert au"serdem, da"s ein Zeilenumbruch an diese Stelle gelegt wird
und ist deshalb in Konstruktionen wie \verb+siehe Kap.~5+ oder \verb+Dr.~Mustermann+ anzuraten.

N.B.: Am besten, man merkt sich {\em ein f"ur allemal}, bestimmte Konstruktionen so
einzutippen:
%
  \begin{center}
  \begin{tabular}{|c|c|}
    \hline
    Gew"unschter Output & Optimale \LaTeX-Notation \\
    \hline
    \hline
    d.h.\ blabla & \verb+d.h.\ blabla+ \\
    z.B.\ blabla & \verb+z.B.\ blabla+ \\
    bzw.\ blabla & \verb+bzw.\ blabla+ \\
    \hline
  \end{tabular}
  \end{center}
%
Auch im Zusammenhang mit Referenzen (s.~Kap.~\ref{SECverw}) sollte man prinzipiell 
folgende Notation w"ahlen:
\verb+aus Kap.~\ref{label1} folgt...+\ oder:\ 
\verb+wie in Gl.~(\ref{label2}) gezeigt...+


\section{Strukturierung eines Dokumentes} \label{SECunter}
%
Die "Uberschrift zu Kapitel~\ref{CPTerst} wurde durch den Befehl
\verb+\chapter{Prinzipielles}+ erzeugt. Dieser Befehl erzeugt auch
die automatische fortlaufende Kapitel-Numerierung. Abschnitte und Unterabschnitte
kommen durch die Verwendung der Befehle \verb+\section{...}+ (s.\ z.B.\ Abschnitt~\ref{SECunter}), 
\verb+\subsection{...}+ und \verb+\subsubsection{...}+ zustande.
Die Eintr"age aller dieser Strukturkommandos werden automatisch ins
Inhaltsverzeichnis "ubernommen.

\section{Aufz"ahlungen} \label{SECaufz}
%
Aufz"ahlung in der folgenden Form,
%
\begin{itemize}
\item d.h.\ mit einer Einr"uckung und kleinen P"unktchen am Anfang ...
\item ... jedes Eintrages
\end{itemize}
%
k"onnen mit der \verb+itemize+-Umgebung erreicht werden:
%
\begin{verbatim}
\begin{itemize}
\item d.h.\ mit einer Einr"uckung und kleinen P"unktchen am Anfang ...
\item ... jedes Eintrages
\end{itemize}
\end{verbatim}
%
Auch innerhalb eines \verb+\item+ darf wieder eine \verb+itemize+-Umgebung
stehen, wobei dann kleine Striche statt P"unktchen erscheinen.

Will man die einzelnen Eintr"age mit laufenden Nummern statt mit P"unktchen
versehen, so ist die \verb+enumerate+-Umgebung zu verwenden.

\section{Tabellen} \label{SECtab}
%
Tabelle~\ref{TABsuper} zeigt ein paar wesentliche Elemente, mit
denen man einfache Tabellen produzieren kann. Die {\tt tabular}-Umgebung
erh"alt dabei das Argument \verb+|l|rc|+, was bedeutet, da"s die Tabelle
drei Spalten haben soll, wobei die Inhalte in der ersten Spalte linksb"undig,
in der zweiten rechtsb"undig und in der dritten Spalte zentriert erscheinen 
sollen. Die senkrechten Striche deuten an, welche Spalten durch Striche
voneinander getrennt werden sollen.
%
\begin{verbatim}
\begin{table}[htb]
  \begin{center}
  \begin{tabular}{|l|rc|}
    \hline
    aa   & bb    & cc \\
    \hline
    \hline
    a    & b     & c  \\
    left & right & center \\
    aa   & !!    & ?? \\
    \hline
  \end{tabular}
  \end{center}
  \caption{Mustertabelle} \label{TABsuper}
\end{table}
\end{verbatim}
%
\begin{table}[htb]
  \begin{center}
  \begin{tabular}{|l|rc|}  % 1. Spalte linksbuendig, 2. rechtsbuendig, 3. Spalte zentriert
    \hline
    aa   & bb    & cc \\
    \hline
    \hline
    a    & b     & c  \\
    left & right & center \\
    aa   & !!    & ?? \\
    \hline
  \end{tabular}
  \end{center}
  \caption{Mustertabelle} \label{TABsuper}
\end{table}

In den Tabelleneintr"agen selbst trennt das Kaufmanns-Und (``\verb+&+'')
die Spalten voneinander. Jede Zeile mit Eintr"agen
mu"s mit einem Zeilenumbruch (``\verb+\\+'') abgeschlossen werden.

{\bf Tabellen-Plazierung:} Da die {\tt table}-Umgebung ein {\em floating object} 
definiert\footnote{Genau so wie die {\tt figure}-Umgebung, s.\ Abschnitt~\ref{SUBSECbilder}.},
das frei beweglich im Text ``schwimmt''\footnote{L"a"st man im \LaTeX-File direkt vor und
nach der {\tt table}-Umgebung keine Leerzeile, so ``scrollt'' der Text um die Tabelle herum.} 
und je nach aktueller Lage der Seitenumbr"uche an einer anderen Stelle landen kann, kann man der 
{\tt table}-Umgebung einen Plazierungswunsch mitgeben: \verb+[htb]+ bedeutet, da"s die Tabelle
vorzugsweise da erscheinen soll, wo sie im \LaTeX-File auftritt ({\tt `h'} wie {\em here}),
andernfalls oben {\em (top)} auf der Seite, notfalls unten {\em (bottom)}.
Allerdings haben schon viele Leute dar"uber geflucht, da"s \LaTeX\ sich ziemlich
oft "uber die W"unsche seiner User hinwegsetzt, insbesondere bei gro"sen Gleitobjekten. 

An dieser Stelle sei angemerkt, da"s es sich als sehr vorteilhaft erweisen
wird, auch das \LaTeX-File einigerma"sen sauber zu formatieren.
Obwohl es \LaTeX\ ``egal'' ist, wie man den Text in die Datei hackt, 
sollten die folgenden Beispiele f"ur sich sprechen. Sowohl der folgende
Ausschnitt
%
\begin{verbatim}
 \begin{table}[htb]\begin{center}\begin{tabular}{|l|r|}\hline links
&rechts     \\\hline        \hline 12    &  13  \\$  \pi  $   &   $
\pi/2$           \\  \hline        \end{tabular}       \end{center}
\caption{Mustertabelle}\end{table}
\end{verbatim}
%
als auch die besser formatierte Version
%
\begin{verbatim}
\begin{table}[htb]
  \begin{center}
  \begin{tabular}{|l|r|}
    \hline
    links & rechts  \\
    \hline
    \hline
    12    & 13      \\
    $\pi$ & $\pi/2$ \\
    \hline
  \end{tabular}
  \end{center}
\caption{Mustertabelle}
\end{table}
\end{verbatim}
%
liefern die folgende Tabelle:
%
\begin{table}[htb]
  \begin{center}
  \begin{tabular}{|l|r|} \hline
    links & rechts  \\   \hline \hline
    12    & 13      \\
    $\pi$ & $\pi/2$ \\   \hline
  \end{tabular}
  \end{center}
\caption{Mustertabelle}
\end{table}

Nur, wehe in der ersten Version hat man sich mit den Tabulatorzeichen (``\verb+&+'')
verz"ahlt \dots


           % Kapitel 2
  %\chapter{Mathematisches} \label{CPTmath}
%
In kaum einem im Arbeitsbereich Nachrichtentechnik verfa"sten Dokument 
wird man wohl um die Darstellung von mathematischen Zusammenh"angen 
herumkommen. Deshalb geht es in diesem Kapitel um die Darstellung von
Formeln, Gleichungen, Matrizen etc. Auf diesem Gebiet zeigen sich viele
St"arken von \LaTeX. So hat es einen speziellen {\em Mathemodus}, in den
man durch ein Dollarzeichen (``\verb+$+'') oder eine der Gleichungsumgebungen
kommt. Im Mathemodus sind dann unz"ahlige Kommandos verf"ugbar, die man
f"ur mathematische Darstellungen unbedingt ben"otigt. 

\section{Mathemodus im Flie"stext} \label{SECfliess}

Variablen, Formelzeichen und kurze Formeln mitten im Satz werden einfach in 
{\tt \$}'s eingeschlossen. So erzeugt zum Beispiel \verb+$12x - y = \sqrt{\pi}$+
die Gleichung $12x - y = \sqrt{\pi}$. 

Allerdings sollte man nicht versuchen,
beliebig gro"se Konstrukte in den Flie"stext einzubauen: insbesondere Br"uche
sollten besser mittels \verb+$1/2$+ als druch \verb+$\frac{1}{2}$+ dargestellt werden, 
da ersteres den lesbaren Output $1/2$ liefert, wogegen die \verb+\frac+-Konstruktion 
mickrig wird: $\frac{1}{2}$.

Man beachte, da"s auch einzelne Buchstaben
(sprich: Variablennamen) im Textmodus anders aussehen als im Mathemodus. Bsp.:
N im Textmodus und $N$ im Mathemodus (\verb+$N$+).

\section{Einfache Gleichungen} \label{SECeinfg}

Komplexere Formeln m"ussen h"aufig {\em abgesetzt} vom Rest des Textes erscheinen, 
wobei man eine Variante {\em ohne} und eine {\em mit} fortlaufender 
Gleichungsnumerierung unterscheidet. Die Zeilen 
%
\begin{verbatim}
\begin{displaymath}
  \sum_{i=0}^{n} |x_i| \neq \int_{-\alpha}^{\infty} \eta \, d\eta
\end{displaymath}
\end{verbatim}
%
ergeben die folgende Gleichung ohne Nummer:
%
\begin{displaymath}
  \sum_{i=0}^{n} |x_i| \neq \int_{-\alpha}^{\infty} \eta \, d\eta
\end{displaymath}
%
Tippm"ude Leute k"onnen statt \verb+\begin{displaymath}+ und \verb+\end{displaymath}+
auch einfach \verb+\[+ bzw.\ \verb+\]+ oder auch \verb+$$+ verwenden.

Die folgende Gleichung mit fortlaufender Numerierung
%
\begin{equation} \label{EQmit}
  \sum_{i=0}^{n-1} x_{ij}^2 \approx 
    2\Delta \cdot \left(\int_{-\pi}^{\pi} \frac{\Omega}{2} \, d\Omega \right)
\end{equation}
%
wurde durch 
%
\begin{verbatim}
\begin{equation} \label{EQmit}
  \sum_{i=0}^{n-1} x_{ij}^2 \approx 
    2\Delta \cdot \left(\int_{-\pi}^{\pi} \frac{\Omega}{2} \, d\Omega \right)
\end{equation}
\end{verbatim}
%
erzielt. Dabei kann mit \verb+\ref{EQmit}+ auf die Gleichungsnummer zugegriffen
werden (s.~Abschnitt~\ref{SECverw}).

\section{Mehrzeilige Gleichungen} \label{SECmehrg}

Mehrzeilige Formeln werden dann verwendet, wenn die Formel ganz einfach zu lang ist
oder mehrere separate Gleichungen an ihren Gleichheitszeichen horizontal ausgerichtet
werden sollen. Dazu verwendet man die {\tt eqnarray}-Umgebung. 
Auch hier kann entweder die automatische Gleichungsnumerierung benutzt 
werden oder auch nicht.
Die Zeilen
%
\begin{verbatim}
\begin{eqnarray*}
a &=& A \times B\\
  &=& \prod_{j=1}^{n} C_j \equiv \prod_{j=1}^{n} A_j  
           \quad \mbox{f"ur}\quad C_j = A_j\quad ,  
\end{eqnarray*}
\end{verbatim}
%
erzeugen die folgende nichtnumerierte Gleichung
%
\begin{eqnarray*}
a &=& A \times B\\
  &=& \prod_{j=1}^{n} C_j \equiv \prod_{j=1}^{n} A_j  
           \quad \mbox{f"ur}\quad C_j = A_j\quad ,  
\end{eqnarray*}
%
w"ahrend mit Numerierung sogar einzelne Zeilen von derselben einzeln 
ausgenommen werden k"onnen, wie die folgende Gleichung zeigt:
%
\begin{verbatim}
\begin{eqnarray} 
a &=&       b + \nu + \mu      \label{EQbnumu}\\
  &=&       3 + 4   + 5        \nonumber\\
A &\neq&    \beta + \nu + \mu  \\
  &\approx& 3.14  + 4.1 + 5.2  \nonumber\\
  &=&       1244\cdot 10^{-2}  \nonumber   \quad .
\end{eqnarray}
\end{verbatim}
%
\begin{eqnarray} 
a &=&       b + \nu + \mu      \label{EQbnumu}\\
  &=&       3 + 4   + 5        \nonumber\\
A &\neq&    \beta + \nu + \mu  \\
  &\approx& 3.14  + 4.1 + 5.2  \nonumber\\
  &=&       1244\cdot 10^{-2}  \nonumber   \quad .
\end{eqnarray}
%
Alle bisherigen Gleichungsformel haben ein gemeinsames Kennzeichen.
Sie setzen \LaTeX\ in den {\em Mathemodus}. Viele der oben gezeigten Befehle
(z.B.\ alle griechischen Buchstaben, die Hoch- und Tiefstellung mit ``\verb+^+'' 
bzw.\ ``\verb+_+'', das Integral-, Summen- und Produktzeichen etc).
sind nur im Mathemodus zul"assig und f"uhren zu Fehlern, wenn man versucht,
sie im Textmodus anzuwenden.

\section{Weitere Befehle im Mathemodus} \label{SECbeisp}
%
\begin{itemize}
\item Vergleichssymbole: $3<2$ (\verb+$3<2$+);\quad  $x \leq y$ (\verb+$x \leq y$+);\quad $a \gg b$ (\verb+$a \gg b$+).
\item Funktionen sind \underline{keine} Variablen, werden also 
      nicht etwa kursiv, sondern als normaler Text dargestellt.
      Dazu stehen die Befehle \verb+\sin, \log, \lim, \max, \det+ im Mathemodus zur Verf"ugung. 
      $$
      \sin{x}\neq sin x, \ \log(10), \ \lim_{t\rightarrow\infty} y(t)=0, \
            \max_{j\neq i} V_j, \ \det{A}, \quad \ldots
      $$
\item Br"uche: \verb+$\frac{\pi^2}{2}$+ erzeugt im Flie"stext den kleinen Output $\frac{\pi^2}{2}$,
      in der Gleichungsumgebung dagegen:
$$\frac{\pi^2}{2}$$
\item Die folgende $(m \times n)$ Matrix ${\bfm A}$ 
%
$$
{\bfm A} := 
\left(\begin{array}{cccc}   % Vier Spalten mit zentrierten ("c") Elementen
  a_{11} & a_{12} & \cdots & a_{1n}\\
  a_{21} & a_{22} & \cdots & a_{2n}\\
  \vdots & \vdots & \ddots & \vdots\\
  a_{m1} & a_{m2} & \cdots & a_{mn}
\end{array}\right)
$$
%
wurde mit Hilfe der {\tt array}-Umgebung dargestellt.
%
\begin{verbatim}
$$
{\bfm A} := 
\left(\begin{array}{cccc}   % Vier Spalten mit zentrierten ("c") Elementen
  a_{11} & a_{12} & \cdots & a_{1n}\\
  a_{21} & a_{22} & \cdots & a_{2n}\\
  \vdots & \vdots & \ddots & \vdots\\
  a_{m1} & a_{m2} & \cdots & a_{mn}
\end{array}\right)
$$
\end{verbatim}
\item Faltung: \verb+$y(t)=h(t)\ast x(t)$+ erzeugt das Resultat $y(t)=h(t)\ast x(t)$.
\item Punkte: $\cdot$  (\verb+$\cdot$+); \quad  $\ldots$ (\verb+$\ldots$+);\quad   
              $\cdots$ (\verb+$\cdots$+);\quad  $\vdots$ (\verb+$\vdots$+);\quad
              $\ddots$ (\verb+$\ddots$+)
\item Folgepfeile: so $U \rightarrow X$ (mit \verb+\rightarrow+) oder so: $U \Rightarrow X$ (mit \verb+\Rightarrow+)
%
\end{itemize}


% EOF
           % Kapitel 3
  %\chapter{ANT-Spezifisches}\label{CPTbeisp} 
%
In diesem Kapitel geht es um Features, die nicht zum Umfang der
\LaTeX-Distribution geh"oren, sondern von flei"sigen
ANT-Mitarbeitern\footnote{\bildsc{ANTonly}{0.134} und
\bildsc{ANTonly}{0.132}} ersonnen wurden, um sich und dem Rest der
ANT-Welt das Textverarbeitungs-Leben leichter zu machen. So definieren
die Dateien {\tt latex2e.tex} und {\tt mathe.tex} Befehle, die mit
hoher Wahrscheinlichkeit auch f"ur Dich n"utzlich sein werden. Sie
stehen allen Benutzern des Rechnernetzes im Arbeitsbereich automatisch
zur Verf"ugung, vorausgesetzt sie werden mit \verb+% #####################################################################################
% ##       latex2e.tex - LaTeX 2e-specific Definitions  by  Dieter Boss, 01/96       ##
% ##  -----------------------------------------------------------------------------  ##
% ##  The following commands are defined appropriately:                              ##
% ##  - \comment{}      : Multiple line comments                                     ##
% ##  - \bild{file}{1.0}: Include graphics and scale in multiples of "\textwidth"    ##
% ##  - \centerbild{...}: Include centered graphics                                  ##
% ##  - \bfm            : Bold face in math mode                                     ##
% ##  - \bf, \sf ...    : can be applied simultaneously (and work together!)         ##
% ##  -----------------------------------------------------------------------------  ##
% ##  Example of a LaTeX2e document header:                                          ##
% ##    \documentclass[twocolumn]{article}                                           ##
% ##    \usepackage{rotate,colordvi}  % Swap "colordvi" for "blackdvi" when printing ##
% ##    \usepackage[english]{babel}   %     a color LaTeX document on a B/W printer. ##
% ##    \usepackage{options,my9pt}    % ... and "showkeys" to view names of labels.  ##
% ##    % #####################################################################################
% ##       latex2e.tex - LaTeX 2e-specific Definitions  by  Dieter Boss, 01/96       ##
% ##  -----------------------------------------------------------------------------  ##
% ##  The following commands are defined appropriately:                              ##
% ##  - \comment{}      : Multiple line comments                                     ##
% ##  - \bild{file}{1.0}: Include graphics and scale in multiples of "\textwidth"    ##
% ##  - \centerbild{...}: Include centered graphics                                  ##
% ##  - \bfm            : Bold face in math mode                                     ##
% ##  - \bf, \sf ...    : can be applied simultaneously (and work together!)         ##
% ##  -----------------------------------------------------------------------------  ##
% ##  Example of a LaTeX2e document header:                                          ##
% ##    \documentclass[twocolumn]{article}                                           ##
% ##    \usepackage{rotate,colordvi}  % Swap "colordvi" for "blackdvi" when printing ##
% ##    \usepackage[english]{babel}   %     a color LaTeX document on a B/W printer. ##
% ##    \usepackage{options,my9pt}    % ... and "showkeys" to view names of labels.  ##
% ##    \input{latex2e.tex}           % Include this file.                           ##
% ##    \input{mathe.tex}             % Must be included AFTER this file!            ##
% #####################################################################################

% Komfortables Auskommentieren ganzer LaTeX-Passagen
\newcommand{\comment}[1]{}

% Grafikeinbindung
\usepackage[dvips]{graphicx}
\newcommand{\bildwi}[2]{\includegraphics[width=#2\textwidth]{#1}}
\newcommand{\bildsc}[2]{\includegraphics[scale=#2]{#1}}
\newcommand{\bild}[2]{\includegraphics[width=#2\textwidth]{#1}}  % for compatibility
\newcommand{\centerbild}[1]{\centerline{#1}}                     % for compatibility

% Fettschrift im Mathemodus
\newcommand{\bfm}[1]{\mathbf{#1}}

% Definiere \bf, \sf etc. so um, dass sie sich bei gleichzeitiger Anwendung 
% sich nicht gegenseitig unwirksam machen.
\renewcommand{\rm}{\rmfamily}
\renewcommand{\sf}{\sffamily}
\renewcommand{\tt}{\ttfamily}
\renewcommand{\bf}{\bfseries}
\renewcommand{\it}{\itshape}
\renewcommand{\sl}{\slshape}
\renewcommand{\sc}{\scshape}
\newcommand{\md}{\mdseries}
\newcommand{\up}{\upshape}

% ###############  EOF  ###############
           % Include this file.                           ##
% ##    % mathe.tex - Mathematische (Mengen)Zeichen, Symbole und Funktionen
%             fuer LaTeX 2.09 und LaTeX 2e
%
%             Original von Dieter Boss, 08/94
%             Letzte Aenderung: H.Schmidt, 26-April-2000 (NDFT ... NIFFT)
%

\newcommand{\ds}[1]{\displaystyle{#1}}    % Kurzschreibweise fuer \displaystyle
\newcommand{\ml}[1]{\hbox{\large $#1$}}   % math large : Angenehme Schrift-
                                          % groesse bei der Darstellung von
                                          % Bruechen z.B. \ml{1\over\sqrt{2}}

% #####  Haeufig benoetigte Funktionen  #####
\newcommand{\E}[1]{\ensuremath{\mathrm{E}\left\{#1\right\}}}
\newcommand{\real}[1]{\ensuremath{\mathrm{Re}\left\{#1\right\}}}
\newcommand{\imag}[1]{\ensuremath{\mathrm{Im}\left\{#1\right\}}}
\newcommand{\rect}[1]{\ensuremath{\mathrm{rect}\left(#1\right)}}
\newcommand{\tri}[1]{\ensuremath{\mathrm{tri}\left(#1\right)}}
\newcommand{\si}{\ensuremath{\mathrm{si}}}
\newcommand{\di}{\ensuremath{\mathrm{di}}}
\newcommand{\ld}{\ensuremath{\mathrm{ld}}}  
\newcommand{\erf}{\ensuremath{\mathrm{erf}}}
\newcommand{\erfc}{\ensuremath{\mathrm{erfc}}}
\newcommand{\eds}[1]{\ensuremath{\mbox{e }^{\ds{#1}}}}
\newcommand{\ex}[1]{\ensuremath{e^{#1}}}
\newcommand{\ejO}{\ex{j\Omega}}

% #####  Mathematische Sonderzeichen  #####
\newcommand{\defas}{\ensuremath{\stackrel{\Delta}{=}}}

% #####  Mengenzeichen  #####
\newcommand{\Reell}{\mathsf{I} \kern -0.15em \mathsf{R}} 
\newcommand{\Nat}{\mathsf{I}  \kern -0.15em \mathsf{N}}
\newcommand{\Feld}{\mathsf{I} \kern -0.15em \mathsf{F}}
\newcommand{\Zahl}{\mathsf{Z} \kern -0.45em \mathsf{Z}}

% ##  Korrespondenz-"Knochen": ##
\newcommand{\korrespond}{\ensuremath{\;\circ \hskip-1ex -\hskip-1.2ex -\hskip-1.2ex- \hskip-1ex \bullet\;}}
\newcommand{\ikorrespond}{\ensuremath{\;\bullet \hskip-1ex -\hskip-1.2ex -\hskip-1.2ex- \hskip-1ex \circ\;}}

% #####  Transformationen  #####
\newcommand{\FT}[1]{\ensuremath{{\cal F}\left\{#1\right\}}}        % (kont.) Fourier-Trafo
\newcommand{\IFT}[1]{\ensuremath{{\cal F}^{-1}\left\{#1\right\}}}
\newcommand{\HT}[1]{\ensuremath{{\cal H}\left\{#1\right\}}}        % Hilbert-Trafo
\newcommand{\IHT}[1]{\ensuremath{{\cal H}^{-1}\left\{#1\right\}}}
\newcommand{\LT}[1]{\ensuremath{{\cal L}\left\{#1\right\}}}        % Laplace-Trafo
\newcommand{\ILT}[1]{\ensuremath{{\cal L}^{-1}\left\{#1\right\}}}
\newcommand{\DFT}[1]{\ensuremath{\mathrm{DFT}\left\{#1\right\}}}   % Diskrete Fourier-Trafo
\newcommand{\IDFT}[1]{\ensuremath{\mathrm{IDFT}\left\{#1\right\}}}
\newcommand{\FFT}[1]{\ensuremath{\mathrm{FFT}\left\{#1\right\}}}   % Fast Fourier-Trafo
\newcommand{\IFFT}[1]{\ensuremath{\mathrm{IFFT}\left\{#1\right\}}}
\newcommand{\NDFT}[2]{\ensuremath{\mathrm{DFT}_{#2}\left\{#1\right\}}}   % Diskrete Fourier-Trafo
\newcommand{\NIDFT}[2]{\ensuremath{\mathrm{IDFT}_{#2}\left\{#1\right\}}}
\newcommand{\NFFT}[2]{\ensuremath{\mathrm{FFT}_{#2}\left\{#1\right\}}}   % Fast Fourier-Trafo
\newcommand{\NIFFT}[2]{\ensuremath{\mathrm{IFFT}_{#2}\left\{#1\right\}}}\newcommand{\ZT}[1]{\ensuremath{{\cal Z}\left\{#1\right\}}}        % Z-Trafo
\newcommand{\IZT}[1]{\ensuremath{{\cal Z}^{-1}\left\{#1\right\}}}
\newcommand{\DTFT}[1]{\ensuremath{\mathrm{DTFT}\left\{#1\right\}}}   % Diskrete Time Fourier-Trafo
\newcommand{\IDTFT}[1]{\ensuremath{\mathrm{IDTFT}\left\{#1\right\}}}

% #####  Einheiten und Groessen  #####
\newcommand{\Hz}{\ensuremath{\mathrm{\:Hz}}}
\newcommand{\kHz}{\ensuremath{\mathrm{\:kHz}}}
\newcommand{\MHz}{\ensuremath{\mathrm{\:MHz}}}
\newcommand{\Mbits}{\ensuremath{\mathrm{\:Mbit/s}}}
\newcommand{\GHz}{\ensuremath{\mathrm{\:GHz}}}
\newcommand{\ms}{\ensuremath{\mathrm{\:ms}}}
\newcommand{\ns}{\ensuremath{\mathrm{\:ns}}}
\newcommand{\mus}{\ensuremath{\mathrm{\:\mu s}}}
\newcommand{\kmh}{\ensuremath{\mathrm{\:km/h}}}
\newcommand{\dB}{\ensuremath{\mathrm{\:dB}}}
\newcommand{\kbits}{\ensuremath{\mathrm{\:kbit/s}}}
\newcommand{\kBaud}{\ensuremath{\mathrm{\:kBaud}}}
\newcommand{\SNR}{\ensuremath{\frac{S}{N}}}
\newcommand{\EbN}{\ensuremath{\frac{E_b}{N_0}}}
\newcommand{\EbNh}{\ensuremath{\frac{E_b}{N_0/2}}}

% #####  Worte, die haeufig in Gleichungen gebraucht werden  #####
\newcommand{\Mit}{\quad\mathrm{mit}\;\,}          % kleingeschrieben existiert \mit schon!
\newcommand{\und}{\quad\mathrm{und}\;\,}
\newcommand{\da}{\quad\mathrm{da}\;\,}
\newcommand{\fuer}{\quad\mathrm{f"ur}\;\,}
\newcommand{\wobei}{\quad\mathrm{wobei}\;\,}
\newcommand{\mindex}[1]{\mbox{\scriptsize \sl #1}}

% #####  Fettschrift fuer Vektoren  #####
\newcommand{\vek}[1]{\ensuremath{\mathbf{#1}}}    % (fette) Vektoren oder Matrizen (mit Buchstabe als Argument)
\newcommand{\bs}[1]{\mbox{\boldmath$#1$}}         % (fette) schraege Vektoren oder Matrizen


% EOF
             % Must be included AFTER this file!            ##
% #####################################################################################

% Komfortables Auskommentieren ganzer LaTeX-Passagen
\newcommand{\comment}[1]{}

% Grafikeinbindung
\usepackage[dvips]{graphicx}
\newcommand{\bildwi}[2]{\includegraphics[width=#2\textwidth]{#1}}
\newcommand{\bildsc}[2]{\includegraphics[scale=#2]{#1}}
\newcommand{\bild}[2]{\includegraphics[width=#2\textwidth]{#1}}  % for compatibility
\newcommand{\centerbild}[1]{\centerline{#1}}                     % for compatibility

% Fettschrift im Mathemodus
\newcommand{\bfm}[1]{\mathbf{#1}}

% Definiere \bf, \sf etc. so um, dass sie sich bei gleichzeitiger Anwendung 
% sich nicht gegenseitig unwirksam machen.
\renewcommand{\rm}{\rmfamily}
\renewcommand{\sf}{\sffamily}
\renewcommand{\tt}{\ttfamily}
\renewcommand{\bf}{\bfseries}
\renewcommand{\it}{\itshape}
\renewcommand{\sl}{\slshape}
\renewcommand{\sc}{\scshape}
\newcommand{\md}{\mdseries}
\newcommand{\up}{\upshape}

% ###############  EOF  ###############
+
bzw.\ \verb+% mathe.tex - Mathematische (Mengen)Zeichen, Symbole und Funktionen
%             fuer LaTeX 2.09 und LaTeX 2e
%
%             Original von Dieter Boss, 08/94
%             Letzte Aenderung: H.Schmidt, 26-April-2000 (NDFT ... NIFFT)
%

\newcommand{\ds}[1]{\displaystyle{#1}}    % Kurzschreibweise fuer \displaystyle
\newcommand{\ml}[1]{\hbox{\large $#1$}}   % math large : Angenehme Schrift-
                                          % groesse bei der Darstellung von
                                          % Bruechen z.B. \ml{1\over\sqrt{2}}

% #####  Haeufig benoetigte Funktionen  #####
\newcommand{\E}[1]{\ensuremath{\mathrm{E}\left\{#1\right\}}}
\newcommand{\real}[1]{\ensuremath{\mathrm{Re}\left\{#1\right\}}}
\newcommand{\imag}[1]{\ensuremath{\mathrm{Im}\left\{#1\right\}}}
\newcommand{\rect}[1]{\ensuremath{\mathrm{rect}\left(#1\right)}}
\newcommand{\tri}[1]{\ensuremath{\mathrm{tri}\left(#1\right)}}
\newcommand{\si}{\ensuremath{\mathrm{si}}}
\newcommand{\di}{\ensuremath{\mathrm{di}}}
\newcommand{\ld}{\ensuremath{\mathrm{ld}}}  
\newcommand{\erf}{\ensuremath{\mathrm{erf}}}
\newcommand{\erfc}{\ensuremath{\mathrm{erfc}}}
\newcommand{\eds}[1]{\ensuremath{\mbox{e }^{\ds{#1}}}}
\newcommand{\ex}[1]{\ensuremath{e^{#1}}}
\newcommand{\ejO}{\ex{j\Omega}}

% #####  Mathematische Sonderzeichen  #####
\newcommand{\defas}{\ensuremath{\stackrel{\Delta}{=}}}

% #####  Mengenzeichen  #####
\newcommand{\Reell}{\mathsf{I} \kern -0.15em \mathsf{R}} 
\newcommand{\Nat}{\mathsf{I}  \kern -0.15em \mathsf{N}}
\newcommand{\Feld}{\mathsf{I} \kern -0.15em \mathsf{F}}
\newcommand{\Zahl}{\mathsf{Z} \kern -0.45em \mathsf{Z}}

% ##  Korrespondenz-"Knochen": ##
\newcommand{\korrespond}{\ensuremath{\;\circ \hskip-1ex -\hskip-1.2ex -\hskip-1.2ex- \hskip-1ex \bullet\;}}
\newcommand{\ikorrespond}{\ensuremath{\;\bullet \hskip-1ex -\hskip-1.2ex -\hskip-1.2ex- \hskip-1ex \circ\;}}

% #####  Transformationen  #####
\newcommand{\FT}[1]{\ensuremath{{\cal F}\left\{#1\right\}}}        % (kont.) Fourier-Trafo
\newcommand{\IFT}[1]{\ensuremath{{\cal F}^{-1}\left\{#1\right\}}}
\newcommand{\HT}[1]{\ensuremath{{\cal H}\left\{#1\right\}}}        % Hilbert-Trafo
\newcommand{\IHT}[1]{\ensuremath{{\cal H}^{-1}\left\{#1\right\}}}
\newcommand{\LT}[1]{\ensuremath{{\cal L}\left\{#1\right\}}}        % Laplace-Trafo
\newcommand{\ILT}[1]{\ensuremath{{\cal L}^{-1}\left\{#1\right\}}}
\newcommand{\DFT}[1]{\ensuremath{\mathrm{DFT}\left\{#1\right\}}}   % Diskrete Fourier-Trafo
\newcommand{\IDFT}[1]{\ensuremath{\mathrm{IDFT}\left\{#1\right\}}}
\newcommand{\FFT}[1]{\ensuremath{\mathrm{FFT}\left\{#1\right\}}}   % Fast Fourier-Trafo
\newcommand{\IFFT}[1]{\ensuremath{\mathrm{IFFT}\left\{#1\right\}}}
\newcommand{\NDFT}[2]{\ensuremath{\mathrm{DFT}_{#2}\left\{#1\right\}}}   % Diskrete Fourier-Trafo
\newcommand{\NIDFT}[2]{\ensuremath{\mathrm{IDFT}_{#2}\left\{#1\right\}}}
\newcommand{\NFFT}[2]{\ensuremath{\mathrm{FFT}_{#2}\left\{#1\right\}}}   % Fast Fourier-Trafo
\newcommand{\NIFFT}[2]{\ensuremath{\mathrm{IFFT}_{#2}\left\{#1\right\}}}\newcommand{\ZT}[1]{\ensuremath{{\cal Z}\left\{#1\right\}}}        % Z-Trafo
\newcommand{\IZT}[1]{\ensuremath{{\cal Z}^{-1}\left\{#1\right\}}}
\newcommand{\DTFT}[1]{\ensuremath{\mathrm{DTFT}\left\{#1\right\}}}   % Diskrete Time Fourier-Trafo
\newcommand{\IDTFT}[1]{\ensuremath{\mathrm{IDTFT}\left\{#1\right\}}}

% #####  Einheiten und Groessen  #####
\newcommand{\Hz}{\ensuremath{\mathrm{\:Hz}}}
\newcommand{\kHz}{\ensuremath{\mathrm{\:kHz}}}
\newcommand{\MHz}{\ensuremath{\mathrm{\:MHz}}}
\newcommand{\Mbits}{\ensuremath{\mathrm{\:Mbit/s}}}
\newcommand{\GHz}{\ensuremath{\mathrm{\:GHz}}}
\newcommand{\ms}{\ensuremath{\mathrm{\:ms}}}
\newcommand{\ns}{\ensuremath{\mathrm{\:ns}}}
\newcommand{\mus}{\ensuremath{\mathrm{\:\mu s}}}
\newcommand{\kmh}{\ensuremath{\mathrm{\:km/h}}}
\newcommand{\dB}{\ensuremath{\mathrm{\:dB}}}
\newcommand{\kbits}{\ensuremath{\mathrm{\:kbit/s}}}
\newcommand{\kBaud}{\ensuremath{\mathrm{\:kBaud}}}
\newcommand{\SNR}{\ensuremath{\frac{S}{N}}}
\newcommand{\EbN}{\ensuremath{\frac{E_b}{N_0}}}
\newcommand{\EbNh}{\ensuremath{\frac{E_b}{N_0/2}}}

% #####  Worte, die haeufig in Gleichungen gebraucht werden  #####
\newcommand{\Mit}{\quad\mathrm{mit}\;\,}          % kleingeschrieben existiert \mit schon!
\newcommand{\und}{\quad\mathrm{und}\;\,}
\newcommand{\da}{\quad\mathrm{da}\;\,}
\newcommand{\fuer}{\quad\mathrm{f"ur}\;\,}
\newcommand{\wobei}{\quad\mathrm{wobei}\;\,}
\newcommand{\mindex}[1]{\mbox{\scriptsize \sl #1}}

% #####  Fettschrift fuer Vektoren  #####
\newcommand{\vek}[1]{\ensuremath{\mathbf{#1}}}    % (fette) Vektoren oder Matrizen (mit Buchstabe als Argument)
\newcommand{\bs}[1]{\mbox{\boldmath$#1$}}         % (fette) schraege Vektoren oder Matrizen


% EOF
+ eingebunden (wie es auch hier der Fall
ist).  Dabei kann {\tt mathe.tex}  erst {\em hinter} {\tt latex2e.tex}
eingebunden werden, in der umgekehrten Reihenfolge funktioniert es
nicht!

Auch hier gilt, da"s auf Anregung der Benutzer weitere Elemente
eingebracht werden k"onnen. Die "Anderungen kann allerdings nicht jeder
Fuzzi eigenst"andig vornehmen (ein Gl"uck), er sollte sich bei Bedarf
an eine kompetente Person wenden.


\section{Die Datei {\tt latex2e.tex}} \label{SECbild}

\subsection{L"angere Kommentare}
%
\LaTeX\ erlaubt das Auskommentieren von Passagen aus dem \LaTeX-File
nur durch voranstellen eines Prozentzeichens. Dies kommentiert allerdings
nur den Test hinter dem Prozentzeichen {\em bis zum n"achsten Zeilenumbruch} aus.
Will man ganze Abschnitte auskommentieren, so m"u"ste man jeder Zeile
m"uhsam ein Prozentzeichen verpassen --~wenn es da nicht den Befehl
\verb+\comment{...}+ aus der Datei {\tt latex2e.tex} g"abe:
Damit sind Kommentare f"ur l"angere Textpassagen komfortabel zu
erreichen. Die (nicht) gew"unschte Textstelle einfach in die geschweiften 
Klammern des Befehls setzen --- der Text wird dann beim Compilieren nicht mehr
ber"ucksichtigt. Bei l"angeren Auskommentierungen ist allerdings sehr
zu empfehlen, das Ende des Kommentars nicht nur durch eine geschweifte Klammer
zu kennzeichnen (die man sp"ater nie wieder findet, bei soo vielen Klammern),
sondern durch eine Zeile wie z.B.\ ``\verb+} % END OF \comment+''. So hat man auch
mit einem ``dummen'' Texteditor die Chance, das Kommentarende wiederzufinden.

\subsection{Einbinden von Grafiken} \label{SUBSECbilder}
%
Die Datei {\tt latex2e.tex} definiert auch Befehle zur Grafikeinbindung.
So binden die folgenden Zeilen das Postscript-Bild {\tt muster.pst} im 
Directory {\tt PS} in das \LaTeX-Dokument ein.
%
\begin{figure}[htb]
  \centerline{ \bildsc{PS/muster.pst}{0.48} }
  \caption{Fehlerraten. Anmerkung:
           Mehrzeilige Bildunterschriften sehen "ubrigens auch super
           aus, es sei denn, man w"ahlt \emph{nicht} die hier voreingestellte
           Konfiguration, sondern versucht es auf eigene Faust \ldots}
  \label{FIGmust}
\end{figure}
%
\begin{verbatim}
\begin{figure}[htb]
  \centerline{ \bildsc{PS/muster.pst}{0.48} }
  \caption{Fehlerraten. Anmerkung: ...}
  \label{FIGmust}
\end{figure}
\end{verbatim}
%
Dazu wird der Befehl \verb+\bildsc{FILENAME}{0.48}+ verwendet,
der das Bild auf 48\% seiner urspr"unglichen Gr"o"se skaliert.
W"urde man statt \verb+\bildsc+ den Befehl \verb+\bildwi+ benutzen,
so w"urde das Bild auf 48\% der Textbreite einer Seite 
skaliert\footnote{Eselsbr"ucke:
{\tt bildsc} kommt von {\em scale}, bei {\tt bildwi} wird auf feste {\em width} skaliert.}.\\
Mit \verb+\caption{xxx}+ wird der Abbildung eine Bildnummer und die Bildunterschrift
{\tt xxx} zugewiesen. Definiert man mit \verb+\label{FIGmust}+ ein label,
so kann aus dem Text durch \verb+\ref{FIGmust}+ auf die Bildnummer zugegriffen werden
(s.a.\ Abschnitt~\ref{SECverw}).

{\bf Bild-Plazierung:} Da die {\tt figure}-Umgebung ebenso wie die
Tabellen-Umgebung {\tt table} ein {\em floating object} definiert,
gelten f"ur die Plazierung von Bildern die in Abschnitt~\ref{SECtab}
f"ur Tabellen getroffenen Aussagen.  Im allgemeinen l"a"st sich der
Plazierungswunsch mit umso geringerer Wahrscheinlichkeit erf"ullen, je
gr"o"ser das Gleitobjekt ist. Mal sehen, wo Bild~\ref{FIGmust}
landet...

Mehrere Bilder bindet man wie folgt ein. Um die vier Bilder {\tt
b1.eps} $\cdots$ {\tt b4.eps} in einer $2 \times 2$ Matrix-Anordnung
einzubinden, k"onnte man folgende Konstruktion verwenden:

\begin{verbatim}
  \centerline{ \bildsc{PS/b1.eps}{0.16}\quad\bildsc{PS/b2.eps}{0.16} }
  \centerline{ \bildsc{PS/b3.eps}{0.16}\quad\bildsc{PS/b4.eps}{0.16} }
\end{verbatim}

Je nach IQ des Lesers d"urften somit auch andere Bildanordnungen (k)ein Problem sein ...

\subsection{Fettschrift im Mathemodus}

Weiterhin wird in {\tt latex2e.tex} das Kommando \verb+\bfm+ definiert,
das in seiner Form \verb+{\bfm H}+ im Mathemodus ein fettgedrucktes `H' darstellt.



\section{Die Datei {\tt mathe.tex}} \label{SUBmathe}
%
Die Datei {\tt mathe.tex} ist als ASCII-Text in Anhang~\ref{CPTmehrascii}
auf Seite~\pageref{CPTmehrascii} eingebunden und sollte vom interessierten Nutzer
aufmerksam gelesen werden. In Tabelle~\ref{TABmathe} sind die wichtigsten Definitionen
und ihre Wirkungen (im Mathemodus) angegeben.
%
\begin{table}[htb]\label{TABmathe}
\begin{center}
\begin{tabular}{|l|l|l|}
  \hline
  Thema               & Befehl                                    & Wirkung (im Mathemodus)\\
  \hline
  Komplexe Zahlen     & \verb+ \real{2j}, \imag{3j}             + & $\real{2j}, \imag{3j}$\\
                      & \verb+ \ejO, \ex{nix}                   + & $\ejO, \ex{nix}$\\
  Funktionen          & \verb+ \rect{T_0}, \tri{\frac{T}{2}}    + & $\rect{T_0}, \tri{\frac{T}{2}}$\\
                      & \verb+ \si(\omega t), \ld(2)            + & $\si(\omega t), \ld(2)$\\
  Einheiten           & \verb+ 4\kHz, 3\MHz, 2\mus, 1\ms        + & $4\kHz, 3\MHz, 2\mus, 1\ms$\\
                      & \verb+ 3\dB                             + & $3\dB$\\
  Vek.~\& Matrizen    & \verb+ \vek{a}, \vek{A}                 + & $\vek{a}, \vek{A}$\\
  Transformationen    & \verb+ f \korrespond F, F \ikorrespond f+ & $f \korrespond F, F \ikorrespond f$\\
                      & \verb+ \FT{K\,\delta(t)}=K              + & $\FT{K\,\delta(t)}=K$\\
                      & \verb+ \IFT{2\pi K\,\delta(\omega)}=K   + & $\IFT{2\pi K\,\delta(\omega)}=K$\\
                      & \verb+ \DFT{.}, \IDFT{.}                + & $\DFT{.}, \IDFT{.}$\\
                      & \verb+ \FFT{.}, \IFFT{.}                + & $\FFT{.}, \IFFT{.}$\\
                      & \verb+ \ZT{.},\IZT{.},\HT{.},\IHT{.}    + & $\ZT{.}, \IZT{.}, \HT{.}, \IHT{.}$\\
  K"urzel             & \verb+ \SNR, \EbN                       + & $\SNR, \EbN$\\
  \hline
\end{tabular}
\end{center}
\end{table}

Wie man am Beispiel des Befehls \verb+\tri{.}+ erkennen kann, sind
alle Kommandos, an die ein Argument "ubergeben wird, so definiert,
da"s sich die Gr"o"se der geschweiften Klammern automatisch an die 
Gr"o"se des Argumentes anpa"st.

In Gleichungen ben"otigt man des "ofteren bestimmte W"orter, die im
normalen Textmodus erscheinen sollen, wie es im folgenden Beispiel 
beim Wort ``f"ur'' der Fall ist:\ \ $x=\sqrt{\lambda} \fuer \lambda\leq 0$.
Deshalb sind in {\tt mathe.tex} die Befehle \verb+\und+, 
\verb+\da+, \verb+fuer+, \verb+\Mit+ und \verb+\wobei+ 
definiert, die einen Zwischenraum lassen (\verb+\quad+), dann das entsprechende
Wort im Textmodus darstellen und danach noch einen kleinen Zwischenraum lassen 
(\verb+\;\,+).

% EOF
           % Kapitel 4
  %\chapter{Cleveres} \label{CPTsim} 
%
\section{Symbolische Verweise} \label{SECverw}
%
Wer Verweise auf Seitennummern nach dem Muster ``wie man auf Seite 35 sieht'' 
{\em fest verdrahtet}, wird sich ziemlich bald dar"uber "argern, da"s er alle derartigen 
Verweise ab"andern mu"s, nur weil er {\em vor} Seite~35 Text eingef"ugt hat, 
so da"s sich alles folgende entsprechend verschiebt.

Deshalb fangen kluge Leute einen derartigen Pfusch gleich gar nicht an
und verwenden prinzipiell {\em symbolische Verweise}, d.h.\ Verweise nicht direkt
auf eine Zahl, sondern auf ein {\em label}. Dies gilt nicht nur f"ur Seitennummern,
sondern auch f"ur Kapitel-, Gleichungs-, Tabellen- und Bildernummern.
In \LaTeX\ steht dazu der Befehl \verb+\ref{LABELNAME}+ zur Verf"ugung, 
wobei dies voraussetzt, da"s man an der entsprechenden Stelle den Label {\tt LABELNAME}
mittels \verb+\label{LABELNAME}+ definiert hat.

Beispiel: Nehmen wir an, die Zeile \verb+\section{Fazit} \label{SECfazit}+
steht in einem \LaTeX-Dokument und erzeugt die "Uberschrift ``5.4 Fazit''.
Der symbolische Verweis auf Abschnitt~5.4 sieht dann so aus:\quad
\verb+wie in Abschnitt~\ref{SECfazit} gezeigt wurde+. Dabei stellt die
Tilde sicher, da"s \LaTeX\ keinen Zeilenumbruch zwischen {\tt Abschnitt} und
{\tt 5.4} vornimmt.

"Aquivalentes gilt nat"urlich auch f"ur \verb+\chapter{...}+, \verb+\subsection{...}+ etc..
Prinzipiell mu"s der \verb+\label+-Befehl immer direkt nach dem Anfang der entsprechenden
Umgebung stehen, bei Gleichungen also nach \verb+\begin{equation}+ oder der entsprechenden
Konstruktion, bei Tabellen nach \verb+\begin{table}+. Ausnahme: bei Abbildungen mu"s
der \verb+\label+-Befehl nach \verb+\caption{...}+ und vor \verb+\end{figure}+ stehen.
F"ur Verweise auf Seitennummern gibt es den speziellen Befehl \verb+\pageref{LABELNAME}+.
Der dazugeh"orige \verb+\label+-Befehl kann irgendwo im \LaTeX-Dokument zwischen 
\verb+\begin{document}+ und \verb+\end{document}+ stehen.

N.B.: Damit \LaTeX\ die symbolischen Verweise richtig aufl"ost,
mu"s zweimal kompiliert werden! Beim ersten Mal werden die Labels erzeugt und
beim zweiten Mal von \verb+\ref{...}+ ausgelesen. Kompiliert man nicht oft genug,
so erscheinen {\tt ??} statt der gew"unschten Referenzen.

Bemerkung: die einzigen Verweise, die nicht mit dem \verb+\label+- 
und \verb+\ref+-Befehlen bewerkstelligt werden, sind Referenzen auf die Literaturliste
(Bibliographie). Dies geschieht mit {\tt bibtex} und dem Befehl \verb+\cite{...}+
und ist in Abschnitt~\ref{SECbib} beschrieben.

\section{Literaturverzeichnis mit BibTeX} \label{SECbib}
%
Verweise auf das Literatur-Verzeichnis erfolgen mit dem
\verb+\cite{QUELLExx}+-Befehl, wobei {\tt QUELLExx} ein Label ist, das in der
Datei {\tt quellen.bib} definiert sein mu"s.  In dieser Datei werden
solchen Labeln alle Angaben zugewiesen (Autoren, Titel, Journal etc.).

Die Default-Datei enth"alt bereits einige Labels: {\tt KK92}, {\tt Kam92}, {\tt Pro95}.
Es hat sich als zweckm"a"sig erwiesen, die Labels bei nur einem Autor aus den drei Anfangsbuchstaben
des Nachnamens und der Jahreszahl zusammenzusetzen. Bei mehreren Autoren benutzt man die
Initialen der Nachnamen und die Jahreszahl.
Verwendet man \verb+\bibliographystyle{alpha}+ (wie es im ANT-Musterbericht der Fall ist),
so erzeugt der \verb+\cite+-Befehl Verweise in der Form \cite{KK92,Kam92} oder \cite{Pro95},
sofern das \LaTeX-Dokument einmal mit Bibtex ``behandelt'' und dann zweimal compiliert wurde.

% EOF
           % Kapitel 5
  %\begin{appendix}
\chapter{Einbinden von ASCII-Dateien} \label{CPTapp}
\thispagestyle{fancyplain}
%
Eine h"aufig auftretende Aufgabe bei der Dokumentation von studentischen
Ausarbeitungen ist das Aufnehmen von Programmlistings in das Dokument.
Solche als ASCII-Dateien vorliegenden Texte k"onnen komfortabel in 
\LaTeX-Dokumente eingebunden werden und m"ussen nicht etwa losgel"ost 
vom "ubrigen Text als Kopien beigelegt werden! Die Problematik um
dieses Thema wird h"aufig auch als ``Simulieren getippter Texte'' 
bezeichnet.

Folgenderma"sen bindet man ASCII-Dateien, z.B.\ {\sc Matlab}-m-Files oder 
auch diese Datei selbst, als reinen Text in \LaTeX\ ein: \verb+\getASCII{ANTanh.tex}+

Die Wirkung ist dann folgende:
\getASCII{ANTanh.tex}


\chapter{Die Datei {\tt mathe.tex}} \label{CPTmehrascii}
%
Ein weiteres Beispiel einer eingebundenen ASCII-Datei zeigt die 
z.Z.\ aktuelle, (hoffentlich) ber"uhmte Datei {\tt mathe.tex}:
\getASCII{mathe.tex}

\chapter{Zum Einbinden kurzer ASCII-Textstellen}
%
Ein alternative Methode des Einbindens besteht in der {\tt verbatim}-Umgebung.
Setzt man den folgenden Text zwischen \verb+\begin{verbatim}+ und
\verb+\end{verbatim}+, so erscheint er wie folgt:
%
\begin{verbatim}
Ein Satz, wie er 1:1 erscheint.
  Ein Satz, wie er 1:1 erscheint.
    Ein Satz, wie er 1:1 erscheint.
 Alle Leerzeichen tauchen also auf.
Alle  Leerzeichen tauchen also auf.
Alle Leerzeichen  tauchen also auf.

Sonderzeichen sind hier kein Problem: # $ ^ \ \\ " @  (u.v.m.)
\end{verbatim}

\end{appendix}
            % Anhang

% Es wird normalerweise nur in die Literaturliste uebernommen,
% was im Text durch \cite{} angesprochen wird. Mit \nocite{} koennen
% weitere Literaturstellen uebernommen werden, ohne dass diese im 
% weiteren Text auftauchen, z.B.:
% \nocite{Pro95,KK92}
% 
%   \bibliographystyle{alpha}
%   \thispagestyle{fancyplain}  % Literaturverzeichnis
%   {\small
%    \bibliography{quellen}
%    \addcontentsline{toc}{chapter}{Literatur}
%   }
% 


\comment{%
 Noch Einbringen, weitere Ideen:
   - Leerzeichen nach Punkt
   - Unterschiedliche Bindestriche
   - ...
} % end of \comment


\end{document}

# EOF
