\chapter{Introduction}
In 3G mobile communication systems Code Division Multiple Access (CDMA) has become widely acceptable access scheme, that fulfills the requirements to share a common medium or channel among users who take part in simultaneous transmission. Since CDMA uses codes to separate users, the interference among the users depends on the choice of the codes. Interference can be null when the codes are orthogonal to each other or it can be at certain tolerable level when non-orthogonal codes are used. This feature makes CDMA more attract-full for mobile communication environment. \\ \\
When the orthogonality is maintained, then the separation of the users becomes very simple at the receiver. But if the orthogonality is lost during the transmission over a channel, then each user see other users as potential interferer. In that case the detection of users require more sophisticated treatment. \\ \\
Optimum algorithms were derived for joint-detection of users which, later proved to be more complex for practical implementation. In sub-optimum approaches the joint-detection of users was achieved by separating the detector and decoder components and process the components individually. Such an approach leads to reduced complex structures but, with the cost of slight poor performance at low Signal-to-Noise ratios. \\ \\
Employing individual decoder and a joint detector and applying Turbo principle \cite{TUR} i.e., a part of the output from a component is feed back as an input to other component and vice-verse, leads to signification performance in achieving very low bit-error rates. This stunning performance has lead to analyze the convergence properties of the iterative decoding scheme.\\ \\
By the application of turbo principle in joint detection of users, we are now motivated to analyze the convergence properties of the iterative detection scheme. Such an analysis require to track the behavior of a single parameter which can characterize the whole system  and also common for all users. The single-parameter dynamical model of a coded CDMA was presented by \cite{JC} has lead to analyze the coded CDMA system in a unique way and also paved the way for various analysis tools. \\ \\
In our work we study the convergence properties of the joint-detector and decoder by tracking the parameters such as, variance of an estimation error, multiple-access interference and mutual information, which are common to all users and also characterize the system very well. We present the analysis tools in detail after the formal introduction to the coded CDMA systems. \\ \\
Documentation is organized in a way that \textbf{chapter 2} address the issue of which access scheme is more feasible to employ in mobile communication environment, by studying the advantages and disadvantages of various access schemes. After the brief study about the access schemes we present the uplink-transmission model in \textbf{chapter 3} and study each and every block in detail. \textbf{chapter 4} address the receiver structure designed in according to turbo-principle and we see why such a structure is necessary, by analyzing the optimum and sub-optimum detection schemes. In \textbf{chapter 5} we introduce the various analysis tools to study the convergence properties of the coded CDMA system and we present some of the simulated results as a proof for our claim.In \textbf{chapter 6} conclusion is drawn and also we state the reason for a kind of behavior seen in in the simulated results.
	
	

	


