\chapter{Mathematisches} \label{CPTmath}
%
In kaum einem im Arbeitsbereich Nachrichtentechnik verfa"sten Dokument 
wird man wohl um die Darstellung von mathematischen Zusammenh"angen 
herumkommen. Deshalb geht es in diesem Kapitel um die Darstellung von
Formeln, Gleichungen, Matrizen etc. Auf diesem Gebiet zeigen sich viele
St"arken von \LaTeX. So hat es einen speziellen {\em Mathemodus}, in den
man durch ein Dollarzeichen (``\verb+$+'') oder eine der Gleichungsumgebungen
kommt. Im Mathemodus sind dann unz"ahlige Kommandos verf"ugbar, die man
f"ur mathematische Darstellungen unbedingt ben"otigt. 

\section{Mathemodus im Flie"stext} \label{SECfliess}

Variablen, Formelzeichen und kurze Formeln mitten im Satz werden einfach in 
{\tt \$}'s eingeschlossen. So erzeugt zum Beispiel \verb+$12x - y = \sqrt{\pi}$+
die Gleichung $12x - y = \sqrt{\pi}$. 

Allerdings sollte man nicht versuchen,
beliebig gro"se Konstrukte in den Flie"stext einzubauen: insbesondere Br"uche
sollten besser mittels \verb+$1/2$+ als druch \verb+$\frac{1}{2}$+ dargestellt werden, 
da ersteres den lesbaren Output $1/2$ liefert, wogegen die \verb+\frac+-Konstruktion 
mickrig wird: $\frac{1}{2}$.

Man beachte, da"s auch einzelne Buchstaben
(sprich: Variablennamen) im Textmodus anders aussehen als im Mathemodus. Bsp.:
N im Textmodus und $N$ im Mathemodus (\verb+$N$+).

\section{Einfache Gleichungen} \label{SECeinfg}

Komplexere Formeln m"ussen h"aufig {\em abgesetzt} vom Rest des Textes erscheinen, 
wobei man eine Variante {\em ohne} und eine {\em mit} fortlaufender 
Gleichungsnumerierung unterscheidet. Die Zeilen 
%
\begin{verbatim}
\begin{displaymath}
  \sum_{i=0}^{n} |x_i| \neq \int_{-\alpha}^{\infty} \eta \, d\eta
\end{displaymath}
\end{verbatim}
%
ergeben die folgende Gleichung ohne Nummer:
%
\begin{displaymath}
  \sum_{i=0}^{n} |x_i| \neq \int_{-\alpha}^{\infty} \eta \, d\eta
\end{displaymath}
%
Tippm"ude Leute k"onnen statt \verb+\begin{displaymath}+ und \verb+\end{displaymath}+
auch einfach \verb+\[+ bzw.\ \verb+\]+ oder auch \verb+$$+ verwenden.

Die folgende Gleichung mit fortlaufender Numerierung
%
\begin{equation} \label{EQmit}
  \sum_{i=0}^{n-1} x_{ij}^2 \approx 
    2\Delta \cdot \left(\int_{-\pi}^{\pi} \frac{\Omega}{2} \, d\Omega \right)
\end{equation}
%
wurde durch 
%
\begin{verbatim}
\begin{equation} \label{EQmit}
  \sum_{i=0}^{n-1} x_{ij}^2 \approx 
    2\Delta \cdot \left(\int_{-\pi}^{\pi} \frac{\Omega}{2} \, d\Omega \right)
\end{equation}
\end{verbatim}
%
erzielt. Dabei kann mit \verb+\ref{EQmit}+ auf die Gleichungsnummer zugegriffen
werden (s.~Abschnitt~\ref{SECverw}).

\section{Mehrzeilige Gleichungen} \label{SECmehrg}

Mehrzeilige Formeln werden dann verwendet, wenn die Formel ganz einfach zu lang ist
oder mehrere separate Gleichungen an ihren Gleichheitszeichen horizontal ausgerichtet
werden sollen. Dazu verwendet man die {\tt eqnarray}-Umgebung. 
Auch hier kann entweder die automatische Gleichungsnumerierung benutzt 
werden oder auch nicht.
Die Zeilen
%
\begin{verbatim}
\begin{eqnarray*}
a &=& A \times B\\
  &=& \prod_{j=1}^{n} C_j \equiv \prod_{j=1}^{n} A_j  
           \quad \mbox{f"ur}\quad C_j = A_j\quad ,  
\end{eqnarray*}
\end{verbatim}
%
erzeugen die folgende nichtnumerierte Gleichung
%
\begin{eqnarray*}
a &=& A \times B\\
  &=& \prod_{j=1}^{n} C_j \equiv \prod_{j=1}^{n} A_j  
           \quad \mbox{f"ur}\quad C_j = A_j\quad ,  
\end{eqnarray*}
%
w"ahrend mit Numerierung sogar einzelne Zeilen von derselben einzeln 
ausgenommen werden k"onnen, wie die folgende Gleichung zeigt:
%
\begin{verbatim}
\begin{eqnarray} 
a &=&       b + \nu + \mu      \label{EQbnumu}\\
  &=&       3 + 4   + 5        \nonumber\\
A &\neq&    \beta + \nu + \mu  \\
  &\approx& 3.14  + 4.1 + 5.2  \nonumber\\
  &=&       1244\cdot 10^{-2}  \nonumber   \quad .
\end{eqnarray}
\end{verbatim}
%
\begin{eqnarray} 
a &=&       b + \nu + \mu      \label{EQbnumu}\\
  &=&       3 + 4   + 5        \nonumber\\
A &\neq&    \beta + \nu + \mu  \\
  &\approx& 3.14  + 4.1 + 5.2  \nonumber\\
  &=&       1244\cdot 10^{-2}  \nonumber   \quad .
\end{eqnarray}
%
Alle bisherigen Gleichungsformel haben ein gemeinsames Kennzeichen.
Sie setzen \LaTeX\ in den {\em Mathemodus}. Viele der oben gezeigten Befehle
(z.B.\ alle griechischen Buchstaben, die Hoch- und Tiefstellung mit ``\verb+^+'' 
bzw.\ ``\verb+_+'', das Integral-, Summen- und Produktzeichen etc).
sind nur im Mathemodus zul"assig und f"uhren zu Fehlern, wenn man versucht,
sie im Textmodus anzuwenden.

\section{Weitere Befehle im Mathemodus} \label{SECbeisp}
%
\begin{itemize}
\item Vergleichssymbole: $3<2$ (\verb+$3<2$+);\quad  $x \leq y$ (\verb+$x \leq y$+);\quad $a \gg b$ (\verb+$a \gg b$+).
\item Funktionen sind \underline{keine} Variablen, werden also 
      nicht etwa kursiv, sondern als normaler Text dargestellt.
      Dazu stehen die Befehle \verb+\sin, \log, \lim, \max, \det+ im Mathemodus zur Verf"ugung. 
      $$
      \sin{x}\neq sin x, \ \log(10), \ \lim_{t\rightarrow\infty} y(t)=0, \
            \max_{j\neq i} V_j, \ \det{A}, \quad \ldots
      $$
\item Br"uche: \verb+$\frac{\pi^2}{2}$+ erzeugt im Flie"stext den kleinen Output $\frac{\pi^2}{2}$,
      in der Gleichungsumgebung dagegen:
$$\frac{\pi^2}{2}$$
\item Die folgende $(m \times n)$ Matrix ${\bfm A}$ 
%
$$
{\bfm A} := 
\left(\begin{array}{cccc}   % Vier Spalten mit zentrierten ("c") Elementen
  a_{11} & a_{12} & \cdots & a_{1n}\\
  a_{21} & a_{22} & \cdots & a_{2n}\\
  \vdots & \vdots & \ddots & \vdots\\
  a_{m1} & a_{m2} & \cdots & a_{mn}
\end{array}\right)
$$
%
wurde mit Hilfe der {\tt array}-Umgebung dargestellt.
%
\begin{verbatim}
$$
{\bfm A} := 
\left(\begin{array}{cccc}   % Vier Spalten mit zentrierten ("c") Elementen
  a_{11} & a_{12} & \cdots & a_{1n}\\
  a_{21} & a_{22} & \cdots & a_{2n}\\
  \vdots & \vdots & \ddots & \vdots\\
  a_{m1} & a_{m2} & \cdots & a_{mn}
\end{array}\right)
$$
\end{verbatim}
\item Faltung: \verb+$y(t)=h(t)\ast x(t)$+ erzeugt das Resultat $y(t)=h(t)\ast x(t)$.
\item Punkte: $\cdot$  (\verb+$\cdot$+); \quad  $\ldots$ (\verb+$\ldots$+);\quad   
              $\cdots$ (\verb+$\cdots$+);\quad  $\vdots$ (\verb+$\vdots$+);\quad
              $\ddots$ (\verb+$\ddots$+)
\item Folgepfeile: so $U \rightarrow X$ (mit \verb+\rightarrow+) oder so: $U \Rightarrow X$ (mit \verb+\Rightarrow+)
%
\end{itemize}


% EOF
