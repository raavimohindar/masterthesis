\chapter{Introduction to Channel Access Schemes}
In this chapter we study about various channel access schemes in detail and we choose the best suited access scheme for mobile communication environment.
\section{The Multi-Access Channel}
The idea of sending information from several users through a common channel is not new. Such an idea was realized through telegraph system, in which common channel is used for transmitting and receiving the information. 
\begin{figure}[htb]
  \centerline{ \bildsc{ps/c1_mu.eps}{1.0} }
  \caption{Multiple-Access Communication}
%  \label{Multiple-Access\\ Communication}
\end{figure}
In \textbf{Figure 2.1} shows the miniaturized scenario of multi-user communication in which several users share the common channel to transmit the information from one end to another end. \\ \\
In order to utilize the channel by all the users at maximum efficiency, several channel access schemes were proposed. Some of those schemes are Frequency Division Multiple Access (FDMA), Time Division Multiple Access (TDMA) and Code Division Multiple Access (CDMA). We study first two schemes and state the reasons why those schemes are not fully suitable to employ in multi-user communication environment.
\section{Frequency Division Multiple Access}
Frequency Division Multiple Access (FDMA) is an analogue transmission techniques as shown in \textbf{Figure 2.2}, in which the available bandwidth is divided into sub-bands of frequencies or channel and each sub-band of frequency or channel is allocated to one or more users to carry either voice conversation or digital data. Using FDMA in this way, several users can share the available band without the risk of interference between them. FDMA is the primary access scheme used in 1G (first-generation) analogue systems such as AMPS (Advanced Mobile Phone Systems).
\begin{figure}[htb]
  \centerline{ \bildsc{ps/fdma.eps}{1.0} }
  \caption{Frequency Division Multiple Access}
%  \label{FIGmust}
\end{figure}\\
We now list out the advantages and disadvantages of Frequency Division Multiple Access.
\subsection{Advantages of FDMA}
\begin{itemize}
\item Channels can be assigned on-demand when a user needs to communicate from one end to another.
\item Due to large symbol duration when compared to average delay spread leads to low Inter Symbol Interference (ISI).
\item Since transmission is continuous, little synchronization is required in FDMA.
\end{itemize}
\subsection{Disadvantages of FDMA}
\begin{itemize}
\item Each channel must contain guard bands. In a bandwidth constrained channels guard bands occupies the considerable amount of bandwidth, resulting in waste of available resources.
\item Each user is assigned to one or more bands, when users has got nothing to transmit those bands remain un-used and that leads to spectral in-efficiency.
\item It requires expensive filters to reduce adjacent channel interference.
\item Inter-modulation leads to undesired RF radiation that leaks into other channel in FDMA systems.
\item Generation of undesirable harmonics due to Inter-modulation that cause interference to other users in multi-user systems.
\item During roaming from one cell to another cell, users are not allocated to a particular time-slot on one cell to a particular time-slot of another cell. When all the time-slots are occupied in the another cell there is possibility that a call might be dropped.
\end{itemize}
\section{Time-Division Multiple Access}
With the growth in complexity of the transmission schemes and increase in the demand for more number of services, FDMA shows some in-consistency to adopt to those requirements. So the resulting access scheme is modified in way that available resource is divided into time-slices rather then sub-bands of frequency as shown in \textbf{Figure 2.3}, 
\begin{figure}[htb]
  \centerline{ \bildsc{ps/tdma.eps}{1.0} }
  \caption{Time Division Multiple Access}
%  \label{FIGmust}
\end{figure}
and one or more time slots are assigned to each user depending upon the demand and each user is served either in a round-robin fashion or with certain scheduling when the priority of certain user is higher then the others.
\subsection{Advantages of TDMA}
\begin{itemize}
\item TDMA substantially improves the efficiency of analog cellular.
\item The most obvious advantage is that it can be easily adopted to the transmission of data as well as voice communication.
\item In telephone lines using TDMA can carry data at the rates of 64 kbps to 128 Mbps. This enables to offer services like fax, data and SMS as well as bandwidth-intensive applications such as multimedia and videoconferencing.
\item TDMA separates users in time, which ensures that they will not interfere with other users from simultaneous transmissions.
\end{itemize}
\subsection{Disadvantages of TDMA}
\begin{itemize}
\item Requires careful time synchronization.
\item One of the primary disadvantages of TDMA is that each user has a pre-assigned time slot. When user has got nothing transmit during its turn, those time-slots cannot be utilized and leads to possible wasting of bandwidth.
\item During roaming from one cell to another cell, users are not allocated to a particular time-slot on one cell to a particular time-slot of another cell. When all the time-slots are occupied in the another cell there is possibility that a call might be dropped.
\end{itemize}
\section{Code Division Multiple Access}
The channel-sharing approaches which we discuss in \textbf{section 2.2} and \textbf{section 2.3} are based on the philosophy that only one user can occupy the time/frequency slot at a given time, if this condition is violated then the receiver could not able to demodulate the user successfully.
\begin{figure}[htb]
  \centerline{ \bildsc{ps/cdma.eps}{1.0} }
  \caption{Code Division Multiple Access}
%  \label{FIGmust}
\end{figure}
The multiple-access schemes in \textbf{section 2.2} and \textbf{section 2.3} fall into a drawback particularly when applied to mobile communication environment, where all users have to transmit simultaneously over a common channel. \\ \\
Now the question arises, how users co-exist in time/frequency without interfering with each other? \\ \\
Answer is to use codes or signals to separate users as shown in \textbf{Figure 2.4}. The codes assigned to one user is assumed to be orthogonal to the codes assigned to other users. If the orthogonality is maintained the separation of users become very simple at the receiver side.\\ \\
We now have to discuss an inherent component in CDMA called spreading. \\ \\
Spreading is a process of extending the bandwidth of the information signals much beyond the information rate thereby signal energy spreads beyond the bandwidth of the signal. Systems which employs CDMA as access schemes can use either frequency hopping or direct-sequence techniques to spread the signals, since the type of spreading to be employed in a communication system solely depends upon the type of modulation used in that system \cite{Pro01}. Since, we use BPSK to modulate the signals and our obvious choice is to choose direct-sequence spread spectrum techniques to spread the information signals. Extension of band is done by multiplying the information signal to Pseudo-Noise sequence there by signals resembles likes a noise so it is very difficult for the unintended user to demodulate it.\\ \\
At the receiver information is recovered by de-spreading the signal.
\subsection{Direct-Sequence Spread Spectrum}
\begin{figure}[htb]
  \centerline{ \bildsc{ps/ds_cdma.eps}{1.0} }
  \caption{Direct-Sequence Spread-Sprectrum Techniques}
%  \label{FIGmust}
\end{figure}
Direct-Sequence spread spectrum is very simple spreading technique in which the phase of the information bits is shifted according to the spreading sequence. As we know that spreading sequence are like those of pseudo-noise and the signals after spreading are like those of pseudo-noise to the un-intended user, hence the demodulation is not possible unless the user knows that particular spreading sequence. This is one of the ideal requirements for a access scheme to be employed in multi-user communication environment. \\ \\
We will illustrate the process of spreading with very simple example as shown in \textbf{Figure 2.5}. The pulse duration $T_c$ of the PN sequence is called \textit{chips} and $W=1/T_c$ be the rate of the chips and $T_b$ is the pulse duration of the information bit and $R_b=1/T_b$ be the information rate and we see $T_c$ is $N_s$ times faster then $T_b$.
\begin{equation}
N_s=T_b/T_c
\end{equation}
where $N_s$ is the spreading factor and in most of the practical systems $N_s$ is an integer and which tells number of chips per information bit. We now define the bandwidth expansion factor as $B_e$
\begin{equation}
B_e=W/R_b
\end{equation}
and the load of the system $\beta=u/N_s$, where $1\,\leq\, u\, \leq\, U$ are users in the communication system.
