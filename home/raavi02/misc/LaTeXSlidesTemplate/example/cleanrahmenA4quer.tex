\special{landscape}
% ###################################################
% ##       A4-quer-Rahmen fuer Folien  etc.        ##
% ###################################################
% Autoren:  Boss, Jelonnek, Dickmann;   07/94...02/96
%
% Jede Seite hat A4-Querformat und folgendes Layout:
%
%      ----------------------------------------
%      |                                      |
%      |                                      |
%      |                                      |
%      |                                      |
%      |                                      |
%      |                                      |
%      |                                      |
%      |                                      |
%      |                                      |
%      |                                      |
%      ----------------------------------------
%      | \datum |    \titelzeile    | Seite x |
%      ----------------------------------------
%
% - Empfohlene Seitenmasse (im top level LaTeX file definieren):
%     \textwidth25.7cm
%     \textheight16cm
%     \topmargin-16mm
%     \oddsidemargin-2.5mm
%     \evensidemargin-2.5mm
% - Die Variablen \titelzeile und \datum sind im Top level LaTeX File
%   wie folgt zu definieren:
%     \setbox\titelzeile=\hbox{ITG-Fachgruppensitzung}
%     \setbox\datum=\hbox{\large 25.02.94}
% - Ob "Seite x" oder "Page x" etc erscheint, haengt von der gewaehlten
%   Sprache ab ("\usepackage[german]{babel}" etc.).
% - Zum Einbinden eines Postscript-Logos in das Datum-Feld ist der Befehl "\includelogo"
%   vordefiniert, der das Kommando "\bildwi{}{}" benoetigt. In dieser Version
%   ist die Logo-Einbindung auskommentiert. Falls dies geaendert werden soll,
%   ist nach "LOGO" zu suchen, um die relevanten Stellen zu finden.
% - Anschauen mit "xdvi -s 4 -paper a4r FILENAME &"
%
%/usr/local/lib/texmf/tex/latex/ANT/rahmenlogo.eps
%\newcommand{\includelogo}{\bildwi{rahmenlogo.eps}{0.15}}   % LOGO: diese Zeile nicht auskommentierten
\newcommand{\horizontaloffset}{-18mm}
\newcommand{\vertikaloffset}{0mm}       % -16mm
\newcommand{\gesamtbreite}{27.5cm}
\newcommand{\gesamthoehe}{19cm}
\newcommand{\kopfhoehe}{14mm}
\newcommand{\fusshoehe}{14mm}
\newcommand{\datumbreite}{38mm}         % 28mm; mit LOGO: 38mm
\newcommand{\titelbreite}{19.85cm}      % 21cm; mit LOGO: 19.85cm
\newcommand{\seitennrbreite}{38mm}      % 28mm; mit LOGO: 38mm
\newcommand{\linienabstand}{0.5mm}

% Kopfdeklaration (siehe ST-Computer Juni`90)
\newbox\titelzeile
\setbox\titelzeile=\hbox{}
\newbox\datum
\setbox\datum=\hbox{}
\newbox\author
\setbox\author=\hbox{}
\makeatletter
\def\ps@headings{\let\@mkboth\markboth
  \def\@oddfoot{}
  \def\@evenfoot{}
  \def\@evenhead{}

  \def\@oddhead{\protect\hbox to \gesamtbreite{\hss \hskip \horizontaloffset\vtop to 0cm{\vskip\vertikaloffset
                   \hbox{ \vtop to \gesamthoehe{
                         \hbox to \gesamtbreite{
                         \hrulefill\hspace{2mm}\raisebox{-4mm}[4mm]{\bildwi{ANT.eps}{0.1}}\hspace{2mm}\rule{10mm}{0.15mm}}
                         \vss
                         \vspace{\linienabstand}
                         \hrule
                         \hbox to \gesamtbreite{
                              \vbox to \fusshoehe{\vss
                                   \hbox to \datumbreite{\hss \vbox{\hbox{\sf\unhcopy\datum}}\hss}
                                   \vss}
                              \hss
                              \vbox to \fusshoehe{\vss
                                    \hbox to \titelbreite{\hss\vbox{\hbox{\sf\unhcopy\titelzeile}}\hss}
                                    \vss}
                              \hss
                              \vbox to \fusshoehe{\vss\hbox to \seitennrbreite{\hss \vbox{\hbox{\sf\huge \thepage \ }}\hss}\vss}
                              \hss
                          }
                        }}
              \vss}\hss}}}
\makeatother

\setbox\datum=\hbox{\hspace*{0.25cm}\bildwi{uniHB2.eps}{0.26}}
