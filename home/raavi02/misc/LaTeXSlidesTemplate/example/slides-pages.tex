%%% Folie 1 %%%%%%%%%%%%%%%%%%%%%%%%%%%%%%%%%%%%%%%%%%%%%%%%%%%%%%%%%%%%%%%%%%%%
\setbox\titelzeile=\hbox{\Large\sc Inhalt}
\vbox to \vsize{\vss
\vspace*{2cm}
%\hbox{{\bf .} }}
\begin{huge}
  \setlength{\unitlength}{1cm}
  \begin{picture}(26,16)
    \put(10,14){\today}
    \put(4,12){\Huge \Blue{Quellenseparierung mit dem fastICA-Algorithmus}}
    \put(10,10){J�rgen Rinas}

    \put(4,5){Inhalt}
    \put(4,4){$\bullet$ Ansatz $\Longrightarrow$ Algorithmus}
    \put(4,3){$\bullet$ Leistungsf�higkeit (Vergleich mit JADE)}
    \put(4,2){$\bullet$ Zusammenfassung, Ausblick}
  \end{picture}
\end{huge}
\vfill
\vss}
\newpage
%%% Folie 2 %%%%%%%%%%%%%%%%%%%%%%%%%%%%%%%%%%%%%%%%%%%%%%%%%%%%%%%%%%%%%%%%%%%%
\setbox\titelzeile=\hbox{\Large\sc Ansatz $\Longrightarrow$ Algorithmus}
\vbox to \vsize{\vss
\vspace*{2cm}
\hbox{\textbf{Ansatz} $\Longrightarrow$ Algorithmus}
\begin{huge}
  \setlength{\unitlength}{1cm}
  \begin{picture}(26,16)
    \put(0,14){$\bullet$ �bertragungssystem: $\mathbf{x} = \mathbf{H} \mathbf{c} + \mathbf{\nu}$}
    \put(0,13){\phantom{$\bullet$ �bertragungssystem:} $\mathbf{x}$ \begin{Large}$n_R \times 1$\end{Large} Empfangssignale}
    \put(0,12){\phantom{$\bullet$ �bertragungssystem:} $\mathbf{H}$ \begin{Large}$n_R \times n_T$\end{Large} Kanalmatrix (Block-Fading, Schmalbandmodell)}
    \put(0,11){\phantom{$\bullet$ �bertragungssystem:} $\mathbf{c}$ \begin{Large}$n_T \times 1$\end{Large} Sendesignale}
    \put(0,10){\phantom{$\bullet$ �bertragungssystem:} $\mathbf{\nu}$ \begin{Large}$n_R \times 1$\end{Large} Kanalrauschen}

    \put(0, 8){$\bullet$ Empf�nger: $\tilde{c}_t = \mathbf{b}^T_t \overbrace{\mathbf{W} \mathbf{x}}^{=: \mathbf{y}}$}
    \put(0, 7){\phantom{$\bullet$ Empf�nger:}  $\mathbf{W}$ \begin{Large}$n_T \times n_R$\end{Large} Whitening-Matrix (bestimmt durch EVD von $\mathbf{R_{xx}} = \E{\mathbf{x} \mathbf{x}^T}$)}
    \put(0, 6){\phantom{$\bullet$ Empf�nger:}  $\mathbf{b}_t^T$ \begin{Large}$1 \times n_T$\end{Large} Separationsvektor f�r ein gesendetes Signal $t$}
    \put(0, 5){\phantom{$\bullet$ Empf�nger:}  $\tilde{c}_t$ \begin{Large}Skalar\end{Large} -- ein separiertes Signal}

    \put(0, 4){zur Implementierung werden aus allen Vektoren Matrizen der Signall�nge $L$}

    \put(0,3){$\bullet$ Ansatz: Maximierung oder Minimierung der Kurtosis $\mathrm{cum_4}(\tilde{\mathbf{c}}_t) = \mathrm{kurt}(\tilde{\mathbf{c}}_t)$}
    \put(0,2){$\bullet$ Beispiel: BPSK  $d \in \lbrace -1, 1\rbrace$}

    \put(0,1){\phantom{$\bullet$ Beispiel: BPSK} $\mathrm{cum_1}(d) = 0$, $\mathrm{cum_2}(d) = 1$, $\mathrm{cum_3}(d) = $, $\mathrm{cum_4}(d) = -2$}
  \end{picture}
\end{huge}
\vfill
\vss}
\newpage
%%% Folie 3 %%%%%%%%%%%%%%%%%%%%%%%%%%%%%%%%%%%%%%%%%%%%%%%%%%%%%%%%%%%%%%%%%%%%
\setbox\titelzeile=\hbox{\Large\sc Ansatz $\Longrightarrow$ Algorithmus}
\vbox to \vsize{\vss
\vspace*{2cm}
\hbox{\textbf{Ansatz} $\Longrightarrow$ Algorithmus}
\begin{huge}
  \setlength{\unitlength}{1cm}
  \begin{picture}(26,16)
    \put(0,15){$\mathrm{kurt}(\tilde{\mathbf{c}}_t)$ = $\mathrm{kurt}(\mathbf{b}^T_t \mathbf{y})$}
    \put(0,14){\phantom{$\mathrm{kurt}(\tilde{\mathbf{c}}_t)$} $\downarrow$ $\mathrm{kurt}(a) = \E{a^4} - 3(\E{a^2})^2$}
    \put(0,13){\phantom{$\mathrm{kurt}(\tilde{\mathbf{c}}_t)$} = $\E{(\mathbf{b}^T_t \mathbf{y})^4} - 3(\E{(\mathbf{b}^T_t \mathbf{y})^2})^2$}
    \put(0,12){\phantom{$\mathrm{kurt}(\tilde{\mathbf{c}}_t)$} $\downarrow$ $\mathbf{y}$ ist wei�}
    \put(0,11){\phantom{$\mathrm{kurt}(\tilde{\mathbf{c}}_t)$} = $\E{(\mathbf{b}^T_t \mathbf{y})^4} - 3 ||\mathbf{b}_t||^4$}
    \put(0,10){$\mathbf{b}_t$ soll die Norm $1$ haben (alle $\mathbf{b}_t$ bilden sp�ter eine unit�re Matrix $\mathbf{B}$)}
    \put(0, 9){$\Longrightarrow$ "`penalty term"' einf�hren}
    \put(0, 8){\phantom{$\Longrightarrow$} $J(\mathbf{b}_t) = \E{(\mathbf{b}^T_t \mathbf{y})^4} - 3 ||\mathbf{b}_t||^4 + F(||\mathbf{b}_t||^2)  $  $\Leftarrow$ Kostenfunktion $\min\limits_{\mathbf{b}_t}$ oder $\max\limits_{\mathbf{b}_t}$}
    \put(0, 7){Differenzieren nach $\mathbf{b}_t$:}
    \put(0, 6){\phantom{$\Longrightarrow$ $J(\mathbf{b}_t) = $} $ \E{\mathbf{y}(\mathbf{b}^T_t \mathbf{y})^3} - 3 ||\mathbf{b}_t||^2 \mathbf{b}_t + f(||\mathbf{b}_t||^2) \mathbf{b}_t = 0  $ }
    \put(0, 5){L�sen mit gew�hnlichem Iterationsverfahren (\textit{fixed-Point Iteration}) $x=f(x)$}
    \put(0, 4){\phantom{$\Longrightarrow$ $J(\mathbf{b}_t) = $} $ \mathbf{b}_t = \underbrace{- 1 /f(||\mathbf{b}_t||^2)}_{\mathrm{Skalar}} \left( \E{\mathbf{y}(\mathbf{b}^T_t \mathbf{y})^3} - 3 ||\mathbf{b}_t||^2 \mathbf{b}_t \right)  $ }
    \put(0, 2){Norm von $\mathbf{b}_t$ hier unwichtig bzw. $||\mathbf{b}_t||^2=1$ sichergestellt}
    \put(0, 1){\phantom{$\Longrightarrow$ $J(\mathbf{b}_t) = $} $ \mathbf{b}_t =  \E{\mathbf{y}(\mathbf{b}^T_t \mathbf{y})^3} - 3 \mathbf{b}_t$ }
  \end{picture}
\end{huge}
\vfill
\vss}
\newpage
%%% Folie 4 %%%%%%%%%%%%%%%%%%%%%%%%%%%%%%%%%%%%%%%%%%%%%%%%%%%%%%%%%%%%%%%%%%%%
\setbox\titelzeile=\hbox{\Large\sc Ansatz $\Longrightarrow$ Algorithmus}
\vbox to \vsize{\vss
\vspace*{2cm}
\hbox{Ansatz $\Longrightarrow$ \textbf{Algorithmus}}
\begin{huge}
  \setlength{\unitlength}{1cm}
  \begin{picture}(26,16)
    \linethickness{1mm}
    \put(0,14){Separation \textit{eines} Signals $c_t$ -- \textit{fixed point} Iteration bzgl. $\mathbf{b}_t$}
    \put(0,12){\begin{Large}$n_t \times 1$ Vektor\end{Large} $\mathbf{b}_t(0)^\dagger$ mit Zufallszahlen initialisieren}
    \put(0, 11){$\mathbf{b}_t(0) = \mathbf{b}_t^\dagger(i+1) / ||\mathbf{b}_t^\dagger(0)|| $ \hspace{2cm} $\Leftarrow$ $||\mathbf{b}_t(0)||^2\stackrel{!}{=}1$}
    \put(0.5,10){\line(1,0){1}}
    \put(0.5,10){\line(0,-1){3}}
    \put(0.5, 7){\line(1,0){1}}
    \put(2,10){\begin{Large}Iteration $i=0 \dots 9$\end{Large}}
    \put(1, 9){$\mathbf{b}_t^\dagger(i+1) = \E{\mathbf{y}(\mathbf{b}^T_t(i) \mathbf{y})^3} - 3 \mathbf{b}_t(i)$}
    \put(1, 8){$\mathbf{b}_t(i+1) = \mathbf{b}_t^\dagger(i+1) / ||\mathbf{b}_t^\dagger(i+1)|| $ \hspace{2cm} $\Leftarrow$ $||\mathbf{b}_t(i+1)||^2\stackrel{!}{=}1$}
    \put(0, 6){$\tilde{c}_t = \mathbf{b}^T_t \mathbf{y}$}
  \end{picture}
\end{huge}
\vfill
\vss}
\newpage
%%% Folie 5 %%%%%%%%%%%%%%%%%%%%%%%%%%%%%%%%%%%%%%%%%%%%%%%%%%%%%%%%%%%%%%%%%%%%
\setbox\titelzeile=\hbox{\Large\sc Ansatz $\Longrightarrow$ Algorithmus}
\vbox to \vsize{\vss
\vspace*{2cm}
\hbox{{Ansatz $\Longrightarrow$ \textbf{Algorithmus}}}
\begin{huge}
  \setlength{\unitlength}{1cm}
  \begin{picture}(26,16)
    \linethickness{1mm}
    \put(0,14){Separation \textit{aller} Signale $c_t$}
    \put(0,13){\ANTblau{$\mathbf{B}_0=[\,\,]$}}
    \put(0.5,12){\line(1,0){1}}
    \put(0.5,12){\line(0,-1){11}}
    \put(0.5, 1){\line(1,0){1}}
    \put(2, 12){\begin{Large}Iteration  $t=0 \dots n_T-1$\end{Large}}

    \put(1,11){\begin{Large}$n_t \times 1$ Vektor\end{Large} $\mathbf{b}_t^\ddagger(0)$ mit Zufallszahlen initialisieren }
    \put(1,10){\ANTblau{$\mathbf{b}_t^\dagger(0) = \mathbf{b}_t^\ddagger(0) -\mathbf{B}_0\mathbf{B}_0^T\mathbf{b}_t^\ddagger(0)$}}
    \put(1, 9){$\mathbf{b}_t(0) = \mathbf{b}_t^\dagger(0) / ||\mathbf{b}_t^\dagger(0)|| $ \hspace{2cm} $\Leftarrow$ $||\mathbf{b}_t(0)||^2\stackrel{!}{=}1$}
    \put(1.5, 8){\line(1,0){1}}
    \put(1.5, 8){\line(0,-1){4}}
    \put(1.5, 4){\line(1,0){1}}
    \put(3, 8){\begin{Large}Iteration  $i=0 \dots 9$\end{Large}}
    \put(2, 7){$\mathbf{b}_t^\ddagger(i+1) = \E{\mathbf{y}(\mathbf{b}^T_t(i) \mathbf{y})^3} - 3 \mathbf{b}_t(i)$}
    \put(2, 6){\ANTblau{$\mathbf{b}_t^\dagger(i+1) = \mathbf{b}_t^\ddagger(i+1) -\mathbf{B}_t\mathbf{B}_t^T\mathbf{b}_t^\ddagger(i+1)$}}
    \put(2, 5){$\mathbf{b}_t(i+1) = \mathbf{b}_t^\dagger(i+1) / ||\mathbf{b}_t^\dagger(i+1)|| $ \hspace{2cm} $\Leftarrow$ $||\mathbf{b}_t(i+1)||^2\stackrel{!}{=}1$}
    \put(1, 3){\ANTblau{$\mathbf{B}_{t+1} = \left[ \mathbf{B}_t, \, \mathbf{b}_t(9) \right]$}}
    \put(1, 2){$\tilde{c}_t = \mathbf{b}^T_t \mathbf{y}$}
  \end{picture}
\end{huge}
\vfill
\vss}
\newpage
%%% Folie 6 %%%%%%%%%%%%%%%%%%%%%%%%%%%%%%%%%%%%%%%%%%%%%%%%%%%%%%%%%%%%%%%%%%%%
\setbox\titelzeile=\hbox{\Large\sc Leistungsf�higkeit}
\vbox to \vsize{\vss
\vspace*{2cm}
\hbox{{\bf Leistungsf�higkeit}}
\begin{huge}
  \setlength{\unitlength}{1cm}
  \begin{picture}(26,16)
\put(1,15){$2 \times 2$ System, QPSK, $E_b/N_0 = 20 \,\mathrm{dB}$, $L=500$}
\put(0,13){Mixtur}
\put(0,4){\psfig{figure=testsigficajade.m.1.eps,angle=0,width=4cm}}
\put(6,13){fastICA, $i=10$}
\put(6,4){\psfig{figure=testsigficajade.m.2.eps,angle=0,width=4cm}}
\put(12,13){JADE}
\put(12,4){\psfig{figure=testsigficajade.m.3.eps,angle=0,width=4cm}}
\put(0,2){fastICA ben�tigt bei einem $8 \times 12$ System ca. 1/5 der {\sc Matlab}-flops von JADE}
\put(0,1){jedoch: Struktur vom fastICA ist sehr viel einfacher!}
  \end{picture}
\end{huge}
\vfill
\vss}
\newpage
%%% Folie 7 %%%%%%%%%%%%%%%%%%%%%%%%%%%%%%%%%%%%%%%%%%%%%%%%%%%%%%%%%%%%%%%%%%%%
\setbox\titelzeile=\hbox{\Large\sc Leistungsf�higkeit (Vergleich mit JADE)}
\vbox to \vsize{\vss
\vspace*{2cm}
\hbox{{\bf Leistungsf�higkeit} (Vergleich mit JADE)}
\begin{huge}
  \setlength{\unitlength}{1cm}
  \begin{picture}(26,16)
\put(-1,2){\psfig{figure=pltcfastica8x12.m.eps,angle=0,width=17cm}}
    \put(16,14){$\bullet$ $8 \times 12$ System}	      
    \put(16,13){$\bullet$ QPSK}	      
    \put(16,11){$\bullet$ JADE bei geringem SNR}	      
    \put(16,10){\phantom{$\bullet$} g�nstiger}	      
    \put(17, 9){\phantom{$\bullet$} - fastICA "`sieht"' nur ein}	      
    \put(17, 8){\phantom{$\bullet$ -} Ausgangssignal}	      
    \put(17, 7){\phantom{$\bullet$} - JADE ber�cksichtigt auch}	      
    \put(17, 6){\phantom{$\bullet$ -} Kreuzkumulanten}	      
    %\put(0,1){$\bullet$ Umsetzung $P_\mathrm{drop} \rightarrow \mathrm{BER}$ mit Faktor~3}
  \end{picture}
\end{huge}
\vfill
\vss}
\newpage
%%% Folie 8 %%%%%%%%%%%%%%%%%%%%%%%%%%%%%%%%%%%%%%%%%%%%%%%%%%%%%%%%%%%%%%%%%%%%
\setbox\titelzeile=\hbox{\Large\sc Zusammenfassung, Ausblick}
\vbox to \vsize{\vss
\vspace*{1cm}
\hbox{{\bf Zusammenfassung},  Ausblick}
\begin{huge}
  \setlength{\unitlength}{1cm}
  \begin{picture}(26,16)
    \put(2,14){$\bullet$ JADE    $\Longleftrightarrow$ Batch-Algorithmus }
    \put(2,13){$\bullet$ fastICA $\Longleftrightarrow$ \textit{one-unit} Algorithmus}
    \put(2,12){\phantom{$\bullet$ fastICA $\Longleftrightarrow$} Interference Cancellation? Sortierung?}
    \put(2,11){\phantom{$\bullet$} Subraumeinschr�nkung nach jeder Iteration}
    \put(2,10){$\bullet$ ca. 20\% des Rechenaufwands von JADE,}
    \put(2, 9){\phantom{$\bullet$} vergleichbare Leistungsf�higkeit bei hohem SNR}
    \put(2, 8){$\bullet$ Vorteile bei kurzen Blockl�ngen $L$}
  \end{picture}
\end{huge}
\vfill
\vss}
\newpage

%%% EOF %%%%%%%%%%%%%%%%%%%%%%%%%%%%%%%%%%%%%%%%%%%%%%%%%%%%%%%%%%%%%%%%%%%%%%%%
