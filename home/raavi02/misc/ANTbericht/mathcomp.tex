% mathecomp.tex - Mathematische (Mengen)Zeichen, Symbole und Funktionen
%                 fuer LaTeX 2.09 und LaTeX 2e.
%                 Kompatibilitaetsdatei als Anhang von mathe.tex.
%                 
%
%             Original von Dirk Nikolai, 01/96
%             Letzte Aenderung: Nikolai, 25/01/96
%

% #################################################################################################
% Alles folgende ist nur aus Kompatibilitaetsgruenden in mathe.tex eingebunden und sollte deshalb 
% in der Zukunft nicht mehr verwendet werden, weil alle folgenden \newcommand's kuenftig NICHT 
% weiter unterstuetzt werden!!
% #################################################################################################

\newcommand{\realu}[2]{\mathrm{Re}#1\{#2#1\}} % muss statt \real verwendet werden, falls das
                                              % Argument \underbrace o.ae. enthaelt, #1 ist eine
                                              % zulaessige TeX-Groessenangabe, #2 der 
                                              % eigentliche Text

% Definition der mathematischen Zahlenmengen (Symbole fuer die natuerl., reellen etc. Zahlen) (Thomas Boltze)
% Damit sind solche Dinge wie ... \in \R^{m \times n+1} mit kniffligen Definitionen von \R passe.
% Die neuen Befehle lauten z.B. "\In{R}[m][n+1]" oder "\In{C}[n]"
\newcommand{\C}{\mathrm C \hspace{-0.43em}\rule[0.1ex]{0.02em}{1.45ex}\hspace{0.58em}}
\newcommand{\R}{\mathrm R \hspace{-0.44em}\rule[0.05ex]{0.02em}{1.45ex}\hspace{0.5em}}
\newcommand{\N}{\mathrm N \hspace{-0.44em}\rule[0.05ex]{0.02em}{1.45ex}\hspace{0.5em}}
\newcommand{\Z}{\mathrm Z \hspace{-0.44em}/\hspace{0.58em}}
\mathchardef\inn="3232                    % Verr"ucken des \in Zeichens nach oben
\def\in{\raisebox{0.2ex}{$\inn$}}
\catcode`\@=11                            % ein wenig Tex-Gewurstel f"ur die optionalen Parameter:
\def\room#1{\@ifnextchar[{\@nameuse{#1}\@roomarg}{\@nameuse{#1}}}
\def\@roomarg[#1]{\@ifnextchar[{\@@roomarg{#1}}{^#1}}
\def\@@roomarg#1[#2]{^{#1 \times #2}}
\catcode`\@=12
\def\In#1{\raisebox{0.2ex}{$\inn$} \room{#1}}

% \newcommand{\matC}{\makebox[.6em]{\makebox[-.18em]{C}\rule{.03em}{1.5ex}}}
\newcommand{\matC}{\mathrm C \hspace{-0.43em}\rule[0.1ex]{0.02em}{1.45ex}\hspace{0.58em}}
\newcommand{\matRR}{\mathrm R \hspace{-0.44em}\rule[0.05ex]{0.02em}{1.45ex}\hspace{0.5em}}

\newcommand{\matN}{\mathsf{I\!N}}         % Setze die math. Sonderzeichen zusammen
\newcommand{\matR}{\mathsf{I\!R}}         % aus Sans Serif-Zeichen.
\newcommand{\matZ}{\mathsf{Z\!\!\!\!\:Z}} % Problem:  "C" fehlt.


\newcommand{\nullvec}{\mathbf{0}}   % (fette) Vektoren oder Matrizen (jeder einzeln...)
\newcommand{\avec}{\mathbf{a}}
\newcommand{\Avec}{\mathbf{A}}
\newcommand{\bvec}{\mathbf{b}}
\newcommand{\Bvec}{\mathbf{B}}
\newcommand{\cvec}{\mathbf{c}}
\newcommand{\Cvec}{\mathbf{C}}
\newcommand{\dvec}{\mathbf{d}}
\newcommand{\Dvec}{\mathbf{D}}
\newcommand{\evec}{\mathbf{e}}
\newcommand{\Evec}{\mathbf{E}}
\newcommand{\fvec}{\mathbf{f}}
\newcommand{\Fvec}{\mathbf{F}}
\newcommand{\gvec}{\mathbf{g}}
\newcommand{\Gvec}{\mathbf{G}}
\newcommand{\hvec}{\mathbf{h}}
\newcommand{\Hvec}{\mathbf{H}}
\newcommand{\ivec}{\mathbf{i}}
\newcommand{\Ivec}{\mathbf{I}}
\newcommand{\lvec}{\mathrm{\mathbf $\ell$}}
\newcommand{\Mvec}{\mathbf{M}}
\newcommand{\Nvec}{\mathbf{N}}
\newcommand{\Pvec}{\mathbf{P}}
\newcommand{\Qvec}{\mathbf{Q}}
\newcommand{\rvec}{\mathbf{r}}
\newcommand{\Rvec}{\mathbf{R}}
\newcommand{\svec}{\mathbf{s}}
\newcommand{\Svec}{\mathbf{S}}
\newcommand{\Tvec}{\mathbf{T}}
\newcommand{\uvec}{\mathbf{u}}
\newcommand{\Uvec}{\mathbf{U}}
\newcommand{\vvec}{\mathbf{v}}
\newcommand{\Vvec}{\mathbf{V}}
\newcommand{\wvec}{\mathbf{w}}
\newcommand{\Wvec}{\mathbf{W}}
\newcommand{\xvec}{\mathbf{x}}
\newcommand{\Xvec}{\mathbf{X}}
\newcommand{\yvec}{\mathbf{y}}
\newcommand{\Yvec}{\mathbf{Y}}
\newcommand{\zvec}{\mathbf{z}}
\newcommand{\Zvec}{\mathbf{Z}}


% EOF
