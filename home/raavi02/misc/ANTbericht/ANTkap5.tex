\chapter{Cleveres} \label{CPTsim} 
%
\section{Symbolische Verweise} \label{SECverw}
%
Wer Verweise auf Seitennummern nach dem Muster ``wie man auf Seite 35 sieht'' 
{\em fest verdrahtet}, wird sich ziemlich bald dar"uber "argern, da"s er alle derartigen 
Verweise ab"andern mu"s, nur weil er {\em vor} Seite~35 Text eingef"ugt hat, 
so da"s sich alles folgende entsprechend verschiebt.

Deshalb fangen kluge Leute einen derartigen Pfusch gleich gar nicht an
und verwenden prinzipiell {\em symbolische Verweise}, d.h.\ Verweise nicht direkt
auf eine Zahl, sondern auf ein {\em label}. Dies gilt nicht nur f"ur Seitennummern,
sondern auch f"ur Kapitel-, Gleichungs-, Tabellen- und Bildernummern.
In \LaTeX\ steht dazu der Befehl \verb+\ref{LABELNAME}+ zur Verf"ugung, 
wobei dies voraussetzt, da"s man an der entsprechenden Stelle den Label {\tt LABELNAME}
mittels \verb+\label{LABELNAME}+ definiert hat.

Beispiel: Nehmen wir an, die Zeile \verb+\section{Fazit} \label{SECfazit}+
steht in einem \LaTeX-Dokument und erzeugt die "Uberschrift ``5.4 Fazit''.
Der symbolische Verweis auf Abschnitt~5.4 sieht dann so aus:\quad
\verb+wie in Abschnitt~\ref{SECfazit} gezeigt wurde+. Dabei stellt die
Tilde sicher, da"s \LaTeX\ keinen Zeilenumbruch zwischen {\tt Abschnitt} und
{\tt 5.4} vornimmt.

"Aquivalentes gilt nat"urlich auch f"ur \verb+\chapter{...}+, \verb+\subsection{...}+ etc..
Prinzipiell mu"s der \verb+\label+-Befehl immer direkt nach dem Anfang der entsprechenden
Umgebung stehen, bei Gleichungen also nach \verb+\begin{equation}+ oder der entsprechenden
Konstruktion, bei Tabellen nach \verb+\begin{table}+. Ausnahme: bei Abbildungen mu"s
der \verb+\label+-Befehl nach \verb+\caption{...}+ und vor \verb+\end{figure}+ stehen.
F"ur Verweise auf Seitennummern gibt es den speziellen Befehl \verb+\pageref{LABELNAME}+.
Der dazugeh"orige \verb+\label+-Befehl kann irgendwo im \LaTeX-Dokument zwischen 
\verb+\begin{document}+ und \verb+\end{document}+ stehen.

N.B.: Damit \LaTeX\ die symbolischen Verweise richtig aufl"ost,
mu"s zweimal kompiliert werden! Beim ersten Mal werden die Labels erzeugt und
beim zweiten Mal von \verb+\ref{...}+ ausgelesen. Kompiliert man nicht oft genug,
so erscheinen {\tt ??} statt der gew"unschten Referenzen.

Bemerkung: die einzigen Verweise, die nicht mit dem \verb+\label+- 
und \verb+\ref+-Befehlen bewerkstelligt werden, sind Referenzen auf die Literaturliste
(Bibliographie). Dies geschieht mit {\tt bibtex} und dem Befehl \verb+\cite{...}+
und ist in Abschnitt~\ref{SECbib} beschrieben.

\section{Literaturverzeichnis mit BibTeX} \label{SECbib}
%
Verweise auf das Literatur-Verzeichnis erfolgen mit dem
\verb+\cite{QUELLExx}+-Befehl, wobei {\tt QUELLExx} ein Label ist, das in der
Datei {\tt quellen.bib} definiert sein mu"s.  In dieser Datei werden
solchen Labeln alle Angaben zugewiesen (Autoren, Titel, Journal etc.).

Die Default-Datei enth"alt bereits einige Labels: {\tt KK92}, {\tt Kam92}, {\tt Pro95}.
Es hat sich als zweckm"a"sig erwiesen, die Labels bei nur einem Autor aus den drei Anfangsbuchstaben
des Nachnamens und der Jahreszahl zusammenzusetzen. Bei mehreren Autoren benutzt man die
Initialen der Nachnamen und die Jahreszahl.
Verwendet man \verb+\bibliographystyle{alpha}+ (wie es im ANT-Musterbericht der Fall ist),
so erzeugt der \verb+\cite+-Befehl Verweise in der Form \cite{KK92,Kam92} oder \cite{Pro95},
sofern das \LaTeX-Dokument einmal mit Bibtex ``behandelt'' und dann zweimal compiliert wurde.

% EOF
