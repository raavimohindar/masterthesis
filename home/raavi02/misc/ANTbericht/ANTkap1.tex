\chapter{Einleitung} \label{CPTeinleit}
%
Dieses Kapitel dient einer ersten Erl"auterung, was mit diesem 
Musterbericht beabsichtigt ist. Jeder Student,
der im Arbeitsbereich Nachrichtentechnik (ANT) einen Bericht mit
dem wissenschaftlichen Textverarbeitungsprogramm \LaTeX\ anfertigen
m"ochte, kann diesen Musterbericht als Vorlage verwenden. Dadurch wird 
zum einen viel Arbeit gespart, zum anderen "ahneln sich die gesamten
Arbeiten dann in ihrer Aufmachung und haben daher einen gewissen 
Wiedererkennungswert.

\section{Wieso gerade \LaTeX?}
%
Im Gegensatz zu allen anderen Textverarbeitungssystemen bietet
\LaTeX\ die folgenden vier Hauptvorteile:
%
\begin{enumerate}
\item Es ist {\em public domain} und damit (als source code) frei kopierbar.
\item Es ist f"ur (fast) alle Betriebssysteme verf"ugbar.
\item Bei der Darstellung von mathematischen Zusammenh"angen ist es ungeschlagen.
\item Es wird hervorragend gewartet und erweitert.
\end{enumerate}


\section{Ist aller \LaTeX-Anfang schwer?} \label{SECallg}
%
Diese Frage k"onnte just in diesem Augenblick im Mittelpunkt stehen.
Da man \LaTeX\ am besten anhand eines Beispiel-Files lernt, versorgen
wir uns zun"achst einmal mit einem Musterdokument. Wieso sollen wir
nicht genau diesen Musterbericht dazu nehmen?
Alle Dateien, die f"ur den Musterbericht notwendig sind, k"onnen von jedem
User der ANT-Workstations mit dem Befehl {\tt get} in sein Home-Verzeichnis 
kopiert werden. Dabei wird im \LaTeX-Verzeichnis ein Unterdirectory namens
{\tt ANTbericht} angelegt und einige Files hineinkopiert.
Dabei ist die Datei {\tt ANTbericht.tex} das {\em top level}\/ \LaTeX\ File,
das alle anderen Dateien einbindet.

Ist das erledigt, so stellt sich die Frage, wie man nun mit \LaTeX\ arbeitet.
Wann mu"s ich welches Programm aufrufen, um irgendwas zu machen?
Zum Gl"uck steht dazu das sogenannte {\tt LaTeX-Tool} von Dieter Boss
zur Verf"ugung\footnote{Diese Passage hat Dirk Nikolai geschrieben, und nicht
der Gelobte!}, das einem viel unn"otige Tipparbeit erspart und
dar"uber hinaus auch weniger versierten \LaTeX-Benutzern erlaubt, 
sofort einzusteigen. Vielleicht hast Du es auch in diesem Moment, in dem 
Du diese Zeilen liest, in Gebrauch.

Falls Du es noch nicht kennen 
solltest, tippe einfach in einer x-beliebigen Shell {\tt lat FILENAME} ein,
wobei {\tt FILENAME.tex} ein \LaTeX-File im aktuellen Verzeichnis sein mu"s
(z.B.\ {\tt ANTbericht.tex}).
Mit der Option {\tt `c'} wie {\em compile}\/ wird dieses File kompiliert 
und mit {\tt `v'} wie {\em view}\/ angezeigt. Mehr mu"s man im Moment nicht wissen.
Spiel' mit dem Tool einfach ein wenig herum. Kaputtmachen kann man dabei
kaum etwas -- zumindest hat das bisher noch keiner geschafft.
Sollten trotzdem Probleme auftreten, frage einfach eine Person um Rat, 
die sich schon etwas besser auskennt, z.B.\ einen Kommilitonen, der schon
l"anger ``dabei ist''.


\section{Sinn und Zweck des ANT Musterberichts}
%
Dieser Musterbericht verfolgt ein doppeltes Ziel:
%
\begin{enumerate}
\item Er liefert --~wie der Name schon andeutet~-- ein Muster f"ur Berichte "uber
      Diplom-, Studien- und Projektarbeiten. Eigentlich mu"s der (k"unftige)
      Berichtverfasser (nur noch) diese \LaTeX-Files entsprechend anpassen.
\item Weiterhin liefert der Musterbericht eine extrem kurze Einf"uhrung in \LaTeX.
      Jedoch soll hier ``nur das unbedingt Erforderliche'' erkl"art werden bzw.\ das,
      was "ublicherweise nicht in B"uchern steht, man aber aus Erfahrung lernt.
      Dabei wird {\em Learning by Doing} unterst"utzt: durch das
      \underline{gleichzeitige} Betrachten beispielsweise dieser Textstelle
      in der ASCII-Datei {\tt ANTkap1.tex} und der entsprechenden Stelle
      im fertigen Dokument sollte z.B.\ die Wirkung des \LaTeX-Befehls 
      \verb+\underline{...}+ deutlich werden.

Eine {\em umfassende} Darstellung von \LaTeX-Befehlen, die ein Lesen
weiterer Dokumente zum Thema "uberfl"ussig machen w"urde, ist nicht
beabsichtigt.
Eine ausf"uhrlichere Einf"uhrung in \LaTeX\ sowie weitere Manuals stehen im Software-Labor
zur Einsicht bereit und k"onnen auch mittels {\tt help}-Kommando {\em on-line}
auf den Bildschirm geholt werden. Sollte es erforderlich sein, so k"onnen auch
B"ucher zu diesem Thema ausgeliehen werden (c/o Betreuer).
\end{enumerate}


\section{Fragen und Anregungen zum Musterbericht}
%
Obwohl den Studenten mit diesem Musterbericht Arbeit abgenommen werden soll,
kann (und soll) nicht verhindert werden, da"s diese sich dar"uber hinaus 
mit dem Thema \LaTeX\ besch"aftigen\footnote{Zugegeben: 
Ein bi"schen kompliziert erscheint \LaTeX\ auf den ersten Blick ja
schon...}. Hierbei sollte ein Nachschlagen in der einschl"agigen
Literatur helfen, weiterhin k"onnen Kommilitonen, notfalls auch der
Betreuer oder andere Mitarbeiter zu Rate gezogen werden.

Falls einige Sachverhalte, die durch dieses Dokument erl"autert bzw.\
abgedeckt werden sollen, nur sehr nebul"os verstanden werden, so macht
das zun"achst nichts. Schlimmer ist es dagegen, wenn man durch ein
schlecht gemachtes Beispiel auf den Holzweg gebracht wird, man ohne
eine Erkl"arung also besser dagestanden h"atte. Tauchen aus Sicht eines
Anwenders solche didaktischen Fehltritte auf, so hilft es wenig, dies
in dieser Kopie des Originals zu verbessern --- die n"achsten Anwender
st"unden dann wahrscheinlich wieder vor dem gleichen Problem. Also: Dem
Autor, den Autoren bzw.\ den sonstwie Verantwortlichen dieses Textes
``Bescheid'' sagen und am besten gleich den Verbesserungsvorschlag
mitbringen! Die Eingaben werden dann wohlwollend gepr"uft und ggf.\ die
entsprechenden "Anderungen eingebracht.

"Anderungsvorschl"age z.Z.\ bitte an:
\vspace*{-1em}
\begin{tabbing}
  Dieter Boss\quad  \= $\Rightarrow$ \= {\tt boss@comm.uni-bremen.de} \\
  Dirk Nikolai      \> $\Rightarrow$ \> {\tt nikolai@comm.uni-bremen.de} 
\end{tabbing}


% EOF

