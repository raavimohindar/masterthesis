\chapter{Prinzipielles} \label{CPTerst}
%
\section{Deutsche Umlaute und G"ansef"u"schen}
%
Sofern man keine Umlaute auf der Tastatur hat, m"ussen Umlaute in \LaTeX\ so 
eingegeben werden:

\qquad "a durch \verb+"a+,\quad "o durch \verb+"o+,\quad "u durch \verb+"u+,\quad "s durch \verb+"s+.

Die gro"sen Umlaute funktionieren entsprechend. Damit dies alles so klappt,
mu"s "ubrigens der {\tt german} style eingebunden worden sein (was hier der Fall ist).

Verwendet man die Umlaute auf der deutschen Tastatur(in Latin-1 Codierung), so
mu� �brigens das Paket \verb+inputenc+ mit der Option \verb+latin1+
eingebunden werden.\\
\verb+\usepackage[latin1]{inputenc}+


Was die G"ansef"u"schen angeht, so stellt sich zun"achst die Frage, wie man sie gerne haben will.
Zieht man ``diese'' Form vor (d.h.\ vorne die 66er-, hinten die 99er-Gestalt),
so mu"s man das in \LaTeX\ mit dem doppelten {\em accent grave} bzw.\ Apostroph formulieren: \verb+``Zitat''+.

Die "`richtigen"' deutschen G"ansehaxen (vorne unten die 99er-, hinten oben die 66er-Gestalt)
sehen allerdings "`so"' aus und sind noch komplizierter einzutippen: \verb+"`Zitat"'+ .

A vous de choisir! Nur von den doppelten Anf"uhrungszeichen auf der Tastatur ist abzuraten, 
da sie mit der oben erw"ahnten Umlaut-Konstruktion kollidieren und au"serdem vorne und hinten die 
gleiche Gestalt produzieren.


\section{Schriftstile und -gr"o"sen}
%
{\bf Fettdruck} erreicht man mittels des \verb+\bf+-Kommandos (engl.\ {\em boldface}):
\verb+{\bf Fettdruck}+. Analog dazu kann man Text auch durch
{\em Kursivschrift} hervorheben: \verb+{\em Kursivschrift}+.
Mit \verb+\tt+ wie {\em teletype} erreicht man die 
{\tt Schreibmaschinenschrift} und mit \verb+\sc+ eine Schriftart, die
sich {\sc Small Capitals} nennt.
Diese (und weitere) Schriftstile stehen allerdings nur im ``Textmodus''
zur Verf"ugung, nicht aber im ``Mathemodus'' (s.~Abschnitt~\ref{CPTmath}).

L"angere Textpassagen kann man z.B.\ durch
%
\begin{verbatim}
\begin{bf}
Lange Textpassage im Fettdruck ...
\end{bf}
\end{verbatim}
%
in den gew"unschten Stil bringen.

Die Schrift{\em gr"o"sen} lauten --~in kleinerwerdender Reihenfolge: 
{\tt Huge, huge, LARGE, Large, large, normalsize, small, footnotesize, tiny}.
Auch sie sind in der Form \verb+{\large xxx}+ sowie mit \verb+\begin{large}+
und \verb+\end{large}+ anzusprechen.

\section{Leerzeichen (Spaces), Abst"ande zwischen Worten}
%
\LaTeX\ behandelt 10 aufeinanderfolgende Leerzeichen im ``\verb+*.tex+''-File 
genauso, als ob man nur ein einziges getippt h"atte. Will man l"angere horizonale 
Abst"ande erzwingen, so stehen dazu die Befehle \verb+\hspace{...}+, \verb+\quad+,
\verb+\qquad+ etc.\ zur Verf"ugung.

Innerhalb eines Absatzes generiert \LaTeX\ einen sch"onen rechten Rand (Blocksatz),
indem die Abst"ande zwischen den Worten k"unstlich gestreckt oder gestaucht 
werden.
Dabei streckt \LaTeX\ die Zwischenr"aume nach Satzzeichen mehr als die anderen, so
da"s sich z.B.~nach einem Punkt ein l"angerer Zwischenraum ergibt. Allerdings kann
\LaTeX\ nicht wissen, welcher Punkt ein Satzende darstellt und welcher nicht.
Deshalb mu"s man \LaTeX\ mitteilen, welche Zwischenr"aume nicht gestreckt werden d"urfen.
Dies kann dadurch geschehen, da"s man ein normales Leerzeichen ersetzt durch
``\verb+\ +'' (Backslash, Space) oder ``\verb+~+'' (Tilde). Beide Befehle bewirken, da"s 
dieses Leerzeichen wie ein Abstand zwischen Worten ohne Satzzeichen behandelt wird.
Die Tilde verhindert au"serdem, da"s ein Zeilenumbruch an diese Stelle gelegt wird
und ist deshalb in Konstruktionen wie \verb+siehe Kap.~5+ oder \verb+Dr.~Mustermann+ anzuraten.

N.B.: Am besten, man merkt sich {\em ein f"ur allemal}, bestimmte Konstruktionen so
einzutippen:
%
  \begin{center}
  \begin{tabular}{|c|c|}
    \hline
    Gew"unschter Output & Optimale \LaTeX-Notation \\
    \hline
    \hline
    d.h.\ blabla & \verb+d.h.\ blabla+ \\
    z.B.\ blabla & \verb+z.B.\ blabla+ \\
    bzw.\ blabla & \verb+bzw.\ blabla+ \\
    \hline
  \end{tabular}
  \end{center}
%
Auch im Zusammenhang mit Referenzen (s.~Kap.~\ref{SECverw}) sollte man prinzipiell 
folgende Notation w"ahlen:
\verb+aus Kap.~\ref{label1} folgt...+\ oder:\ 
\verb+wie in Gl.~(\ref{label2}) gezeigt...+


\section{Strukturierung eines Dokumentes} \label{SECunter}
%
Die "Uberschrift zu Kapitel~\ref{CPTerst} wurde durch den Befehl
\verb+\chapter{Prinzipielles}+ erzeugt. Dieser Befehl erzeugt auch
die automatische fortlaufende Kapitel-Numerierung. Abschnitte und Unterabschnitte
kommen durch die Verwendung der Befehle \verb+\section{...}+ (s.\ z.B.\ Abschnitt~\ref{SECunter}), 
\verb+\subsection{...}+ und \verb+\subsubsection{...}+ zustande.
Die Eintr"age aller dieser Strukturkommandos werden automatisch ins
Inhaltsverzeichnis "ubernommen.

\section{Aufz"ahlungen} \label{SECaufz}
%
Aufz"ahlung in der folgenden Form,
%
\begin{itemize}
\item d.h.\ mit einer Einr"uckung und kleinen P"unktchen am Anfang ...
\item ... jedes Eintrages
\end{itemize}
%
k"onnen mit der \verb+itemize+-Umgebung erreicht werden:
%
\begin{verbatim}
\begin{itemize}
\item d.h.\ mit einer Einr"uckung und kleinen P"unktchen am Anfang ...
\item ... jedes Eintrages
\end{itemize}
\end{verbatim}
%
Auch innerhalb eines \verb+\item+ darf wieder eine \verb+itemize+-Umgebung
stehen, wobei dann kleine Striche statt P"unktchen erscheinen.

Will man die einzelnen Eintr"age mit laufenden Nummern statt mit P"unktchen
versehen, so ist die \verb+enumerate+-Umgebung zu verwenden.

\section{Tabellen} \label{SECtab}
%
Tabelle~\ref{TABsuper} zeigt ein paar wesentliche Elemente, mit
denen man einfache Tabellen produzieren kann. Die {\tt tabular}-Umgebung
erh"alt dabei das Argument \verb+|l|rc|+, was bedeutet, da"s die Tabelle
drei Spalten haben soll, wobei die Inhalte in der ersten Spalte linksb"undig,
in der zweiten rechtsb"undig und in der dritten Spalte zentriert erscheinen 
sollen. Die senkrechten Striche deuten an, welche Spalten durch Striche
voneinander getrennt werden sollen.
%
\begin{verbatim}
\begin{table}[htb]
  \begin{center}
  \begin{tabular}{|l|rc|}
    \hline
    aa   & bb    & cc \\
    \hline
    \hline
    a    & b     & c  \\
    left & right & center \\
    aa   & !!    & ?? \\
    \hline
  \end{tabular}
  \end{center}
  \caption{Mustertabelle} \label{TABsuper}
\end{table}
\end{verbatim}
%
\begin{table}[htb]
  \begin{center}
  \begin{tabular}{|l|rc|}  % 1. Spalte linksbuendig, 2. rechtsbuendig, 3. Spalte zentriert
    \hline
    aa   & bb    & cc \\
    \hline
    \hline
    a    & b     & c  \\
    left & right & center \\
    aa   & !!    & ?? \\
    \hline
  \end{tabular}
  \end{center}
  \caption{Mustertabelle} \label{TABsuper}
\end{table}

In den Tabelleneintr"agen selbst trennt das Kaufmanns-Und (``\verb+&+'')
die Spalten voneinander. Jede Zeile mit Eintr"agen
mu"s mit einem Zeilenumbruch (``\verb+\\+'') abgeschlossen werden.

{\bf Tabellen-Plazierung:} Da die {\tt table}-Umgebung ein {\em floating object} 
definiert\footnote{Genau so wie die {\tt figure}-Umgebung, s.\ Abschnitt~\ref{SUBSECbilder}.},
das frei beweglich im Text ``schwimmt''\footnote{L"a"st man im \LaTeX-File direkt vor und
nach der {\tt table}-Umgebung keine Leerzeile, so ``scrollt'' der Text um die Tabelle herum.} 
und je nach aktueller Lage der Seitenumbr"uche an einer anderen Stelle landen kann, kann man der 
{\tt table}-Umgebung einen Plazierungswunsch mitgeben: \verb+[htb]+ bedeutet, da"s die Tabelle
vorzugsweise da erscheinen soll, wo sie im \LaTeX-File auftritt ({\tt `h'} wie {\em here}),
andernfalls oben {\em (top)} auf der Seite, notfalls unten {\em (bottom)}.
Allerdings haben schon viele Leute dar"uber geflucht, da"s \LaTeX\ sich ziemlich
oft "uber die W"unsche seiner User hinwegsetzt, insbesondere bei gro"sen Gleitobjekten. 

An dieser Stelle sei angemerkt, da"s es sich als sehr vorteilhaft erweisen
wird, auch das \LaTeX-File einigerma"sen sauber zu formatieren.
Obwohl es \LaTeX\ ``egal'' ist, wie man den Text in die Datei hackt, 
sollten die folgenden Beispiele f"ur sich sprechen. Sowohl der folgende
Ausschnitt
%
\begin{verbatim}
 \begin{table}[htb]\begin{center}\begin{tabular}{|l|r|}\hline links
&rechts     \\\hline        \hline 12    &  13  \\$  \pi  $   &   $
\pi/2$           \\  \hline        \end{tabular}       \end{center}
\caption{Mustertabelle}\end{table}
\end{verbatim}
%
als auch die besser formatierte Version
%
\begin{verbatim}
\begin{table}[htb]
  \begin{center}
  \begin{tabular}{|l|r|}
    \hline
    links & rechts  \\
    \hline
    \hline
    12    & 13      \\
    $\pi$ & $\pi/2$ \\
    \hline
  \end{tabular}
  \end{center}
\caption{Mustertabelle}
\end{table}
\end{verbatim}
%
liefern die folgende Tabelle:
%
\begin{table}[htb]
  \begin{center}
  \begin{tabular}{|l|r|} \hline
    links & rechts  \\   \hline \hline
    12    & 13      \\
    $\pi$ & $\pi/2$ \\   \hline
  \end{tabular}
  \end{center}
\caption{Mustertabelle}
\end{table}

Nur, wehe in der ersten Version hat man sich mit den Tabulatorzeichen (``\verb+&+'')
verz"ahlt \dots


