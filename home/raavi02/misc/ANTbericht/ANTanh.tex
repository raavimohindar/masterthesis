\begin{appendix}
\chapter{Einbinden von ASCII-Dateien} \label{CPTapp}
\thispagestyle{fancyplain}
%
Eine h"aufig auftretende Aufgabe bei der Dokumentation von studentischen
Ausarbeitungen ist das Aufnehmen von Programmlistings in das Dokument.
Solche als ASCII-Dateien vorliegenden Texte k"onnen komfortabel in 
\LaTeX-Dokumente eingebunden werden und m"ussen nicht etwa losgel"ost 
vom "ubrigen Text als Kopien beigelegt werden! Die Problematik um
dieses Thema wird h"aufig auch als ``Simulieren getippter Texte'' 
bezeichnet.

Folgenderma"sen bindet man ASCII-Dateien, z.B.\ {\sc Matlab}-m-Files oder 
auch diese Datei selbst, als reinen Text in \LaTeX\ ein: \verb+\getASCII{ANTanh.tex}+

Die Wirkung ist dann folgende:
\getASCII{ANTanh.tex}


\chapter{Die Datei {\tt mathe.tex}} \label{CPTmehrascii}
%
Ein weiteres Beispiel einer eingebundenen ASCII-Datei zeigt die 
z.Z.\ aktuelle, (hoffentlich) ber"uhmte Datei {\tt mathe.tex}:
\getASCII{mathe.tex}

\chapter{Zum Einbinden kurzer ASCII-Textstellen}
%
Ein alternative Methode des Einbindens besteht in der {\tt verbatim}-Umgebung.
Setzt man den folgenden Text zwischen \verb+\begin{verbatim}+ und
\verb+\end{verbatim}+, so erscheint er wie folgt:
%
\begin{verbatim}
Ein Satz, wie er 1:1 erscheint.
  Ein Satz, wie er 1:1 erscheint.
    Ein Satz, wie er 1:1 erscheint.
 Alle Leerzeichen tauchen also auf.
Alle  Leerzeichen tauchen also auf.
Alle Leerzeichen  tauchen also auf.

Sonderzeichen sind hier kein Problem: # $ ^ \ \\ " @  (u.v.m.)
\end{verbatim}

\end{appendix}
